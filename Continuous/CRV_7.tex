

%----------------------------------------------------------------------------------------------------%
\begin{frame}
\frametitle{Functions and Definite integrals}
\large
Integration is not part of the syllabus, and it is assumed that students have are not familiar with how to compute definite integrals.\\ \bigskip
 However,  it is useful to know what the purpose of definite integrals are, because we will be using the results derived from definite integrals. \\ It is assumed that students are familiar with functions.

\end{frame}
%----------------------------------------------------------------------------------------------------%
\begin{frame}
\frametitle{Functions}

IMAGE : 5A Functions

\end{frame}
%----------------------------------------------------------------------------------------------------%
\begin{frame}
\frametitle{Definite Integral}

IMAGES : 5A Definite Integrals

\end{frame}
%----------------------------------------------------------------------------------------------------%
\begin{frame}
\frametitle{Definite Integral}

Definite integrals are used to compute the "area under curves". 

The area under the curve between X=1 and  X=2 is depicted in grey. Using definite integrals

\end{frame}

%----------------------------------------------------------------------------------------------------%
\frame{
\large
\frametitle{Probability Density Function}
The probability density function of a continuous random variable is a function which can be integrated to obtain the probability that the random variable takes a value in a given interval.
\\
\bigskip
In probability theory, a probability density function (pdf), or density of a continuous random variable is a function that describes the relative likelihood for this random variable to occur at a given point. 
\\
\bigskip
The probability for the random variable to fall within a particular region is given by the integral of this variable’s density over the region. The probability density function is non-negative everywhere, and its integral over the entire space is equal to one.
}

%----------------------------------------------------------------------------------------------------%
\frame{
\frametitle{Probability Mass Function}
\large
\begin{itemize} \item a probability mass function (pmf) is a function that gives the probability that a discrete random variable is exactly equal to some value. \item The probability mass function is often the primary means of defining a discrete probability distribution \end{itemize}
}




