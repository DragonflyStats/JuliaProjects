\documentclass[12pt, a4paper]{report}
\usepackage{epsfig}
\usepackage{subfigure}
%\usepackage{amscd}
\usepackage{amssymb}
\usepackage{graphicx}
%\usepackage{amscd}

\usepackage{subfiles}
\usepackage{framed}
\usepackage{subfiles}
\usepackage{amsthm, amsmath}
\usepackage{amsbsy}
\usepackage{framed}
\usepackage[usenames]{color}
\usepackage{listings}
\lstset{% general command to set parameter(s)
	basicstyle=\small, % print whole listing small
	keywordstyle=\color{red}\itshape,
	% underlined bold black keywords
	commentstyle=\color{blue}, % white comments
	stringstyle=\ttfamily, % typewriter type for strings
	showstringspaces=false,
	numbers=left, numberstyle=\tiny, stepnumber=1, numbersep=5pt, %
	frame=shadowbox,
	rulesepcolor=\color{black},
	columns=fullflexible
} %
%\usepackage[dvips]{graphicx}
\usepackage{natbib}
\bibliographystyle{chicago}
\usepackage{vmargin}
% left top textwidth textheight headheight
% headsep footheight footskip
\setmargins{3.0cm}{2.5cm}{15.5 cm}{22cm}{0.5cm}{0cm}{1cm}{1cm}
\renewcommand{\baselinestretch}{1.5}
\pagenumbering{arabic}
%\theoremstyle{plain}
\newtheorem{theorem}{Theorem}[section]
\newtheorem{corollary}[theorem]{Corollary}
\newtheorem{ill}[theorem]{Example}
\newtheorem{lemma}[theorem]{Lemma}
\newtheorem{proposition}[theorem]{Proposition}
\newtheorem{conjecture}[theorem]{Conjecture}
\newtheorem{axiom}{Axiom}
\theoremstyle{definition}
\newtheorem{definition}{Definition}[section]
\newtheorem{notation}{Notation}
\theoremstyle{remark}
\newtheorem{remark}{Remark}[section]
\newtheorem{example}{Example}[section]
\renewcommand{\thenotation}{}
%\renewcommand{\thetable}{\thesection.\arabic{table}}
%\renewcommand{\thefigure}{\thesection.\arabic{figure}}

\author{ } \date{ }

\begin{document}
	
	The Exponential and Uniform Distributions
	
	The Exponential Distribution
	The Uniform Disribution
	
	%============================================================ %
	
	Continuous Uniform distribution
	
	\begin{description}
		\item[L] lower bound of an interval
		\item[U] upper bound of an interval
	\end{description}
	Probability of an outcome being between lower bound L and upper bound U
	\[P( L \leq X \leq U)  =  { U - L \over  b – a }\]
	
	%============================================================ %
	
	
	Reminder
	
	" $\leq$" is less than or equal to
	
	" $\geq$" is greater than or equal to
	
	
	\[L \leq X \leq U\]
	
	simply states that X is between L and U inclusively.
	
	("inclusively" mean that X could be exactly L or U also, although the probability of this is extremely low)
	
	
	%============================================================ %
	
	The probability density function is given as
	
	\[f(x) = {1 \over b-a} for a \leq x \leq b\]
	
	For any value "c" between the minimum value a and the maximum value b
	
	\[P(X \ geq c) = {b-c \over b-a}\]
	
	here b is the upper bound while c is the lower bound
	
	
	\[P(X \ leq c) = {c-a \over b-a}\]
	
	here c is the upper bound while a is the lower bound
	
	
\end{document}
