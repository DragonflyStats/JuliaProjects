





%
%
%\subsection{Binomial}
%For the binomial distribution, if the probability of success is greater than 0.5, instead of considering the number of successes, to use the table we consider the number of failures.


 


% Cumulative Distribution Function - Definition (probability mass function)
% Changing the unit space.
% Poisson Example (MA4102)
% Key Rules
% Distinguishing between binomial and poisson.
% Poisson Approximation of Binomial
%

% Last class : Cumulative Tables tables
% Formal Definition
% Sample Space and Partitioning






%%%%%%%%%%%%%%%%%%%%%%%%%%%%%%%%%%%%%%%%%%%%%%%%%%%%%%%%%%%%%%%%%%


\documentclass[a4paper,12pt]{article}
%%%%%%%%%%%%%%%%%%%%%%%%%%%%%%%%%%%%%%%%%%%%%%%%%%%%%%%%%%%%%%%%%%%%%%%%%%%%%%%%%%%%%%%%%%%%%%%%%%%%%%%%%%%%%%%%%%%%%%%%%%%%%%%%%%%%%%%%%%%%%%%%%%%%%%%%%%%%%%%%%%%%%%%%%%%%%%%%%%%%%%%%%%%%%%%%%%%%%%%%%%%%%%%%%%%%%%%%%%%%%%%%%%%%%%%%%%%%%%%%%%%%%%%%%%%%
\usepackage{eurosym}
\usepackage{vmargin}
\usepackage{amsmath}
\usepackage{framed}
\usepackage{graphics}
\usepackage{epsfig}
\usepackage{subfigure}
\usepackage{enumerate}
\usepackage{fancyhdr}

\setcounter{MaxMatrixCols}{10}
%TCIDATA{OutputFilter=LATEX.DLL}
%TCIDATA{Version=5.00.0.2570}
%TCIDATA{<META NAME="SaveForMode"CONTENT="1">}
%TCIDATA{LastRevised=Wednesday, February 23, 201113:24:34}
%TCIDATA{<META NAME="GraphicsSave" CONTENT="32">}
%TCIDATA{Language=American English}

\pagestyle{fancy}
\setmarginsrb{20mm}{0mm}{20mm}{25mm}{12mm}{11mm}{0mm}{11mm}
\lhead{MS4222} \rhead{Kevin O'Brien} \chead{Binomial Distribution} %\input{tcilatex}
\begin{document}
\section{Poisson Approximation of the Binomial}
\begin{itemize}
\item The Poisson distribution can sometimes be used to approximate the binomial distribution
\item When the number of observations n is large, and the success probability p is small, the $\mbox{Bin}(n,p)$ distribution approaches the Poisson distribution 
with the parameter given by $m = np$.
\item This is useful since the computations involved in calculating binomial probabilities are greatly reduced.
\item As a rule of thumb, n should be greater than 50 with p very small, such that $np$ should be less than 5.
\item If the value of $p$ is very high, the definition of what constitutes a ``success" or ``failure" can be switched.
\end{itemize}





\end{document}



