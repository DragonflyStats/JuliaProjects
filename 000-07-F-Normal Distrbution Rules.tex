\documentclass[]{report}

\voffset=-1.5cm
\oddsidemargin=0.0cm
\textwidth = 480pt

\usepackage{framed}
\usepackage{subfiles}
\usepackage{graphics}
\usepackage{newlfont}
\usepackage{eurosym}
\usepackage{amsmath,amsthm,amsfonts}
\usepackage{amsmath}
\usepackage{color}
\usepackage{amssymb}
\usepackage{multicol}
\usepackage[dvipsnames]{xcolor}
\usepackage{graphicx}
\begin{document}



\section{Important rules for Normal distribution}

%=====================================================================================%
\section{Complement Rule}
For some value $A$, and for any continuous distribution $X$ (including any normal distribution and the Z distribution) we can say.
\[ P(X \leq x) = 1 - P(X \geq x) \]

This rule is very simple intuitive. If 70\% of the population is below a certain height, that means that 30\% of the population is above that height. Having said that, it is very powerful in solving normal probability problems.

 \textbf{Complement rule}
  Common to all continuous random variables
 \[P(Z \geq k) = 1 - P(Z \leq k) \]
 Similarly
 \[P(X \leq k) = 1 - P(X \geq k) \]
 
 
 \[P(Z \leq 1.28) = 1 - P(Z \geq 1.28)  = 1-0.1003 = 0.8997\]

Any normal distribution problem can be solved with some combination of the following rules.



This rule applies to all continuous probability distributions, i.e. the normal distribution, the exponential distribution and the uniform distribution.
(Common to all continuous random variables) 


Example
\[P(Z \leq 1.96)   = 1 - P(Z \geq 1.96)   = 1 - 0.0250 = 0.9750\]


\begin{itemize}
\item $P(Z \leq 1.27) = 1-P(Z \geq 1.27) $
\item $ P(Z \geq 1.27) = 1-P(Z \leq 1.27) $
\end{itemize}
\begin{itemize} \item The Complement rule (Common to all continuous random variables)

\[P(Z \geq k) = 1 - P(Z \leq k) \]


Similarly

\[P(X \geq k) = 1 - P(X \leq k) \]

\end{itemize}
%=====================================================================================%
\section{Symmetry Rule}
For the standard normal ($Z$) distribution only, where \[Z \sim \mathcal{N}(0,1^2)\] and $z$ and $-z$ are both values drawn from that distribution, we can say
\[ P(Z \leq -z) = P(Z \geq z) \]

or conversely

\[ P(Z \geq -z) = P(Z \leq z) \]

Important - the symmetry rule can be applied to z-values.



\begin{itemize}
\item
This rule is based on the property of symmetry mentioned previously.
\item
Only the probabilities corresponding to values between 0 and 4 are tabulated in Murdoch Barnes.
\item
If we have a negative value of k, we can use the symmetry rule.
\end{itemize}
\[P(Z \leq -k) = P(Z \geq k) \]
by extension, we can say
\[P(Z \geq -k) = P(Z \leq k) \]

\begin{itemize} 
\item \textbf{Complement rule} is common to all continuous random variables
\[P(X \geq k) = 1 - P(X \leq k) \]
\end{itemize}



\begin{itemize}
\item $ P(Z \leq -1.27) = P(Z \geq 1.27) $
\item $ P(Z \geq -1.27) = P(Z \leq 1.27) $
\end{itemize}

  
\section*{Example}

Find  $P(Z \leq -1.8)$

\noindent \textbf{Solution:} 

$P(Z \leq -1.80) = P(Z \geq 1.80) $ = 0.0359

%============================================================%


%=====================================================================================%
\section{Complement and Symmetry Rules}

Any normal distribution problem can be solved (partially at least) with some combination of the following rules.

\subsection{Z Scores: Example 1 }
Find $P(Z \geq -1.28)$\\
\textbf{Solution}\\
\begin{itemize}
\item Remark \[P(Z \leq 1.28) = 1 - P(Z \geq 1.28)  = 1-0.1003 = 0.8997\]
\item Using the symmetry rule
\[P(Z \geq -1.28) = P(Z \leq 1.28) \]
\item Using the complement rule
\[P(Z \geq -1.28) = 1 - P(Z \geq 1.28) \]
\[P(Z \geq -1.28) = 1 - 0.1003 = 0.8997 \]
\end{itemize}





\section{Example}
Given that the mean $\mu  = 100$ and that the standard deviation $\sigma  = 2.5$, what is the "z-value" for normal random variable x = 106?

\[ z  = \frac{x-\mu}{\sigma}   = \frac{106-100}{2.5}  = \frac{6}{2.5} = 2.40\]

%-----------------------------------------------%

Relationship between "x value" and "z  value"

[VERY IMPORTANT]

\[If   z  = \frac{x-\mu}{\sigma}\]( $z$  and $x$  are some values )

then we can say  \[P(X \geq x) = \Pr(Z \geq z)\]

or equivalently


\[P(X \leq x) = \Pr(Z \leq z)\]

From previous example 

\[P(X \geq 106) = \Pr(Z \geq 2.40)\]


From Murdoch Barnes Table 3, 

\[\P(Z \geq 2.40) = 0.00820\] 


Therefore 


\[P(X \geq 106) = 0.00820\] 




%===============================================%

Find the probability of a ``z" random variable being between -1.8 and 1.96?
i.e. Compute $P(-1.8 \leq Z \leq 1.96)$
Solution
\begin{itemize}
\item Consider the complement event of being in this interval: a combination of being too low or two high.
\item
The probability of being too low for this interval is $P(Z \leq -1.80) = 0.0359$ (from before)
\item
The probability of being too high for this interval is $P(Z \geq 1.96) = 0.0250$ (from before)
\item
Therefore the probability of being \textbf{outside} the interval is 0.0359 + 0.0250 = 0.0609.
\item
Therefore the probability of being \textbf{inside} the interval is 1- 0.0609 = 0.9391
$P(-1.8 \leq Z \leq 1.96) = 0.9391$
\end{itemize}


\subsection{Example }
Find the probability of a "z" random variable greater than (or equal to) -1.8?
Find  $P(Z \geq -1.8)$
\noindent \t{Solution }

(From a previous question, $P(Z \leq -1.8) = 0.0359$)

\[P(Z \geq -1.8) = 1 - P(Z \leq -1.8) \]
\[P(Z \geq -1.8)= 1 - 0.0359 = 0.9641\]


\section{Complement and Symmetry Rules}

For a normally distributed random variable with mean $\mu = 1000$ and standard deviation $\sigma = 100$, compute $P(X \geq 873)$.

\begin{itemize} \item First, find the Z-value using the standardization formula.
\[
z_{873} = {x_o - \mu \over \sigma} = {873 - 1000 \over 100} = {-127 \over 100} = -1.27
\]
\item We can say $P(X \geq 873) = P(Z \geq -1.27)$.
\item Use complement rule and symmetry rule to evaluate  $P(Z \geq -1.27)$.
\item $ P(Z \geq -1.27) = P(Z \leq 1.27) = 1 - P(Z \geq 1.27) $  = 1 - 0.1020 = {0.8980}.
\end{itemize}


%------------------------------------------------------------------%


\section{Summary}

\begin{itemize}

\item \textbf{Complement Rule}\\
For some value $A$, and for any continuous distribution $X$ (including any normal distribution and the Z distribution) we can say.
\[ P(X \leq a) = 1 - P(X \geq A) \]

\item \textbf{Symmetry Rule}\\
For the standard normal ($Z$) distribution only, we can say
\[ P(Z \leq -A) = P(Z \geq A) \]

or conversely

\[ P(Z \geq -A) = P(Z \leq A) \]

\end{itemize}




%===============================================%
\section{Normal Distributions}
\begin{equation}
Z_{o}= \frac{ X_{o}-\mu }{\sigma}
\end{equation}

\begin{equation}
P(Z \geq Z_{o}) = P(X \geq X_{o})
\end{equation}

%------------------------------------------------------------------------%
{
\subsection{Using Murdoch Barnes tables 3}
\begin{itemize}
\item $ P(Z \geq 1.64) = 0.505$
\item $ P(Z \geq 1.65) = 0.495$ \bigskip
\item $ P(Z \geq 1.645)$ is approximately the average value of $ P(Z \geq 1.64)$ and $ P(Z \geq 1.65)$.
\item $ P(Z \geq 1.645)$ = (0.0495 + 0.0505)/2 = 0.0500. ( i.e. $5\%$ )
\end{itemize}
}






%============================================================% 

\subsection{Solving using the Z distribution}
When we have a normal distribution with any mean $\mu$ and any standard deviation $\sigma$ , we answer probability questions about the distribution by first converting all values to corresponding values of the standard normal ("z") distribution.
The formula used to convert any random variable "X" ( with mean $\mu$ and standard deviation $\sigma$ specified) to the standard normal ("z") distribution is given as follows.
\[ Z_o = {X_o - \mu \over \sigma} \]
$Z$ is the standard normal random variable with a mean of zero and a standard deviation of 1.
It can be thought of as a measure of how many standard deviations that a value "x" is from mean $\mu$ .


\noindent \textbf{Remarks}

\begin{itemize}
\item A value of x equal to mean $\mu$  results in a z -value of 0

\[ z = \frac{\mu - \mu}{\sigma} = \frac{0}{\sigma} = 0\]


\item Thus we can see that a value of "x" corresponding to its mean $\mu$ corresponds to a z-value at its mean , which is 0.

\item A value of "x" that is one standard deviation above its mean (i.e. $x=\mu +\sigma$  ), we see that the corresponding z value is 1.


\item Thus a value of x that is one standard deviation away from it's mean yields a z-value of 1.
\end{itemize}







%=======================================================================%

Solution 



Example 
Find the probability of a "z" random variable greater than (or equal to) -1.8?
Find  

Solution





\section{Normal Distribution : Solving problems}
Recap:
\begin{itemize}
\item We must know the normal mean $\mu$ and the normal standard deviation $\sigma$.
\item The normal random variable is $X \sim \mbox{N} ( \mu , \sigma^2)$.\smallskip
\item (If we don't, we usually have to determine them, given the information in the question.)\smallskip
\item The standard normal random variable is $Z\sim \mbox{N} ( 0 , 1^2)$.\smallskip
\item The standard normal distribution is well described in Murdoch Barnes Table 3, which tabulates $P(Z \geq z_o)$ for a range of $Z$ values.
\end{itemize}

%-----------------------------------------------------%


\begin{itemize}
\item For the given value $x_o$ from the variable $X$, we compute the corresponding z-score $z_o$.
\[ z_o = { x_o - \mu \over \sigma} \]
\item When $z_o$ corresponds to $x_o$, the following identity applies:
\[  P(X \geq x_o )= P(Z \geq z_o ) \]
\item Alternatively $ P(X \leq x_o )= P(Z \leq z_o ) $
\end{itemize}




\end{document}



