\documentclass[a4paper,12pt]{article}
%%%%%%%%%%%%%%%%%%%%%%%%%%%%%%%%%%%%%%%%%%%%%%%%%%%%%%%%%%%%%%%%%%%%%%%%%%%%%%%%%%%%%%%%%%%%%%%%%%%%%%%%%%%%%%%%%%%%%%%%%%%%%%%%%%%%%%%%%%%%%%%%%%%%%%%%%%%%%%%%%%%%%%%%%%%%%%%%%%%%%%%%%%%%%%%%%%%%%%%%%%%%%%%%%%%%%%%%%%%%%%%%%%%%%%%%%%%%%%%%%%%%%%%%%%%%
\usepackage{eurosym}
\usepackage{vmargin}
\usepackage{amsmath}
\usepackage{graphics}
\usepackage{epsfig}
\usepackage{subfigure}
\usepackage{enumerate}
\usepackage{fancyhdr}

\setcounter{MaxMatrixCols}{10}
%TCIDATA{OutputFilter=LATEX.DLL}
%TCIDATA{Version=5.00.0.2570}
%TCIDATA{<META NAME="SaveForMode"CONTENT="1">}
%TCIDATA{LastRevised=Wednesday, February 23, 201113:24:34}
%TCIDATA{<META NAME="GraphicsSave" CONTENT="32">}
%TCIDATA{Language=American English}

\pagestyle{fancy}
\setmarginsrb{20mm}{0mm}{20mm}{25mm}{12mm}{11mm}{0mm}{11mm}
\lhead{MS4222} \rhead{Kevin O'Brien} \chead{The Poisson Distribution} %\input{tcilatex}

\begin{document}


\section*{The Poisson Distribution: Introduction}
	\begin{itemize}
		\item The Poisson Distribution is a statistical distribution showing the frequency probability of specific events when the average probability of a single occurrence is known. 
		
		\item The Poisson distribution is a discrete probability distribution.
		
%	\item Application of the Poisson distribution enables managers to introduce optimal scheduling systems. 
		
%	\item For example, if the average number of people that rent movies on a friday night at a single video store location is 400,  a Poisson distribution can answer such questions as, "\textit{What is the probability that more than 600 people will rent movies?}".
		
		
		
		\item One of the most famous historical practical uses of the Poisson distribution was estimating the annual number of Prussian cavalry soldiers killed due to horse-kicks. 
		
		\item Other modern examples include estimating the number of car crashes in a city of a given size. In physiology, this distribution is often used to calculate the probabilistic frequencies of different types of neurotransmitter secretions. 
	\end{itemize}

\subsection*{Poisson Random Variables}

\begin{itemize}
	\item A Poisson random variable is the number of successes that result from a Poisson experiment.
	\item The probability distribution of a Poisson random variable is called a Poisson distribution.
	\item \textbf{Very Important:} This distribution describes the number of occurrences in a \textbf{\emph{unit period (or space)}}
	\item \textbf{Very Important:} The expected number of occurrences is $m$ per unit period (or unit space).
	
% \item The number of occurrences in a 
% \item \texttt{R} refers to the mean number of occurrences as \texttt{lambda} rather than \texttt{m}. 
	

\end{itemize}

\smallskip

\subsection*{Characteristics of a Poisson Experiment}
A Poisson experiment is a statistical experiment that has the following properties:
\begin{itemize}
	\item The experiment results in outcomes that can be classified as successes or failures.
	\item The average number of successes (m) that occurs in a specified unit space is known.
	\item Note that the specified unit space could take many forms. For instance, it could be a length, an area, a volume, a period of time, etc.
	\item  The probability that a success will occur is proportional to the size of the \textbf{\emph{unit space}}.
	\item The probability that a success will occur in an extremely small unit space is virtually zero.
%		\item[$\ast$]  The \texttt{pois} family of functions are used to compute probabilities and quantiles.
	\item  The \textbf{Poisson mean} $m$ (or $\lambda$ pronounced as ``lambda") is the expected number of occurrences per unit space / unit period.
\begin{itemize}
\item[$\ast$] (Remark:  Some texts will use the notation $\lambda$ rather than  $m$ ).
\end{itemize}
\end{itemize}

%
%%---------------------------------------------------------------------------%
%\section{Poisson Distribution}
%
%\begin{itemize}
%\item A discrete random variable that is often used is one which estimates the number of occurrences  over a specified time period or space.
%\item (remark : a specified space can be a specified length , a specified area, or a specified volume.)
%
%\item

%\subsection{Poisson Probability Distribution}
%
%
%
%\begin{itemize}
%	\item The number of occurrences in a unit period (or space)
%	\item The expected number of occurrences is $m$
%	\item Given the mean number of successes ($m$) that occur in a specified region, we can compute the Poisson probability based on the following formula (next slide).
%\end{itemize}


\subsection*{Important Assumptions}
 If the following two properties are satisfied, the number of occurrences is a random variable described by the Poisson probability distribution
%
\begin{itemize}
\item[(1)]      The probability of an occurrence is the same for any two intervals of equal length.
\item[(2)]     The occurrence or non-occurrence in any interval is independent of the occurrence or non-occurrence in any other interval.
\end{itemize}


%---------------------------------------------------------------------------%








%---------------------------------------------------------------------%


\subsection*{When To Use The Poisson Probability Distribution}
Consider cars passing a point on a rarely used country road. Is this a Poisson Random Variable?
Suppose that 
\begin{enumerate}
\item Arrivals occur at an average rate of $m$ cars per unit time.
\item The probability of an arrival in an interval of length $k$
is constant.
\item The number of arrivals in two non-overlapping
intervals of time are independent.
\end{enumerate}
This would be an appropriate use of the Poisson Distribution.

%---------------------------------------------------------------------%


{



%
%Given the mean number of successes ($m$) that occur in a specified region, we can compute the Poisson probability based on the following formula:
%
%
%
%
%
%
%%\frametitle{Poisson Random Variables}
%\noindent\textbf{Question:}\\
%Suppose that random variable X follows a Poisson distribution with rate parameter \texttt{\textbf{L}}. \\
%If we increase the value of \texttt{\textbf{L}}, which of the following is true?







\subsection*{Knowing which distribution to use}
\begin{itemize}
\item For the end of semester examination, you will be required to know when it is appropriate to use the Poisson distribution, and when to use the binomial distribution.
\item Recall the key parameters of each distribution.
\item Binomial : number of \textbf{\emph{successes}} in $n$ \textbf{\emph{independent trials}}.
\item Poisson : number of \textbf{\emph{occurrences}} in a \textbf{\emph{unit space}}.
\end{itemize}













\end{document} 
