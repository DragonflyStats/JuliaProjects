\documentclass[a4]{beamer}
\usepackage{amssymb}
\usepackage{graphicx}
\usepackage{subfigure}
\usepackage{newlfont}
\usepackage{amsmath,amsthm,amsfonts}
%\usepackage{beamerthemesplit}
\usepackage{pgf,pgfarrows,pgfnodes,pgfautomata,pgfheaps,pgfshade}
\usepackage{mathptmx}  % Font Family
\usepackage{helvet}   % Font Family
\usepackage{color}

\mode<presentation> {
 \usetheme{Default} % was Frankfurt
 \useinnertheme{rounded}
 \useoutertheme{infolines}
 \usefonttheme{serif}
 %\usecolortheme{wolverine}
% \usecolortheme{rose}
\usefonttheme{structurebold}
}

\setbeamercovered{dynamic}

\title[MathsCast]{MathsCast Presentations \\ {\normalsize Continuous Probability Distributions}}
\author[Kevin O'Brien]{Kevin O'Brien \\ {\scriptsize kevin.obrien@ul.ie}}
\date{Summer 2011}
\institute[Maths \& Stats]{Dept. of Mathematics \& Statistics, \\ University \textit{of} Limerick}

\renewcommand{\arraystretch}{1.5}


%------------------------------------------------------------------------%
\begin{document}
\frame{
\frametitle{Midterm Exam}
\begin{itemize}
\item 1300hr on Tuesday (Week 7)
\item Worth 15\% of overall mark for module.
\item The test is comprised of 15 short questions, or components of compound questions.
\item Topics: 
\begin{itemize}
\item Basic Probability
\item The Binomial Distribution
\item The Poisson Distribution
\item The Normal Distribution
\end{itemize}
\item Revision lecture on Monday of Week 7.
\end{itemize}
}
%------------------------------------------------------------------------%
\begin{frame}
\large
\frametitle{Today's Class}
\begin{itemize}
\item Introduction to Continuous Random Variables
\item Probability Density Functions
\item Density Curves
\item Cumulative Distribution Functions
\item The Continuous Uniform Distribution
\item The Exponential Distribution
\item Introduction to the Normal Distribution
\item (Tomorrow: More on the Normal Distribution)
\end{itemize}
\end{frame}
%----------------------------------------------------------------------------------------------------%
\frame{
\frametitle{Continuous Random Variable} 
A continuous random variable is one which takes an infinite number of possible values. Continuous random variables are usually measurements. Examples include height, weight, the amount of sugar in an orange, the time required to run a mile.
\\
\bigskip
In this module, we will look at random variables from the following distributions
\begin{itemize}
\item Continuous uniform probability distribution
\item Exponential probabilty distribution
\item Normal probability distribution
\end{itemize}

Before this, we will look at some important concepts.
}

%----------------------------------------------------------------------------------------------------%
\begin{frame}
\frametitle{Functions and Definite integrals}
\large
Integration is not part of the syllabus, and it is assumed that students have are not familiar with how to compute definite integrals.\\ \bigskip
 However,  it is useful to know what the purpose of definite integrals are, because we will be using the results derived from definite integrals. \\ It is assumed that students are familiar with functions.

\end{frame}
%----------------------------------------------------------------------------------------------------%
\begin{frame}
\frametitle{Functions}

IMAGE : 5A Functions

\end{frame}
%----------------------------------------------------------------------------------------------------%
\begin{frame}
\frametitle{Definite Integral}

IMAGES : 5A Definite Integrals

\end{frame}
%----------------------------------------------------------------------------------------------------%
\begin{frame}
\frametitle{Definite Integral}

Definite integrals are used to compute the "area under curves". 

The area under the curve between X=1 and  X=2 is depicted in grey. Using definite integrals

\end{frame}

%----------------------------------------------------------------------------------------------------%
\frame{
\large
\frametitle{Probability Density Function}
The probability density function of a continuous random variable is a function which can be integrated to obtain the probability that the random variable takes a value in a given interval.
\\
\bigskip
In probability theory, a probability density function (pdf), or density of a continuous random variable is a function that describes the relative likelihood for this random variable to occur at a given point. 
\\
\bigskip
The probability for the random variable to fall within a particular region is given by the integral of this variable’s density over the region. The probability density function is non-negative everywhere, and its integral over the entire space is equal to one.
}

%----------------------------------------------------------------------------------------------------%
\frame{
\frametitle{Probability Mass Function}
\large
\begin{itemize} \item a probability mass function (pmf) is a function that gives the probability that a discrete random variable is exactly equal to some value. \item The probability mass function is often the primary means of defining a discrete probability distribution \end{itemize}
}

%----------------------------------------------------------------------------------------------------%

\begin{frame}
\large
\frametitle{Density Curves}
Any curve that is always on or above the horizontal axis and has
total are underneath equal to one is a density curve.
\begin{itemize}
\item Area under the curve in a range of values indicates the proportion of values in that range.
\item Come in a variety of shapes, but the ``normal” family of familiar
bell-shaped densities is commonly used.
\item Remember the density is only an approximation, but it simplifies analysis and is generally accurate enough for practical
use.
\end{itemize}
\end{frame}
%----------------------------------------------------------------------------------------------------%
\frame{
\frametitle{The Cumulative Distribution Function }
The cumulative distribution function (CDF), or just distribution function, describes the probability that a real-valued random variable X with a given probability distribution will be found at a value less than or equal to x.\\ Intuitively, it is the ``area so far" function of the probability distribution. 

\[ F_X(x) = P(X \leq x) \]
}

%------------------------------------------------------------------%
\frame{
\frametitle{Exact Probability}
\large
\alert{Remarks:} This is for continuous distributions only.
\begin{itemize}
\item The probability that a continuous random variable will take an exact value is infinitely small.
We will usually treat it as if it was zero.
\item
When we write probabilities for continuous random variables in mathematical notation, we often retain the equality component (i.e. the "...or equal to..").\\
For example, we would write expressions $P(X \leq 2)$ or $P(X \geq 5)$.
\item
Because the probability of an exact value is almost zero, these two expression are equivalent to $P(X < 2)$
or $P(X > 5)$. \item The complement of $P(X \geq k)$ can be written as $P(X \leq k)$.
\end{itemize}
}



%----------------------------------------------------------------------------------------------------%
\begin{frame}
\frametitle{The Continuous Uniform Distributions}

IMAGE : 5A uniform

\end{frame}



%------------------------------------------------------------------------%
\section{The Continuous Uniform Distribution}
%------------------------------------------------------------------------%
\frame{
\frametitle{Parameters}
\Large

The continuous uniform distribution is characterised by the following parameters

\begin{itemize}
\item The lower limit $a$
\item The upper limit $b$
\end{itemize}

It is not possible to have an outcome that is lower than $a$ or larger than $b$.

\[ P(X < a) = P(X > b) = 0\]
}

%------------------------------------------------------------------------%
\frame{
\Large
\begin{itemize}
\item The only possible outcomes are between $a$ and $b$. Suppose $a = 3$ and $b = 6$.\bigskip
\item The following values are possible outcomes: $3.14, \;3.78,\; 4.66,\; 5.8,\;5.9999.$ \bigskip
\item The probability of being exactly equal to $3$ or $6$ can be assumed to be zero. \bigskip
\item The following outcomes are not possible, either because they are too high or too low.
$1.67,\;2,\;67,\;7.14,\; 8.78.$
\end{itemize}
}


%------------------------------------------------------------------------%
\frame{
\frametitle{Example}
\Large
\begin{itemize}
\item Suppose there is a platform in a subway station in a large large city. \item Subway trains arrive \textbf{every three minutes} at this platform. \item What is the shortest possible time a passenger would have to wait for a train? 
\item What is the longest possible time a passenger will have to wait?
\end{itemize}

}


%------------------------------------------------------------------------%
\frame{
\frametitle{Example}
\Large
\begin{itemize}
 \item What is the shortest possible time a passenger would have to wait for a train? 
%\begin{itemize}
\item If the passenger arrives just before the doors close, then the waiting time is zero.
\[ a = 0 \mbox{ minutes } \]
\end{itemize}
}


%------------------------------------------------------------------------%
\frame{
\frametitle{Example}
\Large
\begin{itemize}
\item What is the longest possible time a passenger will have to wait?
%\begin{itemize}
\item If the passenger arrives just after the doors close, and missing the train, then he or she will have to wait the full three minutes for the next one.
\[ b = 3 \mbox{ minutes }  = 180 \mbox{ seconds}  \]
\end{itemize}
%\end{itemize}

}

%------------------------------------------------------------------------%
\frame{
\frametitle{The Expected Value}
\Large
We are told that, for waiting times,  the lower limit $a$ is 0, and the upper limit $b$ is 3 minutes. \\ \bigskip The expected waiting time $\textrm{E}[X]$ is computed as follows
\vspace{0.1cm}
\[
\textrm{E}[X] = {b + a \over 2} =  {3 + 0  \over 2}  = 1.5 \mbox{ minutes } 
\]

}

%------------------------------------------------------------------------%
\frame{\frametitle{Interval Probability}
\Large
\begin{itemize}
\item We wish to compute the probability of an outcome being within a range of values.
\item We shall call this lower bound of this range $L$ and the upper bound $ U$.
\item Necessarily $L$ and $U$ must be possible outcomes.
\item The probability of $X$ being between $L$ and $U$ is denoted $P( L \leq X \leq U )$.
\end{itemize}
\[
P( L \leq X \leq U ) = { U - L \over b - a}
\]
}
%---------------------------------------------------------------------------------------------------------%
\frame{
The probability density function is given as
\[f(x) = {1 \over b-a} \mbox{ for } a \leq x \leq b\]
For any value ``c" between the minimum value a and the maximum
value b
\[P(X \geq c) = {b-c \over b-a}\]
here $b$ is the upper bound while $c$ is the lower bound
\[P(X \leq c) = {c-a \over b-a}\]
here c is the upper bound while a is the lower bound
}

%---------------------------------------------------------------------------%
\frame{
\frametitle{Continuous Random Variables}
\begin{itemize}
\item Probability Density Function
\item Cumulative Density Function
\end{itemize}
If X is a continuous random variable then we can say that the probability of obtaining a \textbf{precise} value $x$ is infinitely small, i.e. close to zero.
\[P(X=x) \approx 0 \]
Consequently, for continuous random variables (only), $P(X \leq x)$ and $P(X < x)$ can be used interchangeably.
\[P(X \leq x) \approx P(X < x) \]
}
%---------------------------------------------------------------------------------------------------------%
\section{Continuous Uniform distribution}
\frame{
\begin{itemize}
\item $L$ :lower bound of an interval \item $U$: upper bound of an
interval
\end{itemize}
Probability of an outcome being between lower bound L and upper
bound U \[P( L \leq X \leq U) = { U - L \over b - a }\]
\textbf{Reminder}
"$\leq$" is less than or equal to.\\
"$\geq$" is greater than or equal to.\\
$L \leq X \leq U$ xan be verbalized as X between L and U. simply
states that X is between L and U inclusively.
("inclusively" mean that X could be exactly L or U also, although
the probability of this is extremely low)\\
}


%------------------------------------------------%
\frame{
\frametitle{Continuous Uniform Distribution} 
\begin{itemize}
\item 
The Uniform distributions model (some) continuous random variables and (some) discrete random variables. 
\item
The values of a uniform random variable are uniformly distributed over an interval. 
\item
For example, if buses arrive at a given bus stop every 15 minutes, and you arrive at the bus stop at a random time, the time you wait for the 
next bus to arrive could be described by a uniform distribution over the interval from 0 to 15.
\end{itemize}
 
}
%---------------------------------------------------------------------------%
\frame{
\frametitle{Continuous Uniform Distribution}
A random variable X is called a continuous uniform random variable over the interval $(a,b)$ if it's probability density function is given by
\[ f_{X}(x) = { 1 \over b-a} \hspace{2cm} \mbox{ when } a \leq x \leq b\]
The corresponding cumulative density function is
\[ F_x(x) = { x-a \over b-a} \hspace{2cm} \mbox{ when } a \leq x \leq b\]
}
%-----------------------------------------------------------------------------%
\frame{
\frametitle{Continuous Uniform Distribution}
The mean of the continuous uniform distribution is
\[ E(X) = {a+b \over 2}\]
\[ V(X) = {(b-a)^2\over12}\]
}
%------------------------------------------------------------------------%
\frame{\frametitle{Uniform Distribution: Variance}
\Large
The variance  of the continuous uniform distribution, denoted $\textrm{Var}[X]$,  is  computed using the following formula
\vspace{0.1cm}
\[
\textrm{Var}[X] = {(b - a)^2 \over 12}
\]
\vspace{0.1cm}
For our previous example this is
\[
\textrm{Var}[X] = \alert{{(3 - 0)^2 \over 12} =  {3^2 \over 12} = {9 \over 12} = 0.75}
\]
}



%----------------------------------------------------------------------------%
\frame{
\frametitle{The Exponential Distribution}
A continuous random variable having p.d.f. f(x), where:
$f(x) = \lambda x e ^{-\lambda x} $
is said to have an exponential distribution, with parameter $\lambda$. 
The cumulative distribution is given by:
$F(x) = 1 – e^{\lambda x}$

Expectation and Variance
$E(X) = 1 / \lambda$
$V(X) = 1 / \lambda^2$
}

%----------------------------------------------------------------------------%
\frame{
\frametitle{Example}
Suppose that the service time for a customer at a fast-food outlet
has an exponential distribution with mean 3 minutes. What is the probability that a
customer waits more than 4 minutes?

\[ P(X  \leq 4) = 1 -  e^{-4/3} \]

\[ P(X  \leq 4) = e^{-4/3} = 0.2636 \]
}


%---------------------------------------------------------------------------------%
\begin{frame}
\frametitle{Exponential Distribution Lifetimes}
The average lifespan of a laptop is 5 years. You may assume that
the lifespan of computers follows an exponential probability
distribution. \begin{itemize}\item (3 marks) What is the
probability that the lifespan of the laptop will be at least 6
years? \item (3 marks)
What is the probability that the lifespan of the laptop will not
exceed 4 years? \item(3 marks) What is the probability of the
lifespan being between 5 years and 6 years?
\end{itemize}
Suppose the lifetime of a PC is exponentially distributed with
mean $\mu =5$
We should be told the average lifetime $\mu$.
\[
P( X \geq x_o) = e^{{-x_o \over \mu}}
\]
\end{frame}
%-----------------------------------------------------------------------------%
\frame{
\frametitle{The Memoryless property}
The most interesting property of the exponential distribution is the \textbf{\emph{memoryless}} property. By this , we mean that if the lifetime of a component is exponentially distributed, then an item which has been in use for some time is a good as a brand new item with regards to the likelihood of failure.
The exponential distribution is the only distribution that has this property.
}


\end{document}








