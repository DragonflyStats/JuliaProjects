  \documentclass[a4]{beamer}
\usepackage{amssymb}
\usepackage{graphicx}
\usepackage{subfigure}
\usepackage{newlfont}
\usepackage{amsmath,amsthm,amsfonts}
%\usepackage{beamerthemesplit}
\usepackage{pgf,pgfarrows,pgfnodes,pgfautomata,pgfheaps,pgfshade}
\usepackage{mathptmx} % Font Family
\usepackage{helvet} % Font Family
\usepackage{color}
\mode<presentation> {
\usetheme{Frankfurt} % was Frankfurt
\useinnertheme{rounded}
\useoutertheme{infolines}
\usefonttheme{serif}
%\usecolortheme{wolverine}
% \usecolortheme{rose}
\usefonttheme{structurebold}
}
\setbeamercovered{dynamic}
\title[MA4413]{Statistics for Computing \\ {\normalsize MA4413 Lecture 5A}}
\author[Kevin O'Brien]{Kevin O'Brien \\ {\scriptsize kevin.obrien@ul.ie}}
\date{Autumn 2011}
\institute[Maths \& Stats]{Dept. of Mathematics \& Statistics, \\ University \textit{of} Limerick}
\renewcommand{\arraystretch}{1.5}
%------------------------------------------------------------------------%
\begin{document}


%%%%%%%%%%%%%%%%%%%%%%%%%%%%%%%%%%%%%%%%%%%%%%%%%%%%%%%%%%%%%%%%5

\begin{frame}[fragile]
\frametitle{Exponential Distribution: Example}
Assume that the length of a phone call in minutes is an exponential random variable $X$ with parameter
$\lambda = 1/10$. If someone arrives at a phone booth just before you arrive, find the probability that you
will have to wait \begin{itemize}
\item[(a)] less than 5 minutes,  
\item[(b)] between 5 and 10 minutes.
\end{itemize}
Use the \texttt{R} code on the following slide to help answer these questions.
\end{frame}



%------------------------------------------------------------------------%
\begin{frame}[fragile]
\frametitle{Exponential Distribution: Example}
\begin{verbatim}
> dexp(0:10,rate=0.10)
 [1] 0.10000000 0.09048374 0.08187308 0.07408182 0.06703200 0.06065307
 [7] 0.05488116 0.04965853 0.04493290 0.04065697 0.03678794
>
> pexp(0:10,rate=0.10)
 [1] 0.00000000 0.09516258 0.18126925 0.25918178 0.32967995 0.39346934
 [7] 0.45118836 0.50341470 0.55067104 0.59343034 0.63212056
\end{verbatim}
\end{frame}

%------------------------------------------------------------------------%
\begin{frame}[fragile]
\frametitle{Exponential Distribution: Example}

As it is CDF values that we are interested in, we use the output from the \texttt{pexp()} commands.

\begin{itemize}
\item[(a)] $P(X \leq 5)$ = 0.39346934 
\item[(b)] $P(5 \leq X \leq 10)$ \\ = $P( X \leq 10) - P( X \leq 5)$ \\ = 0.63212056- 0.39346934 \\ = 0.2386512 \\= 23.84 $\%$
\end{itemize}

\end{frame}




%------------------------------------------------------------%
\begin{frame}
\frametitle{Continuous Random variables}
\begin{itemize}
\item Previously we have been studying discrete random variables, such as the Binomial and the Poisson random variables.
\item Now we turn our attention to continuous random variables.
\item Recall that a continuous random variable is one which takes an infinite number of possible values, rather than just a countable number of distinct values.
\item Continuous random variables are usually measurements.
\item Examples include height, weight, the amount of sugar in an orange, the time required to run a mile.
\end{itemize}

\end{frame}

\end{document}                             
