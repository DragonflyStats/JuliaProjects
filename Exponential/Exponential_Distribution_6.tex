\documentclass[a4]{beamer}
\usepackage{amssymb}
\usepackage{graphicx}
\usepackage{subfigure}
\usepackage{newlfont}
\usepackage{amsmath,amsthm,amsfonts}
%\usepackage{beamerthemesplit}
\usepackage{pgf,pgfarrows,pgfnodes,pgfautomata,pgfheaps,pgfshade}
\usepackage{mathptmx}  % Font Family
\usepackage{helvet}   % Font Family
\usepackage{color}

\mode<presentation> {
 \usetheme{Frankfurt} % was Frankfurt
 \useinnertheme{rounded}
 \useoutertheme{infolines}
 \usefonttheme{serif}
 %\usecolortheme{wolverine}
% \usecolortheme{rose}
\usefonttheme{structurebold}
}

\setbeamercovered{dynamic}

\title[MathsCast]{MathsCast Presentations \\ {\normalsize Continuous Probability Distributions}}
\author[Kevin O'Brien]{Kevin O'Brien \\ {\scriptsize kevin.obrien@ul.ie}}
\date{Summer 2011}
\institute[Maths \& Stats]{Dept. of Mathematics \& Statistics, \\ University \textit{of} Limerick}

\renewcommand{\arraystretch}{1.5}


%------------------------------------------------------------------------%
\begin{document}



%----------------------------------------------------------------------------%
\frame{
\frametitle{The Exponential Distribution}
A continuous random variable having p.d.f. f(x), where:
$f(x) = \lambda x e ^{-\lambda x} $
is said to have an exponential distribution, with parameter $\lambda$. 
The cumulative distribution is given by:
$F(x) = 1 – e^{\lambda x}$

Expectation and Variance
$E(X) = 1 / \lambda$
$V(X) = 1 / \lambda^2$
}

%----------------------------------------------------------------------------%
\frame{
\frametitle{Example}
Suppose that the service time for a customer at a fast-food outlet
has an exponential distribution with mean 3 minutes. What is the probability that a
customer waits more than 4 minutes?

\[ P(X  \leq 4) = 1 -  e^{-4/3} \]

\[ P(X  \leq 4) = e^{-4/3} = 0.2636 \]
}


%---------------------------------------------------------------------------------%
\begin{frame}
\frametitle{Exponential Distribution Lifetimes}
The average lifespan of a laptop is 5 years. You may assume that
the lifespan of computers follows an exponential probability
distribution. \begin{itemize}\item (3 marks) What is the
probability that the lifespan of the laptop will be at least 6
years? \item (3 marks)
What is the probability that the lifespan of the laptop will not
exceed 4 years? \item(3 marks) What is the probability of the
lifespan being between 5 years and 6 years?
\end{itemize}
Suppose the lifetime of a PC is exponentially distributed with
mean $\mu =5$
We should be told the average lifetime $\mu$.
\[
P( X \geq x_o) = e^{{-x_o \over \mu}}
\]
\end{frame}
%-----------------------------------------------------------------------------%
\frame{
\frametitle{The Memoryless property}
The most interesting property of the exponential distribution is the \textbf{\emph{memoryless}} property. By this , we mean that if the lifetime of a component is exponentially distributed, then an item which has been in use for some time is a good as a brand new item with regards to the likelihood of failure.
The exponential distribution is the only distribution that has this property.
}
%-----------------------------------------------------------------------------%


\end{document}








