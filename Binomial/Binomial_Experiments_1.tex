\documentclass[a4]{beamer}
\usepackage{amssymb}
\usepackage{graphicx}
\usepackage{subfigure}
\usepackage{newlfont}
\usepackage{amsmath,amsthm,amsfonts}
%\usepackage{beamerthemesplit}
\usepackage{pgf,pgfarrows,pgfnodes,pgfautomata,pgfheaps,pgfshade}
\usepackage{mathptmx}  % Font Family
\usepackage{helvet}   % Font Family
\usepackage{color}
\mode<presentation> {
 \usetheme{Default} % was
 \useinnertheme{rounded}
 \useoutertheme{infolines}
 \usefonttheme{serif}
 %\usecolortheme{wolverine}
% \usecolortheme{rose}
\usefonttheme{structurebold}
}
\setbeamercovered{dynamic}

\title[MA4704]{Technological Mathematics 4 \\ {\normalsize MA4704 Lecture 3B}}
\author[Kevin O'Brien]{Kevin O'Brien \\ {\scriptsize Kevin.obrien@ul.ie}}
\date{Spring Semester 2013}
\institute[Maths \& Stats]{Dept. of Mathematics \& Statistics, \\ University \textit{of} Limerick}

\renewcommand{\arraystretch}{1.5}

\begin{document}

\begin{frame}
\titlepage
\end{frame}

%---------------------------------------------------------------------------%
\frame{
\frametitle{Probability Mass Function}
(Formally defining something mentioned previously)
\begin{itemize} \item a probability mass function (pmf) is a \textbf{\emph{function}}
that gives the probability that a discrete random variable is exactly equal to some
value.
\[P(X=k)\]
\item The probability mass function is often the primary means of defining a discrete
probability distribution
\item It is conventional to present the probability mass function in the form of a table.
\end{itemize}
}
%--------------------------------------------------------------------------------------%
\frame{
\frametitle{ Binomial Example }
(Revision from Last Class)\\
Suppose a die is tossed 5 times. What is the probability of getting exactly 2 fours?

Solution: This is a binomial experiment in which the number of trials is equal to 5, the number of successes is equal to 2, and the probability of success on a single trial is 1/6 or about 0.167. 

Therefore, the binomial probability is:

\[P(X=2) = ^5C_2 \times (1/6)^2 \times (5/6)^3 = 0.161\]
}
%--------------------------------------------------------------------------------------%
\frame{
\frametitle{Probability Tables}
In the \textbf{sulis} workspace there are two important tables used for this part of the course.
This class will feature a demonstration on how to read those tables.
\begin{itemize}
\item The Cumulative Binomial Tables (Murdoch Barnes Tables 1)
\item The Cumulative Poisson Tables (Murdoch Barnes Tables 2)
\end{itemize}
Please get a copy of each as soon as possible.

}

%---------------------------------------------------------------------------%
\frame{
\frametitle{Probability Tables}
\begin{itemize}
\item For some value $r$ the tables record the probability of $P(X \geq r)$.
\item The Student is required to locate the appropriate column based on the parameter values for the distribution in question.
\item A copy of the Murdoch Barnes Tables will be furnished to the student in the End of Year Exam. The Tables are not required for the first mid-term exam.
\item Knowledge of the sample space, partitioning of the sample points, and the complement rule are advised.
\end{itemize}
}
%---------------------------------------------------------------------------%


\frame{
\frametitle{Binomial Distribution : Using Tables}
It is estimated by a particular bank that 25\% of credit card customers pay only the minimum amount due on their monthly credit card bill and do not pay the total amount due. 50 credit card customers are randomly selected.
\begin{enumerate}
\item (3 marks)	What is the probability that 9 or more of the selected customers pay only the minimum amount due?
\item (3 marks) What is the probability that less than 6 of the selected customers pay only the minimum amount due?
\item (3 marks)	What is the probability that more than 5 but less than 10 of the selected customers pay only the minimum amount due?
\end{enumerate}

}

\frame{
\frametitle{Binomial Distribution : Using Tables}
Demonstration on Blackboard re: how to use tables in class.
\begin{enumerate}
\item $P(X \geq 9) = 0.9084$
\item $P(X < 6) = 1- P(X \geq 6) =1 - 0.9930 = 0.0070$
\item $P(6 \leq X \leq 9) = P(X \geq 6) - P(X \geq 10) = 0.9930 - 0.8363 = 0.1567$
\end{enumerate}

}
