	\documentclass[a4paper,12pt]{article}
%%%%%%%%%%%%%%%%%%%%%%%%%%%%%%%%%%%%%%%%%%%%%%%%%%%%%%%%%%%%%%%%%%%%%%%%%%%%%%%%%%%%%%%%%%%%%%%%%%%%%%%%%%%%%%%%%%%%%%%%%%%%%%%%%%%%%%%%%%%%%%%%%%%%%%%%%%%%%%%%%%%%%%%%%%%%%%%%%%%%%%%%%%%%%%%%%%%%%%%%%%%%%%%%%%%%%%%%%%%%%%%%%%%%%%%%%%%%%%%%%%%%%%%%%%%%
\usepackage{eurosym}
\usepackage{vmargin}
\usepackage{amsmath}
\usepackage{framed}
\usepackage{graphics}
\usepackage{epsfig}
\usepackage{subfigure}
\usepackage{enumerate}
\usepackage{fancyhdr}

\setcounter{MaxMatrixCols}{10}
%TCIDATA{OutputFilter=LATEX.DLL}
%TCIDATA{Version=5.00.0.2570}
%TCIDATA{<META NAME="SaveForMode"CONTENT="1">}
%TCIDATA{LastRevised=Wednesday, February 23, 201113:24:34}
%TCIDATA{<META NAME="GraphicsSave" CONTENT="32">}
%TCIDATA{Language=American English}

\pagestyle{fancy}
\setmarginsrb{20mm}{0mm}{20mm}{25mm}{12mm}{11mm}{0mm}{11mm}
\lhead{MS4222} \rhead{Kevin O'Brien} \chead{Binomial Distribution} %\input{tcilatex}
\begin{document}

%================================================%


\begin{itemize}
	\item
	Let $p$ denote the probability of success in a Bernoulli trial, and so $q = 1 - p$ is the probability of failure.
	A binomial experiment consists of a fixed number of Bernoulli trials. \item A binomial experiment with $n$ trials and
	probability $p$ of success will be denoted by
	\[B(n, p)\]
\end{itemize}



%=================================================%

\subsection{The Binomial Distribution}



Binomial Probability Function

n general, if the random variable X follows the binomial distribution with parameters n ? ? and p ? [0,1], we write X ~ B(n, p). The probability of getting exactly k successes in n trials is given by the probability mass function:

\[f(k;n,p)=\Pr(X=k)={\binom {n}{k}}p^{k}(1-p)^{n-k}\]
for k = 0, 1, 2, ..., n, where

\[{\binom {n}{k}}={\frac {n!}{k!(n-k)!}}\]
is the binomial coefficient, hence the name of the distribution. 

[Remark ; Provided in exam formulae. Please see pg 142]



where

= the probability of   successes in   trials

= the number of trials


\subsection{Probability Mass Function}
(Formally defining something mentioned previously)
\begin{itemize} \item a probability mass function (pmf) is a \textbf{\emph{function}}
	that gives the probability that a discrete random variable is exactly equal to some
	value.
	\[P(X=k)\]
	\item The probability mass function is often the primary means of defining a discrete
	probability distribution
	\item It is conventional to present the probability mass function in the form of a table.
\end{itemize}





%============================================================ %
\section{Binomial Distribution}


Number of independent trials

A coin is tossed eight times. 

 the number of trials is therefore 8.

A group of people or a batch of items can also be considered as a series of independent trials.

%============================================================ %


Probability of a success

A "success" is dependent on how the question is framed, or what is being estimated.



%=================================================%

\subsection{The Binomial Distribution}

\begin{itemize}
\item The discrete random variable X is the number of successes in the n trials. $X$ is modelled by the binomial distribution $B(n,p)$. You can write $X \sim Bin(n, p)$.
\item The binomial distribution can be used to determine the probability of obtaining a designated number of
successes in a Bernoulli process. Three values are required: the designated number of successes (X); the number
of trials, or observations (n); and the probability of success in each trial (p). 
\item P(X = k) gives the probability of$k$  successes in n binomial trials.
\end{itemize}

%--------------------------------------------------------------------------------------%
{
	\subsection{Binomial Distribution (1)}
	\begin{itemize}
		\item Identify the event that can considered the `success'.
		\item (Remark : The success is usually the less likely of two complementary events.)
		\item Determine the probability of a success in a single trial $p$.
		\item Determine the number of independent trials $n$.
	\end{itemize}
	
\end{document}
