\documentclass[a4]{beamer}
\usepackage{amssymb}
\usepackage{graphicx}
\usepackage{subfigure}
\usepackage{newlfont}
\usepackage{amsmath,amsthm,amsfonts}
%\usepackage{beamerthemesplit}
\usepackage{pgf,pgfarrows,pgfnodes,pgfautomata,pgfheaps,pgfshade}
\usepackage{mathptmx}  % Font Family
\usepackage{helvet}   % Font Family
\usepackage{color}

\mode<presentation> {
 \usetheme{Default} % was Frankfurt
 \useinnertheme{rounded}
 \useoutertheme{infolines}
 \usefonttheme{serif}
 %\usecolortheme{wolverine}
% \usecolortheme{rose}
\usefonttheme{structurebold}
}

\setbeamercovered{dynamic}

\title[MA4413]{Statistics for Computing \\ {\normalsize MA4413 Lecture 4A}}
\author[Kevin O'Brien]{Kevin O'Brien \\ {\scriptsize Kevin.obrien@ul.ie}}
\date{Autumn Semester 2012}
\institute[Maths \& Stats]{Dept. of Mathematics \& Statistics, \\ University \textit{of} Limerick}

\renewcommand{\arraystretch}{1.5}

\begin{document}

\begin{frame}
\titlepage
\end{frame}

\frame{
\frametitle{The Binomial Probability Distribution}
A Quick Review of the Binomial Distribution
\begin{itemize}
\item The number of independent trials is denoted $n$.
\item The outcome of interest is known as a ``Success".
\item The other outcome is known as a ``failure".  
\item Often the applications of these names is counter-intuitive, i.e. defective components being the ``success".
\item The probability of a `success' is $p$ 
\item The expected number of `successes' from $n$ trials is $E(X) = np$
\item The \texttt{binom} family of commands in \texttt{R} are what we use to compute necessary values.
\end{itemize}
}
%---------------------------------------------------------------------------%
\frame{
\frametitle{Characteristics of a Poisson Experiment}
A Poisson experiment is a statistical experiment that has the following properties:
\begin{itemize}
\item The experiment results in outcomes that can be classified as successes or failures.
\item The average number of successes (m) that occurs in a specified region is known.
\item The probability that a success will occur is proportional to the size of the \textbf{\emph{region}}.
\item The probability that a success will occur in an extremely small region is virtually zero.
\item The \texttt{pois} family of functions are used to compute probabilities and quantiles.
\end{itemize}
Note that the specified region could take many forms. For instance, it could be a length, an area, a volume, a period of time, etc.
}


%---------------------------------------------------------------------------%
\frame{
\frametitle{The Poisson Probability Distribution}
\begin{itemize}
\item A Poisson random variable is the number of successes that result from a Poisson experiment.
\item The probability distribution of a Poisson random variable is called a Poisson distribution.
\item This distribution describes the number of occurrences in a unit period (or space)
\item The expected number of occurrences is $m$.
\item \text{R} refers to the mean number of occurrences as \texttt{lambda} rather than \texttt{m}. 
\end{itemize}
}

%---------------------------------------------------------------------------%
\frame{
\frametitle{Poisson Formulae}
The probability that there will be $k$ occurrences in a unit time period is denoted $P(X=k)$, and is computed as below. Remark: This is known as the probability density function. The corresponding \texttt{R} command is \texttt{dpois()}.
\Large
\[ P(X = k)=\frac{m^k e^{-m}}{k!} \]


}
%---------------------------------------------------------------------------%
\frame{
\frametitle{Poisson Formulae}
Given that there is on average 2 occurrences per hour, what is the probability of no occurrences in the next hour? \\ i.e. Compute $P(X=0)$ given that $m=2$
\Large
\[ P(X = 0)=\frac{2^0 e^{-2}}{0!} \]
\normalsize
\begin{itemize}
\item $2^0$ = 1
\item $0!$ = 1
\end{itemize}
The equation reduces to
\[ P(X = 0)=e^{-2} = 0.1353\]
}
%---------------------------------------------------------------------------%
\frame{
\frametitle{Poisson Formulae}
What is the probability of one occurrences in the next hour? \\ i.e. Compute $P(X=1)$ given that $m=2$
\Large
\[ P(X = 1)=\frac{2^1 e^{-2}}{1!} \]
\normalsize
\begin{itemize}
\item $2^1$ = 2
\item $1!$ = 1
\end{itemize}
The equation reduces to
\[ P(X = 1) = 2 \times e^{-2} = 0.2706\]
}
%---------------------------------------------------------------%
\frame{
\frametitle{Poisson Distribution (Example)}
\begin{itemize}

\item Suppose that electricity power failures occur according to a Poisson distribution
with an average of 2 outages every twenty weeks. \item Calculate the probability that there will
not be more than one power outage during a particular week.
\end{itemize}

\textbf{Solution:}

\begin{itemize}
\item The average number of failures per week is: $m = 2/20 = 0.10$

\item ``Not more than one  power outage" means we need to compute and add the probabilities for ``0 outages" plus ``1 outage".
\end{itemize}

}

%---------------------------------------------------------------%
\frame{
\frametitle{Poisson Distribution (Example)}

Recall: \[P(X = k) = e^{-m}\frac{m^k}{k!}\]


\begin{itemize}

\item $P(X = 0)$

\[P(X = 0) = e^{-0.10}\frac{0.10^0}{0!} = e^{-0.10} = 0.9048\]


\item $P(X = 1)$

\[P(X = 1) = e^{-0.10}\frac{0.10^1}{1!} = e^{-0.10}\times 0.1 = 0.0905\]

\item $P(X \leq 1)$

\[P(X \leq 1) = P(X = 0) + P(X = 1) = 0.9048 + 0.0905 = 0.995\]

\end{itemize}
}

\frame{
\frametitle{Implementation using \texttt{R}}

\begin{itemize}
\item Probability Density Function $P(X = k)$
\begin{itemize}
\item For a given poisson mean $m$, which in \texttt{R} is specified as \texttt{lambda} 
\item \texttt{dpois(k,lambda = ...)} 
\end{itemize}
\item Cumulative Density Function $P(X \leq k)$
\begin{itemize}
\item \texttt{ppois(k,lambda = ...)}
\end{itemize}
\end{itemize}

}


\begin{frame}[fragile]
\frametitle{Implementation using \texttt{R}}
From before: $P(X = 0)$ given than the mean number of occurrences is 2.

\begin{verbatim}
> dpois(0,lambda=2)
[1] 0.1353353
> dpois(1,lambda=2)
[1] 0.2706706
> dpois(2,lambda=2)
[1] 0.2706706
\end{verbatim}

\end{frame}


\begin{frame}[fragile]
\frametitle{Implementation using \texttt{R}}
Compute the cumulative distribution functions for the values $k=\{0,1,2\}$, given that the mean number of occurrences is 2

\begin{verbatim}
> ppois(0,lambda=2)
[1] 0.1353353
> ppois(1,lambda=2)
[1] 0.4060058
> ppois(2,lambda=2)
[1] 0.6766764
\end{verbatim}

\end{frame}
%---------------------------------------------------------------%
\frame{
\frametitle{Poisson Approximation of the Binomial}
\begin{itemize}
\item The Poisson distribution can sometimes be used to approximate the binomial distribution
\item When the number of observations n is large, and the success probability p is small, the $\mbox{Bin}(n,p)$ distribution approaches the Poisson distribution with the parameter given by $m = np$.
\item This is useful since the computations involved in calculating binomial probabilities are greatly reduced.
\item As a rule of thumb, n should be greater than 50 with p very small, such that $np$ should be less than 5.
\item If the value of $p$ is very high, the definition of what constitutes a ``success" or ``failure" can be switched.
\end{itemize}
}
%---------------------------------------------------------------%
\frame{
\frametitle{Poisson Approximation: Example}

Suppose we sample 1000 items from a production line that is producing, on average, 0.1\% defective components.\\


\bigskip

Using the binomial distribution, the probability of exactly 3 defective items in our sample is

\[P(X=3) = ^{1000}C_3 \times (0.001)^3 \times 0.999^{997} \]

}
%---------------------------------------------------------------%
\frame{
\frametitle{Poisson Approximation: Example}
Lets compute each of the component terms individually.


\begin{itemize}
\item $^{1000}C_3$

\[ ^{1000}C_3 = \frac{1000 \times 999 \times 998}{3 \times 2 \times 1} =
166,167,000 \]

\item $0.001^3$

\[0.001^3 = 0.000000001 \]


\item $0.999^{997}$

\[0.999^{997} = 0.36880 \]

\end{itemize}
Multiply these three values to compute the binomial probability \[P(X=3) = 0.06128 \]

}
%---------------------------------------------------------------%
%---------------------------------------------------------------%
\frame{
\frametitle{Poisson Approximation: Example}
\begin{itemize}
\item Lets use the Poisson distribution to approximate a solution.

\item First check that $n \geq 50$ and $np <5$ (Yes to both).

\item We choose as our parameter value $m = np = 0.001 \times 1000  = 1$

\[P(X=3) = e^{-1}\frac{1^3}{3!} = \frac{e^{-1}}{6} = \frac{0.36787}{6} =  0.06131\]
\item Compare this answer with the Binomial probability \\ $P(X=3) = 0.06128$.
\item Very good approximation, with much less computation effort.
\end{itemize}
}

\begin{frame}[fragile]
\frametitle{Implementation using \texttt{R}}


\begin{verbatim}
> # Poisson Mean m = 1000 * 0.001 = 1
> dbinom(3,size=1000,prob=0.001)
[1] 0.06128251
>
> dpois(3,lambda=1)
[1] 0.06131324
\end{verbatim}

\end{frame}

\end{document}
%---------------------------------------------------------------------------%
\frame{
\frametitle{Continuous Random Variables}

\begin{itemize}
\item Probability Density Function
\item Cumulative Density Function
\end{itemize}


If X is a continuous random variable then we can say that the probability of obtaining a \textbf{precise} value $x$ is infinitely small, i.e. close to zero.

\[P(X=x) \approx 0 \]

Consequently, for continuous random variables (only),  $P(X \leq x)$ and $P(X < x)$ can be used interchangeably.

\[P(X \leq x) \approx P(X < x) \]


}

%---------------------------------------------------------------------------%
\frame{
\frametitle{Continuous Uniform Distribution}
A random variable X is called a continuous uniform random variable over the interval $(a,b)$ if it's probability density function is given by

\[ f_{X}(x)  =  { 1 \over b-a}   \hspace{2cm}  \mbox{ when } a \leq x \leq b\]

The corresponding cumulative density function is

\[ F_x(x) = { x-a \over b-a}   \hspace{2cm}  \mbox{ when } a \leq x \leq b\]

}

%-----------------------------------------------------------------------------%

\frame{

The mean of the continuous uniform distribution is

\[ E(X) = {a+b \over 2}\]

\[ V(X) = {(b-a)^2\over12}\]
}

%-----------------------------------------------------------------------------%

\frame{
\frametitle{The Memoryless property}
The most interesting property of the exponential distribution is the \textbf{\emph{memoryless}} property. By this , we mean that if  the lifetime of a component is exponentially distributed, then an item which has been in use for some time is a good as a brand new item with regards to the likelihood of failure.

The exponential distribution is the only distribution that has this property.
}

%--------------------------------------------------------%

\frame{
\frametitle{Random Variables}
A pair of dice is thrown. Let X denote the minimum of the two numbers which occur.
Find the distributions and expected value of X.
}
%-------------------------------------------------------------%
\frame{
\frametitle{Random Variables}
A fair coin is tossed four times.
Let X denote the longest string of heads.
Find the distribution and expectation of X.}
%-------------------------------------------------------------%
\frame{\frametitle{Random Variables}
A fair coin is tossed until a head or five tails occurs.
Find the expected number E of tosses of the coin.}
%-------------------------------------------------------------%
\frame{\frametitle{Random Variables}A coin is weighted so that P(H) = 0.75 and P(T ) = 0.25

The coin is tossed three times. Let X denote the number of
heads that appear.
\begin{itemize}
\item (a) Find the distribution f of X.
\item (b) Find the expectation E(X).
\end{itemize}
}

%-------------------------------------------------------------%
\frame{
\begin{itemize}
\item Now consider an experiment with only two outcomes. Independent repeated trials of such an experiment are
called Bernoulli trials, named after the Swiss mathematician Jacob Bernoulli (1654�1705). \item The term \textbf{\emph{independent
trials}} means that the outcome of any trial does not depend on the previous outcomes (such as tossing a coin).
\item We will call one of the outcomes the ``success" and the other outcome the ``failure".
\end{itemize}
}

%-------------------------------------------------------------%
\frame{
\begin{itemize}
 \item
Let $p$ denote the probability of success in a Bernoulli trial, and so $q = 1 - p$ is the probability of failure.
A binomial experiment consists of a fixed number of Bernoulli trials. \item A binomial experiment with $n$ trials and
probability $p$ of success will be denoted by
\[B(n, p)\]
\end{itemize}
}
%-------------------------------------------------------------%

%---------------------------------------------------------------------------%
\frame{
\frametitle{Probability Mass Function}
\begin{itemize} \item a probability mass function (pmf) is a function that gives the probability that a discrete random variable is exactly equal to some value. \item The probability mass function is often the primary means of defining a discrete probability distribution \end{itemize}
}
%------------------------------------------------------------------%
\frame{
Thirty-eight students took the test. The X-axis shows various intervals of scores (the interval labeled 35 includes any score from 32.5 to 37.5). The Y-axis shows the number of students scoring in the interval or below the interval.

\textbf{\emph{cumulative frequency distribution}}A  can show either the actual frequencies at or below each interval (as shown here) or the percentage of the scores at or below each interval. The plot can be a histogram as shown here or a polygon.
}



%---------------------------------------------------------------%
\frame{
\frametitle{Notation for Poisson Distribution}
A discrete random variable $X$ is said to follow a Poisson distribution with parameter $m$, written $X \sim \mbox{Po}(m)$, if it has probability distribution


\[ P(X=k) = e^{-m} {m^k \over k!} \]

where
\begin{itemize}
\item $k = 0, 1, 2, \ldots$
\item $m > 0$.
\end{itemize}


}


\end{document}



%---------------------------------------------------------------------------------------------------------------%
%----R Code ----
%---------------------------------------------------------------------------------------------------------------%
n=60000
Y=numeric(n)
for ( i in 1:n){

X=floor(runif(100,1,7))
Y[i]=sum(X)
}

Y
hist(Y,breaks=seq(300,400,by=10),main=c("Totals of 100 Die Throws"),cex.lab=1.4,font.lab=2,xlab=c("Total Score"))

hist(Y,breaks=seq(300,400,by=20),main=c("Totals of 100 Die Throws"),cex.lab=1.4,font.lab=2,xlab=c("Total Score"))



Z=seq(1:n)
Y/Z

plot(Y/Z,type="l",col="red",main=c("Die Rolls: Running Average"),font.lab=2,ylab="Average Value", xlab=
" Number of Throws")
abline(h=3.5,col="green")


#####################################################

plot(Z,Z.y,pch=16,col="red",ylim=c(2.5,5.5),main=c("Variance"),font.lab=2,ylab=" ", xlab="X: Green  Y: Blue  Z: Red" )

points(Y,Y.y,pch=16,col="blue" )
points(X,X.y,pch=16,col="green" )
points(c(1000,1000,1000),c(3,4,5),pch=18,cex=1.2)
lines(c(1000,1000),c(2.75,5.25),lty=3)



n=100000
Y=numeric(n)
for ( i in 1:n){

X=floor(runif(100,1,7))
Y[i]=sum(X)
}

Y
hist(Y,breaks=seq(270,430,by=2),main=c("Totals of 100 Die Throws (n= 100,000)"),cex.lab=1.4,font.lab=2,xlab=c("Total Score")) 