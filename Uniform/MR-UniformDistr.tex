
\documentclass[a4]{beamer}
\usepackage{amssymb}
\usepackage{graphicx}
\usepackage{subfigure}

\usepackage{framed}
\usepackage{newlfont}
\usepackage{amsmath,amsthm,amsfonts}
%\usepackage{beamerthemesplit}
\usepackage{pgf,pgfarrows,pgfnodes,pgfautomata,pgfheaps,pgfshade}
\usepackage{mathptmx}  % Font Family
\usepackage{helvet}   % Font Family
\usepackage{color}


\setbeamercovered{dynamic}

\title[MathResource.com{Continuous Probability Distributions \\ {\normalsize The Uniform Distribution}}
\author[Kevin O'Brien]{Kevin O'Brien \\ {\scriptsize Kevin.obrien@ul.ie}}
\date{Spring Semester 2013}
\institute[Maths \& Stats]{Dept. of Mathematics \& Statistics, \\ University \textit{of} Limerick}

\renewcommand{\arraystretch}{1.5}

\begin{document}

\begin{frame}
\titlepage
\end{frame}
%---------------------------------------------------------------------------%
\frame{
\frametitle{MathsResource.com}
\Large
\begin{itemize}
\item The continuous uniform distribution is very simple to understand and implement, and is commonly used in computer applications (e.g. computer simulation).
\item It is also known as the `Rectangle Distribution' for obvious reasons.
\end{itemize}
}
%---------------------------------------------------------------------------%
\frame{
\frametitle{MathsResource.com}
\Large
\begin{itemize}
\item We specify the word ``continuous" so as to distinguish it from it's discrete equivalent: the discrete uniform distribution.
\item Remark; the dice distribution is a discrete uniform distribution with lower and upper limits 1 and 6 respectively.
\end{itemize}
}

%-----------------------------------------------------%
\frame{
\frametitle{Uniform Distribution Parameters}
\Large

The continuous uniform distribution is characterized by the following parameters

\begin{itemize}
\item The lower limit $a$
\item The upper limit $b$
\item We denote a uniform random variable $X$ as $X \sim U(a,b)$
\end{itemize}

It is not possible to have an outcome that is lower than $a$ or larger than $b$.

\[ P(X < a) = P(X > b) = 0\]
}

%---------------------------------------------------------------------------%
\frame{
\frametitle{MathResource.com}
%{Continuous Uniform Distribution}
\vspace{-1cm}
\Large
A random variable X is called a continuous uniform random variable over the interval $(a,b)$ if it's probability density function is given by
\[ f_{X}(x) = { 1 \over b-a} \hspace{1cm} \mbox{ when } a \leq x \leq b \]\[\mbox{     (otherwise } f_X(x) = 0 ) \]
}
%---------------------------------------------------------------------------%
\frame{
\frametitle{MathResource.com}
%{Continuous Uniform Distribution}
\Large
The corresponding cumulative density function is
\[ F_x(x) = { x-a \over b-a} \hspace{1cm} \mbox{ when } a \leq x \leq b\]
}

%----------------------------------------------------------------------------------------------------%
\frame{
%\frametitle{MathResource.com}
%{Continuous Uniform Distribution}
\Large

\vspace{-0.5cm}

\begin{center}
\includegraphics[scale=0.48]{6AUniform}

\end{center}
}



%------------------------------------------------------------------------%
\frame{\frametitle{Interval Probability}

\begin{itemize}
\item We wish to compute the probability of an outcome being within a range of values.
\item We shall call this lower bound of this range $L$ and the upper bound $ U$.
\item Necessarily $L$ and $U$ must be possible outcomes for the variable $X$.
\item The probability of $X$ being between $L$ and $U$ is denoted $P( L \leq X \leq U )$.

\begin{framed}
\[
P( L \leq X \leq U ) = { U - L \over b - a}
\]
\end{framed}
\item (This equation is based on a definite integral).
\end{itemize}
}

%---------------------------------------------------------------------------------------------------------%
\frame{
\frametitle{Uniform Distribution: Cumulative Distribution}
\Large
\begin{itemize}

\item For any value ``c" between the minimum value a and the maximum
value $b$, we can say
\item $P(X \geq c)$ \[P(X \geq c) = {b-c \over b-a}\]
here $b$ is the upper bound while $c$ is the lower bound
\item $P(X \leq c)$ \[P(X \leq c) = {c-a \over b-a}\]
here $c$ is the upper bound while $a$ is the lower bound.
\end{itemize}
}

%-----------------------------------------------------------------------------%
\frame{
\frametitle{Uniform Distribution: Mean and Variance}
\Large
\begin{itemize}
\item The Expected Value (in other words, the mean)  of the continuous uniform variable $X$ , with parameters $a$ and $b$ is
\[ \mathrm{E}(X) = {a+b \over 2}\]
\item The variance is computed as
\[ \mathrm{Var}(X) = {(b-a)^2 \over 12}\]
\end{itemize}
}
%------------------------------------------------------------------------%
\frame{
\frametitle{Uniform Distribution: Example}
\Large
\textbf{Example}
\begin{itemize}
\item Suppose there is a platform in a subway station in a very large city. \item Subway trains arrive \textbf{every three minutes} at this platform. \item What is the shortest possible time a passenger would have to wait for a train?
\item What is the longest possible time a passenger will have to wait?
\end{itemize}

}


%------------------------------------------------------------------------%
\frame{
\frametitle{Uniform Distribution: Example}
\Large

\begin{itemize}
 \item What is the shortest possible time a passenger would have to wait for a train?
%\begin{itemize}
\item If the passenger arrives just before the doors close, then the waiting time is zero.
\[ a = 0 \mbox{ minutes } \]
\end{itemize}
}


%------------------------------------------------------------------------%
\frame{
\frametitle{Uniform Distribution: Example}
\Large
\begin{itemize}
\item What is the longest possible time a passenger will have to wait?
%\begin{itemize}
\item Suppose a passenger arrives just after the train doors close, thereby missing the train. \item Then he or she will have to wait the full three minutes for the next one train to arrive
\[ b = 3 \mbox{ minutes }  = 180 \mbox{ seconds}  \]
\end{itemize}
%\end{itemize}

}

%------------------------------------------------------------------------%
\frame{
\frametitle{Uniform Distribution: Example}
\Large
\begin{itemize}
\item What is the probability that he will have to wait longer than 2 minutes?
\[ P(X \geq 2)  = {3-2 \over 3-0} = {1 \over 3} \approx 0.33   \]
\end{itemize}
%\end{itemize}

}

%----------------------------------------------------------------------------------------------------%
\begin{frame}
\frametitle{The Continuous Uniform Distribution}

\vspace{-0.75cm}

\begin{center}
\includegraphics[scale=0.50]{6AUniform4}

\end{center}
\end{frame}
%------------------------------------------------------------------------%
\frame{
\frametitle{Uniform Distribution: Example}
\Large
\vspace{-1cm}
\begin{itemize}
\item What is the longest probability that a passenger will have to wait less than than 45 seconds?
\item \textit{Remark : 45 seconds is 0.75 minutes}
\[ P(X \leq 0.75)  = {0.75 - 0 \over 3-0} = {0.75/3} = 0.25  \]
\end{itemize}
%\end{itemize}

}



%----------------------------------------------------------------------------------------------------%
\begin{frame}
\frametitle{The Continuous Uniform Distribution}

\vspace{-0.75cm}

\begin{center}
\includegraphics[scale=0.5]{6AUniform3}

\end{center}
\end{frame}
%------------------------------------------------------------------------%
\frame{
\frametitle{Uniform Distribution: Expected Value}
\Large
We are told that, for waiting times,  the lower limit $a$ is 0, and the upper limit $b$ is 3 minutes. \\ \bigskip The expected waiting time $\textrm{E}[X]$ is computed as follows
\vspace{0.1cm}
\[
\textrm{E}[X] = {a+b \over 2} =  {3 + 0  \over 2}  = 1.5 \mbox{ minutes }
\]

}
%------------------------------------------------------------------------%
\frame{\frametitle{Uniform Distribution: Variance}
\Large
The variance of the continuous uniform variable$X$ is denoted $\textrm{Var}[X]$ and  is  computed using the following formula:
\vspace{0.1cm}
\begin{framed}
\[
\textrm{Var}[X] = {(b - a)^2 \over 12}
\]
\end{framed}
\vspace{0.1cm}
For our subway train example:
\[
\textrm{Var}[X] = {(3 - 0)^2 \over 12} =  {3^2 \over 12} = {9 \over 12} = 0.75
\]
}
%------------------------------------------------------------------------%
\frame{\frametitle{Uniform Distribution: Variance}

End Slide
}



\end{document}

						
0	 < X <	0.16667	&	1	 \\ \hline	
0.16667	 < X <	0.33333	&	2	 \\ \hline	
0.33333	 < X <	0.50000	&	3	 \\ \hline	
0.50000	 < X <	0.66667	&	4	 \\ \hline	
0.66667	 < X <	0.83333	&	5	 \\ \hline	
0.83333	 < X <	1.00000	&	6	 \\ \hline	


%--------------------------------------------------------------------------------%

0.9041541					6	 \\ \hline
0.9305909					6	 \\ \hline
0.1866838					2	 \\ \hline
0.4378959					3	 \\ \hline
0.588299					4	 \\ \hline
0.1617157					1	 \\ \hline

 