
%---------------------------------------------------------------------------%
\frame{
\frametitle{Discrete Probability Distributions}
Three main distributions
\begin{itemize}
\item Binomial Distribution
\item Poisson Distribution
\item Geometric Distribution (mentioned, but not as important as the other two.)
\end{itemize}
}
%---------------------------------------------------------------------------%
\frame{
\frametitle{Binomial Probability Distribution}
Important Points:
\begin{itemize}
\item The experiment is a series of $n$ independent trials.
\item Two possible outcomes from each trial: a success and a failure.
\item The probability of success (i.e. $p$) is constant. 
\item A binomial random variable can be written as 
\[ X \sim B(n,p) \]
\end{itemize}
}
%---------------------------------------------------------------------------%
\frame{
\frametitle{Poisson Probability Distribution}
Important Points:
\begin{itemize}
\item This distribution is concerned with the number of occurrences per unit space.
\item Unit space can mean a unit length, a unit area, a unit volume or a unit period of time.
\item We will concern ourselves with unit time periods mostly.
\item A Poisson random variable can be written as
\[ X \sim Pois(m) \]
\item The Poisson distribution can be used to approximate the binomial distribution under certain conditions.
\end{itemize}
}
%---------------------------------------------------------------------------%
\frame{
\frametitle{PDFs and CDFs}
Important Points:
\begin{itemize}
\item The probability density function (PDF) is the probability of a random variable taking a specific value i.e. $P(X = k)$
\item The appropriate \texttt{R} functions are \texttt{dbinom} and \texttt{dpois}
\item The cumulative distribution function (CDF) is the probability of a random variable not exceeding a specific value i.e. $P(X \leq k)$
\item The appropriate \texttt{R} functions are \texttt{pbinom} and \texttt{ppois}     
\end{itemize}
}

%---------------------------------------------------------------------------%
\frame{
\frametitle{Geometric Probability Distribution}
Important Points:
\begin{itemize}
\item This distribution is closely related to the binomial distribution. 
\item This distribution described the number of failures that occur before the first success, when the probability of success is $p$.
\item The relevant \texttt{R} functions are \texttt{dgeom} and \texttt{pgeom}.
\end{itemize}
}

%---------------------------------------------------------------------------%
\begin{frame}[fragile]
\frametitle{Geometric Probability Distribution : Example}

If the probability of inserting a USB correctly is $0.40$, what is the probability of successfully doing so on the second attempt.
\bigskip.
In essence we have one failure, then one success, and these are independent events. So the probability the second attempt will be successful is $0.6 \times 0.4$. The probability that we are successful on the first attempt (i.e. no failures beforehand) is 0.4

\begin{verbatim}
> dgeom(0,prob=0.4)
[1] 0.4
> dgeom(1,prob=0.4)
[1] 0.24
> dgeom(2,prob=0.4)
[1] 0.144
\end{verbatim}

\end{frame}


