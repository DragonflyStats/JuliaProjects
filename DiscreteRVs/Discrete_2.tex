\documentclass[a4]{beamer}
\usepackage{amssymb}
\usepackage{graphicx}
\usepackage{subfigure}
\usepackage{newlfont}
\usepackage{amsmath,amsthm,amsfonts}
%\usepackage{beamerthemesplit}
\usepackage{pgf,pgfarrows,pgfnodes,pgfautomata,pgfheaps,pgfshade}
\usepackage{mathptmx}  % Font Family
\usepackage{helvet}   % Font Family
\usepackage{color}

\mode<presentation> {
 \usetheme{Frankfurt} % was Frankfurt
 \useinnertheme{rounded}
 \useoutertheme{infolines}
 \usefonttheme{serif}
 %\usecolortheme{wolverine}
% \usecolortheme{rose}
\usefonttheme{structurebold}
}

\setbeamercovered{dynamic}

\title[MA4413]{Statistics for Computing \\ {\normalsize MA4413 Lecture 4B}}
\author[Kevin O'Brien]{Kevin O'Brien \\ {\scriptsize Kevin.obrien@ul.ie}}
\date{Autumn Semester 2012}
\institute[Maths \& Stats]{Dept. of Mathematics \& Statistics, \\ University \textit{of} Limerick}

\renewcommand{\arraystretch}{1.5}

\begin{document}

\begin{frame}
\titlepage
\end{frame}

%---------------------------------------------------------------%
\frame{
\frametitle{Today's Class}
\begin{itemize}
\item Review of Discrete Probability Distributions
\item \texttt{R} Implementation
\item A Few Examples
\item Introduction to Continuous Probability Distributions
\item The Uniform Distribution
\end{itemize}
}
%---------------------------------------------------------------------------%
\frame{
\frametitle{Discrete Probability Distributions}
Three main distributions
\begin{itemize}
\item Binomial Distribution
\item Poisson Distribution
\item Geometric Distribution (mentioned, but not as important as the other two.)
\end{itemize}
}
%---------------------------------------------------------------------------%
\frame{
\frametitle{Binomial Probability Distribution}
Important Points:
\begin{itemize}
\item The experiment is a series of $n$ independent trials.
\item Two possible outcomes from each trial: a success and a failure.
\item The probability of success (i.e. $p$) is constant. 
\item A binomial random variable can be written as 
\[ X \sim B(n,p) \]
\end{itemize}
}
%---------------------------------------------------------------------------%
\frame{
\frametitle{Poisson Probability Distribution}
Important Points:
\begin{itemize}
\item This distribution is concerned with the number of occurrences per unit space.
\item Unit space can mean a unit length, a unit area, a unit volume or a unit period of time.
\item We will concern ourselves with unit time periods mostly.
\item The average number of occurrence per unit period is denoted $m$ i.e. the Poisson mean. In many texts the Poisson mean is denoted $\lambda$. \item If the unit period changes, then the Poisson mean changes accordingly.
\item A Poisson random variable can be written as
\[ X \sim Pois(m) \]
\item The Poisson distribution can be used to approximate the binomial distribution under certain conditions.
\end{itemize}
}
%---------------------------------------------------------------------------%
\frame{
\frametitle{Implementation in \texttt{R}}
Important Points:
\begin{itemize}
\item The binomial parameters $n$ and $p$ are written as \texttt{size} and \texttt{prob}.
\item The Poisson mean $m$ is written as \texttt{lambda}. \bigskip
\item The probability density function (PDF) is the probability of a random variable taking a specific value i.e. $P(X = k)$
\item The appropriate \texttt{R} functions are \texttt{dbinom} and \texttt{dpois}.\bigskip
\item The cumulative distribution function (CDF) is the probability of a random variable not exceeding a specific value i.e. $P(X \leq k)$
\item The appropriate \texttt{R} functions are \texttt{pbinom} and \texttt{ppois}.
\end{itemize}
}

%---------------------------------------------------------------------------%
\begin{frame}[fragile]
\frametitle{Implementation in \texttt{R}}
\begin{itemize}
\item We can compute probabilities for a series of values, rather than just one at a time.
\item For this, we will use sequences using the ``:" operator.
\end{itemize}
\begin{verbatim}
> 2:7
[1] 2 3 4 5 6 7
> dpois(2:7,lambda=3)
[1] 0.22404181 0.22404181 0.16803136 0.10081881 0.05040941
[6] 0.02160403
\end{verbatim}
\begin{itemize}
\item Poisson process with mean $m=3$
\item $P(X=2) =0.224$, $P(X=3) = 0.224$, $P(X=4) = 0.168)$ etc
\end{itemize}
\end{frame}

%---------------------------------------------------------------------------%
\frame{
\frametitle{Geometric Probability Distribution}
Important Points:
\begin{itemize}
\item This distribution is closely related to the binomial distribution.
\item This distribution described the number of failures that occur before the first success, when the probability of success is $p$.
\item The relevant \texttt{R} functions are \texttt{dgeom} and \texttt{pgeom}.
\end{itemize}
}

%---------------------------------------------------------------------------%
\begin{frame}[fragile]
\frametitle{Geometric Probability Distribution : Example}

If the probability of inserting a USB correctly is $0.40$, what is the probability of successfully doing so on the second attempt.\\
\bigskip
In essence we have one failure, then one success, and these are independent events. So the probability the second attempt will be successful is $0.6 \times 0.4$. The probability that we are successful on the first attempt (i.e. no failures beforehand) is 0.4\\
\bigskip
Question: What is the probability of needing more than two attempts?
\begin{verbatim}
> dgeom(0,prob=0.4)
[1] 0.4
> dgeom(1,prob=0.4)
[1] 0.24
> dgeom(2,prob=0.4)
[1] 0.144
\end{verbatim}

\end{frame}




%---------------------------------------------------------------------------%
\begin{frame}[fragile]
\frametitle{Binomial Probability Distribution: Example}

\begin{itemize}
\item Consider a binomial experiment with $n = 20$ and $p = 0.50$.
\item Use the following output to compute $P(X > 10)$
\end{itemize}
\begin{verbatim}
> dbinom(9,size=20,prob=0.50)
[1] 0.1601791
> dbinom(10,size=20,prob=0.50)
[1] 0.1761971
> dbinom(11,size=20,prob=0.50)
[1] 0.1601791
>
> pbinom(9,size=20,prob=0.50)
[1] 0.4119015
> pbinom(10,size=20,prob=0.50)
[1] 0.5880985
> pbinom(11,size=20,prob=0.50)
[1] 0.7482777
\end{verbatim}
\end{frame}

%---------------------------------------------------------------------------%
\begin{frame}[fragile]
\frametitle{Binomial Probability Distribution: Example}

\begin{itemize}
\item Consider the sample space for the number of successes. There can be between 0 and 20 successes.

\[ S = \{0,1,2,3,\ldots,8,9,10,11,12 \ldots,19,20\}
\]

\item  $P(X > 10)$ is the probability of X being greater than 10.
\item This is equivalent to the probability of X being X or greater: i.e. $P(X \geq 11)$.
\item We can't determine this directly using our \texttt{R} output.

\item We can determine the probability of the complementary event. i.e. $P(X \leq 10)$ using calculators.
\end{itemize}
\begin{verbatim}
> pbinom(10,size=20,prob=0.50)
[1] 0.5880985
>
> 1-pbinom(10,size=20,prob=0.50)
[1] 0.4119015
\end{verbatim}
\end{frame}
%---------------------------------------------------------------------------%

%---------------------------------------------------------------------------%
\frame{
\frametitle{Sample Space and Sample Points }

\begin{itemize}
\item Knowledge and use of sample points, and the sample space is quite helpful for these questions.
\item What are the sample points for the event where the number of success is less than or equal to 9? (Lets call this event $A$.)
\[ A = \{0,1,2,3,\ldots,8,9\}
\]
\item What are the sample points for the \textbf{\emph{complement event}}. (Lets call this event $A^c$).
\[ A^c = \{10,11,12 \ldots,19,20\}
\]

\end{itemize}
}



\end{document}

