\documentclass[a4]{beamer}
\usepackage{amssymb}
\usepackage{graphicx}
\usepackage{subfigure}
\usepackage{newlfont}
\usepackage{amsmath,amsthm,amsfonts}
%\usepackage{beamerthemesplit}
\usepackage{pgf,pgfarrows,pgfnodes,pgfautomata,pgfheaps,pgfshade}
\usepackage{mathptmx}  % Font Family
\usepackage{helvet}   % Font Family
\usepackage{color}

\mode<presentation> {
 \usetheme{Default} % was Frankfurt
 \useinnertheme{rounded}
 \useoutertheme{infolines}
 \usefonttheme{serif}
 %\usecolortheme{wolverine}
% \usecolortheme{rose}
\usefonttheme{structurebold}
}

\setbeamercovered{dynamic}

\title[MA4413]{Statistics for Computing (MA4413) } % \\ {\normalsize {Lecture 1A}}
\author[Kevin O'Brien]{Kevin O'Brien \\ {\scriptsize Kevin.obrien@ul.ie}}
\date{Autumn Semester 2011 : Lecture 1A}
\institute[Maths \& Stats]{Dept. of Mathematics \& Statistics, \\ University \textit{of} Limerick}

\renewcommand{\arraystretch}{1.5}

\begin{document}

\begin{frame}
\titlepage
\end{frame}



\section{Lecture 1A : About this module}

\frame{
\frametitle{MA4413 : Statistics For Computing}
\textbf{Lectures : }
\begin{itemize}
\item Monday 12-13 in Room $C063$
\item Tuesday 13-14 in Room $S206$
\item No classes on Bank Holiday Monday and during the open days.
\item Tutorials
\end{itemize}


}

%-----------------------------------------------------------%
\frame{
\frametitle{Today's Lecture}
\begin{itemize}
\item Description of module syllabus
\item Details on the assessment
\item Teaching materials
\begin{itemize}
\item Lecture notes : available on SULIS system
\item Murdoch Barnes statistical tables
\end{itemize}
\item \texttt{R} code, a statistical programming language, will also be included.
\end{itemize}

}
%----------------------------------------------------------%

\frame{
\frametitle{Syllabus}
The syllabus is made up of the following subject areas :
\begin{itemize}
\item Probability theory,
\item Descriptive statistics and graphical methods,
\item Probability distributions,
\item Information theory and data compression,
\item Confidence intervals and hypothesis testing.
\end{itemize}

}
%----------------------------------------------------------%
\frame{
\frametitle{Learning Outcomes}
On successful completion of this module, students should be able to:
\begin{enumerate}
\item Apply probability theory to problem solving.

\item Employ the concepts of random variables and probability distributions to problem solving.

\item Apply information theory to solve problems in data compression and transmission.

\item Analyse rates and proportions.

\item Perform hypothesis tests for a variety of statistical problems.
\end{enumerate}
}

%---------------------------------------------------------%
\frame{
\frametitle{Module Assessment}
\begin{itemize}
\item End of semester examination (85\%).
\item Mid term examination (15\%).
\begin{itemize}
\item Provisionally scheduled for Tuesday (Week 8).
\item I will re-arrange the date if needs be.
\end{itemize}
\item Past papers will be made available through the SULIS system.
\end{itemize}

}




%----------------------------------------------%
\frame{
\frametitle{Definitions}
%--http://www.stats.gla.ac.uk/steps/glossary/probability.html#probability
\begin{enumerate}
\item Probability
\item Random Experiment %Yes
\item Outcome %Yes
\item Sample Space %Yes
\item Event %Yes
\item Conditional Probability %Yes
\item Independent Events %Yes
\item Mutually Exclusive Events %Yes
\item Addition Rule %Yes
%\item Multiplication Rule
%\item Law of Total Probability
%\item Bayes' Theorem
\end{enumerate}
}



%------------------------------------------------------------------%
\frame{
\frametitle{Probability}
\begin{itemize}
\item Probability theory is the mathematical study of randomness. A probability model of a random experiment is
defined by assigning probabilities to all the different outcomes. \item Probability is a numerical measure of the likelihood that an event will occur. Thus, probabilities can be used as measures of degree of uncertainty associated with outcomes of an experiment.

\item Probability values are always assigned on a scale from $0$ to $1$.

\item A probability of $0$ means that the event is impossible, while a probability near $0$
means that it is highly unlikely to occur. \item Similarly an event with probability $1$ is certain to occur, whereas an event with a probability near to $1$ is very likely to occur.
\end{itemize}
}
\frame{
\section{Definitions}
%------------------------------------------------------------------%
\frametitle{Experiments and Outcomes}
\begin{itemize}
\item In the study of probability any process of observation is referred to as an \textbf{\emph{experiment}}.
\item The results of an experiment (or other situation involving uncertainty) are called the \textbf{\emph{outcomes }}of the experiment.
\item An experiment is called a \textbf{\emph{random experiment}} if the outcome can not be predicted.
\item Typical examples of a random experiment are \begin{itemize} \item a role of a die, \item a toss of a coin, \item drawing a card from a deck.
        \end{itemize}
 \item If the experiment is yet to be performed we refer to `possible outcomes' or `possibilities' for short. If the experiment has been performed, we refer to `realized outcomes' or `realizations'.
\end{itemize}
}

%--------------------------------------------------------%
\frame{
\frametitle{Sample Spaces and Events}
\begin{itemize}
\item The set of all possible outcomes of a probability experiment is called a \textbf{\emph{sample space}}, which is usually denoted by $\boldsymbol{S}$.
\item The sample space is an exhaustive list of all the possible outcomes of an experiment. We call individual elements of this list \textbf{\emph{sample points}}.
\item Each possible outcome is represented by one and only one sample point in the sample space.
\end{itemize}
}
%--------------------------------------------------------%
\frame{
\frametitle{Example} 
For each of the following experiments, write out the sample space.
\begin{itemize}
\item Experiment: Rolling a die once
\begin{itemize}
\item Sample space $\boldsymbol{S} = \{1,2,3,4,5,6 \}$ \end{itemize}
\item Experiment:  Tossing a coin
\begin{itemize}
\item Sample space $\boldsymbol{S} = \{\mbox{ Heads },\mbox{ Tails} \}$ \end{itemize}
\item Experiment:  Measuring a randomly selected person's height (cms)
\begin{itemize} \item Sample space $\boldsymbol{S} = \mbox{ The set of all possible real numbers }$
\end{itemize}\end{itemize}
}

%--------------------------------------------%
\frame{
\frametitle{Events}
\begin{itemize}
\item
An \textbf{\emph{event}} is a specific outcome, or any collection of outcomes of an experiment.
\item Formally, any subset of the sample space is an event.
\item Any event which consists of a single outcome in the sample space is called an \textbf{\emph{elementary}}  or \textbf{\emph{simple event}}.
\item Events which consist of more than one outcome are called \textbf{\emph{compound events}}.
\item For example, an elementary event associated with the die example could be the ``die shows $3$".
\item An compound event associated with the die example could be the ``die shows an even number".
\end{itemize}
}
%--------------------------------------------%
\frame{
\frametitle{The Complement Event}
\begin{itemize}
\item The complement of an event $A$ is the set of all outcomes in the sample space that are not included in the outcomes of event $A$.  
    \item We call the complement event of $A$ as $A^c$.
\item    
The complement event of a die throw resulting in an even number is the die throwing an odd number.
\end{itemize}
}


%--------------------------------------------%

\frame{

\frametitle{Set Theory : Union and Intersection}
Set theory is used to represent relationships among events.\\ \bigskip
\textbf{Union of two events: }\\
The union of events $A$ and $B$ is the event containing all the sample points belonging to A or B or both.
This is denoted $A \cup B$, (pronounce as ``A union B").\\ \bigskip

\textbf{Intersection of two events:  }\\
The intersection of events $A$ and $B$ is the event containing all the sample points common to both $A$ and $B$. This is denoted
$A \cap B$, (pronounce as ``A intersection B").


}


%-----------------------------------------------------------%
\frame{
\frametitle{More Set Theory}
 In general, if $A$ and $B$ are two events in the sample space $\boldsymbol{S}$, then
\begin{itemize}
\item $A \subseteq B$(A is a subset of B) = 'if $A$ occurs, so does $B$'.\\
\item $\boldsymbol{\oslash}$ (the empty set) = an impossible event.\\
\item $S$ (the sample space) = an event that is certain to occur.\\
\end{itemize}
}
%--------------------------------------------%

\frame{
\frametitle{Examples of Events}
Consider the experiment of rolling a die once. From before, the sample space is given as $S = \{1,2,3,4,5,6\}$. The following are examples of possible events.

\begin{itemize}
\item $A$ = score $< 4$ = $\{1,2,3\}$.
\item $B$ = `score is even' $= \{2,4,6\}$
\item $C$ = `score is 7' = $\boldsymbol{\oslash}$\\
\item $A \cup B$ = `the score is $< 4$ or even or both' = $\{1,2,3,4,6\}$\\
\item $A \cap B$ = `the score is $< 4$ and even' = $\{2\}$\\
\item $A^C$  = 'event $A$ does not occur' = $\{4,5,6\}$
\end{itemize}
}
%--------------------------------------------%
\frame{
\frametitle{Probability}
If there are n possible outcomes to an experiment, and $m$ ways in which event $A$ can happen, then the probability of event $A$ ( which we write as $P(A)$) is
\[P(A) = \frac{m}{n}.\]

The probability of the event $A$ may be interpreted as the proportion of times that event $A$ will occur if we repeat the random experiment an infinite number of times.\\ \bigskip

\textbf{Rules}
\begin{enumerate}
\item $0 \leq P(A) \leq 1$ ; the probability of any event lies between 0 and 1 inclusive.
\item P(S) = 1 ; the probability of the sample space is always equal to 1.
\item $P(A^c) = 1- P(A)$; how to compute the probability of the complement.
\end{enumerate}
}
%---------------------------------------------------------------%
\section{Conditional Probability}
%--------------------------------------------------------------------------------------%
\frame{
\frametitle{Conditional Probability}
%The conditional probability of an event A given event B, denoted P(A|B) is defiend as
%P(A|B)  = \frac{P(A \mbox{ and } B)}{P(B)}  where $ P(B) > 0$
% Two events A and B are said to be statistically independent if and only if  P(A and B) = P(A) \times P(B)


Suppose $B$ is an event in a sample space $S$ with $P(B) > 0$. \\\bigskip The probability that an event $A$ occurs once $B$ has occurred or, specifically, the conditional probability of $A$ given $B$, written $P(A|B)$, is defined as follows:
\[P(A|B) = { P(A \cap B) \over P(B) }\]
This can be expressed as a multiplication theorem
\[P(A \cap B) = P(A|B)\times P(B) \]
\begin{itemize}
\item The symbol $|$ is a vertical line and does not imply division. 
\item Also $P(A|B)$ is not the same as $P(B|A)$.
\end{itemize}
}

%---------------------------------------------------------------%
\section{Independent and Mutually Exclusive Events}
\frame{
\frametitle{Independent Events}

Events $A$ and $B$ in a probability space $\boldsymbol{S}$ are said to be \textbf{\emph{independent}} if the occurrence of one of them does not influence the occurrence of the other.\\ \bigskip
More specifically, $B$ is independent of $A$ if $P(B)$ is the same as $P(B|A)$.
Now substituting $P(B)$ for $P(B|A)$ in the multiplication theorem from the previous slide yields
\[P(A \cap B) = P(A)\times P(B).\]

We formally use the above equation as our definition of independence.
}
%--------------------------------------------------------------%
\frame{
\frametitle{Mutually Exclusive Events}
\begin{itemize}
\item Two events are mutually exclusive (or disjoint) if it is impossible for them to occur together.
\item Formally, two events $A$ and $B$ are mutually exclusive if and only if
\[A \cap B = \oslash
\]
\end{itemize}

Consider our die example
\begin{itemize}
\item Event $A$ = 'observe an odd number' = $\{1,3,5\}$
\item Event $B$ = 'observe an even number' = $\{2,4,6\}$
\end{itemize}
$A \cap B = \oslash $ = the empty set, so A and B are mutually exclusive.\\

}
%----------------------------------------------------------------------%
\section{Important Rules for Probability}
\frame{
\frametitle{Addition Rule}
The addition rule is a result used to determine the probability that event $A$ or event $B$ occurs or both occur. The result is often written as follows, using set notation:
\[ P(A \cup B) = P(A) + P(B) - P(A \cap B) \]

\begin{itemize}
\item $P(A)$ = probability that event $A$ occurs.
\item $P(B)$ = probability that event $B$ occurs.
\item $P(A \cup B)$ = probability that either event $A$ or event $B$ occurs, or both occur.
\item $P(A \cap B)$ = probability that event $A$ and event $B$ both occur.
\end{itemize}

\textbf{Remark: }$P(A \cap B)$ is subtracted to prevent the relevant outcomes being counted twice.
}
\frame{
\frametitle{Addition Rule (Continued)}
For mutually exclusive events, that is events which cannot occur together:
$P(A \cap B) = 0$. The addition rule therefore reduces to
\[ P(A \cup B) = P(A) + P(B) \]

}
%---------------------------------------------------------------------------------------%
\frame{
\frametitle{Addition Rule: Worked Example}
Suppose we wish to find the probability of drawing either a Queen or a Heart in a single draw from a pack of 52 playing cards.
We define the events $Q$ = `draw a queen' and $H$ = `draw a heart'
\begin{itemize}
\item $P(Q)$ probability that a random selected card is a Queen
\item $P(H)$ probability that a randomly selected card is a Heart.
\item $P(Q \cap H)$ probability that a randomly selected card is the Queen of Hearts.
\item $P(Q \cup H)$ probability that a randomly selected card is a Queen or a Heart.
\end{itemize}
}
%---------------------------------------------------------------%
\frame{
\frametitle{Solution}
\begin{itemize}
\item Since there are 4 Queens in the pack and 13 Hearts, so the $P(Q) = 4/52$ and $P(H) = 13/52$ respectively.
\item
The probability of selecting the Queen of Hearts is $P(Q \cap H) = 1/52$.
\item We use the addition rule to find $P(Q \cup H)$:
\[P(Q \cup H) = 4/52 + 13/52 - 1/52 = 16/52 \]

So, the probability of drawing either a queen or a heart is $16/52 (= 4/13)$.
\end{itemize}

}
\frame{
\frametitle{Tommorow's Class}
\begin{itemize}
\item Continue with probability theory
\item Look at some more worked examples  
\item Introduce the concept random variables
\end{itemize}

}

%----------------------------------------------------------------------------%
\end{document}

