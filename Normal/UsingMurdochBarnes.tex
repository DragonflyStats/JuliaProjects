
%------------------------------------------------------------------------%
\section{Using the Murdoch Barnes Tables 3}
\frame{
\frametitle{Using Murdoch Barnes Tables 3}
\begin{itemize}
\item For some value $z_o$, between 0 and 4, the Murdoch Barnes tables set 3 tabulate $P(Z \geq z_o)$
\item Ideally $z_o$ would be specified to 2 decimal places. If it is not, round to the closest value.
\item We call the third digit (i.e. the digit in the second decimal place) the ``second precision".
\end{itemize}
}

%------------------------------------------------------------------------%
\frame{
\frametitle{Using Murdoch Barnes Tables 3}
\begin{itemize}
\item To compute the relevant probability we express $z_o$ as the sum of $z_o$ without the second precision, and the second precision.(For example $1.28 = 1.2 + 0.08$.)
\item Select the row that corresponds to $z_o$ without the second precision (e.g. 1.2).
\item Select the column that corresponds to the second precision(e.g. 0.08).
\item The value that contained on the intersection is $P(Z \geq z_o)$
\end{itemize}
}

%------------------------------------------------------------------------%
\frame{
\begin{table}[ht]
\frametitle{Find $ P(Z \geq 1.28)$}
\vspace{-1.5cm}
%\caption{Standard Normal Distribution } % title of Table
\centering % used for centering table
\begin{tabular}{|c|| c c c c c c|} % centered columns (4 columns)
\hline %inserts double horizontal lines
& \ldots & \ldots & 0.006 &0.07&0.08&0.09 \\
%heading
\hline \hline% inserts single horizontal line
\ldots & \ldots & \ldots &\ldots& \ldots &\ldots&\dots \\ % inserting body of the table
1.0 & \ldots & \ldots &0.1446& 0.1423 &0.1401&0.1379 \\ % inserting body of the table
1.1 & \ldots & \ldots&0.1230& 0.1210 &0.1190&0.1170 \\ % inserting body of the table
1.2 & \ldots & \ldots&0.1038 & 0.1020 &\alert{0.1003}&0.0985\\
1.3 & \ldots & \ldots &0.0869& 0.0853 &0.0838&0.0823 \\ % inserting body of the table
\ldots & \ldots &\ldots&\ldots & \ldots &\ldots&\ldots\\
\hline %inserts single line
\end{tabular}
%\label{table:nonlin} % is used to refer this table in the text
\end{table}
}

%------------------------------------------------------------------------%
\frame{
\frametitle{Using Murdoch Barnes tables 3}
\begin{itemize}
\item Find $ P(Z \geq 0.60)$
\item Find $ P(Z \geq 1.64)$
\item Find $ P(Z \geq 1.65)$
\item Estimate $P( Z \geq 1.645)$
\end{itemize}
}

%------------------------------------------------------------------------%
\frame{
\begin{table}[ht]
\frametitle{Find $ P(Z \geq 0.60)$}
\vspace{-1.5cm}
%\caption{Standard Normal Distribution } % title of Table
\centering % used for centering table
\begin{tabular}{|c|| c c c c c c|} % centered columns (4 columns)
\hline %inserts double horizontal lines
& 0.00 & 0.01 & 0.02 &0.03&\ldots&\ldots \\
%heading
\hline \hline% inserts single horizontal line \hline
\ldots & \ldots &\ldots &\ldots& \ldots &\ldots&\ldots \\ % inserting body of the table
0.4 & 0.3446 & 0.3409&0.3372 & 0.3336 &\ldots&\ldots\\
0.5 & 0.3085 & 0.3050 &0.3015& 0.2981 &\ldots&\dots \\ % inserting body of the table
0.6 & \alert{0.2743} & 0.2709&0.2676 & 0.2643 &\ldots&\ldots\\
0.7 & 0.2420 & 0.2389 &0.2358& 0.2327 &\ldots&\dots \\ % inserting body of the table
\ldots & \ldots &\ldots &\ldots& \ldots &\ldots&\ldots \\ % inserting body of the table
\hline %inserts single line
\end{tabular}
%\label{table:nonlin} % is used to refer this table in the text
\end{table}
}

%------------------------------------------------------------------------%
\frame{
\begin{table}[ht]
\frametitle{Find $ P(Z \geq 1.64)$ and $ P(Z \geq 1.65)$}
\vspace{-1.5cm}
%\caption{Standard Normal Distribution } % title of Table
\centering % used for centering table
\begin{tabular}{|c|| c c c c c c|} % centered columns (4 columns)
\hline %inserts double horizontal lines
& \ldots & \ldots & 0.04 & 0.05 &0.06&0.07 \\
%heading
\hline \hline% inserts single horizontal line
\ldots & \ldots &\ldots &\ldots& \ldots &\ldots&\ldots \\ %Checked
1.5 & \ldots & 0.0630&0.0618& 0.0606 &0.0594&\dots \\ % inserting body of the table
1.6 & \ldots &0.0516& \alert{0.0505} & \alert{0.0495} &0.0485&\ldots\\
1.7 & \ldots &0.0418 &0.0409& 0.0401 &0.0392&\dots \\ % inserting body of the table
\ldots & \ldots &\ldots &\ldots& \ldots &\ldots&\ldots \\ %Checked
\hline %inserts single line
\end{tabular}
%\label{table:nonlin} % is used to refer this table in the text
\end{table}
}

%------------------------------------------------------------------------%
\frame{
\frametitle{Using Murdoch Barnes tables 3}
\begin{itemize}
\item $ P(Z \geq 1.64) = 0.505$
\item $ P(Z \geq 1.65) = 0.495$ \bigskip
\item $ P(Z \geq 1.645)$ is approximately the average value of $ P(Z \geq 1.64)$ and $ P(Z \geq 1.65)$.
\item $ P(Z \geq 1.645)$ = (0.0495 + 0.0505)/2 = 0.0500. ( i.e. $5\%$ )
\end{itemize}
}
%------------------------------------------------------------------%
\frame{
\frametitle{Exact Probability}
\large
\alert{Remarks:} This is for continuous distributions only.
\begin{itemize}
\item The probability that a continuous random variable will take an exact value is infinitely small.
We will usually treat it as if it was zero.
\item
When we write probabilities for continuous random variables in mathematical notation, we often retain the equality component (i.e. the "...or equal to..").\\
For example, we would write expressions $P(X \leq 2)$ or $P(X \geq 5)$.
\item
Because the probability of an exact value is almost zero, these two expression are equivalent to $P(X < 2)$
or $P(X > 5)$. \item The complement of $P(X \geq k)$ can be written as $P(X \leq k)$.
\end{itemize}
}

%%%%%%%%%%%%%%%%%%%%%%%%%%%%%%%%%%%%%%%%%%%%%%%%%%%%%%%%%%%%%%%%%%%%%%%%%%%%%%%%%%%%%%%%%%%%%%%%%%
