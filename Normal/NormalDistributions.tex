\documentclass[a4]{beamer}
\usepackage{amssymb}
\usepackage{graphicx}
\usepackage{subfigure}
\usepackage{newlfont}
\usepackage{amsmath,amsthm,amsfonts}
%\usepackage{beamerthemesplit}
\usepackage{pgf,pgfarrows,pgfnodes,pgfautomata,pgfheaps,pgfshade}
\usepackage{mathptmx}  % Font Family
\usepackage{helvet}   % Font Family
\usepackage{color}

\mode<presentation> {
 \usetheme{Default} % was Frankfurt
 \useinnertheme{rounded}
 \useoutertheme{infolines}
 \usefonttheme{serif}
 %\usecolortheme{wolverine}
% \usecolortheme{rose}
\usefonttheme{structurebold}
}

\setbeamercovered{dynamic}

\title[MA4413]{MA4413 Statistics for Computing \\ {\normalsize Lecture 6B : Revision}}
\author[Kevin O'Brien]{Kevin O'Brien \\ {\scriptsize kevin.obrien@ul.ie}}
\date{Autumn 2012}
\institute[Maths \& Stats]{Dept. of Mathematics \& Statistics, \\ University \textit{of} Limerick}


\renewcommand{\arraystretch}{1.5}
%------------------------------------------------------------------------%

\begin{document}

%------------------------------------------%

\begin{frame}[fragile]
\frametitle{Normal Distribution: Probability Density Function}
Recall: As the Normal distribution is a continuous distribution, the PDF for a particular observed value will not give us an intuitive
result (as far as this module is concerned). It is, in fact, the height of the density curve at a particular point.\\ \bigskip

Nonetheless, the relevant \texttt{R} code may be included in exam questions, so as to add complexity to questions.

\begin{verbatim}
> dnorm(0.7)
[1] 0.3122539
>
> dnorm(1.7)
[1] 0.09404908

\end{verbatim}
\end{frame}
%------------------------------------------%

\begin{frame}[fragile]
\frametitle{\texttt{R} Implementation}
\[ z_o = \frac{x_o - \mu}{\sigma}  = \frac{2340-2000}{200} = 1.7\]

Using the following \texttt{R} code, we can determine $P(Z \leq 1.7)$.
\begin{verbatim}

> pnorm(1.7)
[1] 0.9554345

\end{verbatim}
\end{frame}

%------------------------------------------%

\begin{frame}[fragile]
\frametitle{Direct \texttt{R} Implementation}

This can easily be implemented directly - without using the standardisation formula.
However, we will not be using this approach in this module.
\begin{verbatim}

> pnorm(2340,mean=2000,sd=200)
[1] 0.9554345


\end{verbatim}
\end{frame}

%------------------------------------------%
\end{document}




