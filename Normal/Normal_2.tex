
%------------------------------------------------------------------%
\frame{
\frametitle{ Characteristics of the Normal probability distribution}
\begin{itemize}
\item[1]  the mean, which is also the median and mode of the distribution. \smallskip
\item[2] \alert{[VERY IMPORTANT]}
The normal probability curve is bell-shaped and symmetric, with the shape of the curve to the left of the mean a mirror image of the shape of the curve to the right of the mean. (This is the basis of an important rule, called the \textbf{Symmetry Rule}, that we shall meet later.) \smallskip
\item[3] The standard deviation determines the width of the curve. Larger values of the the standard deviation result in wider flatter curves, showing more dispersion in data.
\item[4] As with all density curves, the total area under the curve for the normal probability distribution is 1.
\end{itemize}
}


%------------------------------------------------------------------%
\frame{
\frametitle{Exact Probability}
\large
\alert{Remarks:} This is for continuous distributions only.
\begin{itemize}
\item The probability that a continuous random variable will take an exact value is infinitely small.
We will usually treat it as if it was zero.
\item
When we write probabilities for continuous random variables in mathematical notation, we often retain the equality component (i.e. the "...or equal to..").\\
For example, we would write expressions $P(X \leq 2)$ or $P(X \geq 5)$.
\item
Because the probability of an exact value is almost zero, these two expression are equivalent to $P(X < 2)$
or $P(X > 5)$. \item The complement of $P(X \geq k)$ can be written as $P(X \leq k)$.
\end{itemize}
}





%-----------------------------------------------------%
\begin{frame}
\frametitle{Normal Distribution : Solving problems}
Recap:
\begin{itemize}
\item We must know the normal mean $\mu$ and the normal standard deviation $\sigma$.
\item The normal random variable is $X \sim \mbox{N} ( \mu , \sigma^2)$.\smallskip
\item (If we don't, we usually have to determine them, given the information in the question.)\smallskip
\item The standard normal random variable is $Z\sim \mbox{N} ( 0 , 1^2)$.\smallskip
\item The standard normal distribution is well described in Murdoch Barnes Table 3, which tabulates $P(Z \geq z_o)$ for a range of $Z$ values.
\end{itemize}
\end{frame}
%-----------------------------------------------------%
\begin{frame}
\frametitle{Normal Distribution : Solving problems}
\begin{itemize}
\item For the given value $x_o$ from the variable $X$, we compute the corresponding z-score $z_o$.
\[ z_o = { x_o - \mu \over \sigma} \]
\item When $z_o$ corresponds to $x_o$, the following identity applies:
\[  P(X \geq x_o )= P(Z \geq z_o ) \]
\item Alternatively $ P(X \leq x_o )= P(Z \leq z_o ) $
\end{itemize}
\end{frame}
%-----------------------------------------------------%
\begin{frame}
\frametitle{Normal Distribution : Solving problems}
\begin{itemize}
\item \textbf{Complement Rule}: \[ P(Z \leq k) = 1-P(Z \geq k) \] for some value $k$
\item Alternatively $ P(Z \geq k) = 1-P(Z \leq k) $
\item \textbf{Symmetry Rule}: \[ P(Z \leq -k) = P(Z \geq k) \] for some value $k$
\item Alternatively $ P(Z \geq -k) = P(Z \leq k) $
\end{itemize}
\end{frame}
%-----------------------------------------------------%
\begin{frame}
\frametitle{Normal Distribution : Solving problems}
\begin{itemize}

\item \textbf{Intervals}: \[ P(L \leq Z \leq U) = 1- [ P(Z \leq L) + P(Z \geq U)] \]
where $L$ and $U$ are the lower and upper bounds of an interval.
\item Probability of having a value too low for the interval : $P(Z \leq L)$
\item Probability of having a value too high for the interval : $P(Z \geq U)$
\end{itemize}
\end{frame}
%-----------------------------------------------------%


%\frame{
%\frametitle{Using Murdoch Barnes Tables 3}
%
%Find $ P(Z \geq 1.64)$ and $ P(Z \geq 1.65)$.\\\bigskip Which row and column?
%\begin{itemize}
%\item 1.64 = \color{blue}{1.6}+\color{orange}{0.04} \color{black}\hspace{2cm}$ P(Z \geq 1.64) =0.0505$
%\item 1.65 = \color{blue}{1.6}+\color{green}{0.05}  \color{black} \hspace{2cm}$ P(Z \geq 1.65) =0.0495$
%\end{itemize}
%\bigskip
%\small
%\begin{table}[ht]
%%\caption{Standard Normal Distribution } % title of Table
%\centering % used for centering table
%\begin{tabular}{|c|| c c c c c c|} % centered columns (4 columns)
%\hline %inserts double horizontal lines
%& & \ldots & \color{orange}{0.04} & \color{green}{0.05} &0.06&0.07\ldots \\
%%heading
%\hline \hline% inserts single horizontal line
%\ldots & \ldots &\ldots &\ldots& \ldots &\ldots&\ldots \\ %Checked
%1.5 & \ldots & 0.0630&0.0618& 0.0606 &0.0594&\dots \\ % inserting body of the table
%\color{blue}{1.6} & \ldots &0.0516& \alert{0.0505} & \alert{0.0495} &0.0485&\ldots\\
%1.7 & \ldots &0.0418 &0.0409& 0.0401 &0.0392&\dots \\ % inserting body of the table
%\ldots & \ldots &\ldots &\ldots& \ldots &\ldots&\ldots \\ %Checked
%\hline %inserts single line
%\end{tabular}
%%\label{table:nonlin} % is used to refer this table in the text
%\end{table}
%}