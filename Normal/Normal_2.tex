\documentclass[a4]{beamer}
\usepackage{amssymb}
\usepackage{graphicx}
\usepackage{subfigure}
\usepackage{newlfont}
\usepackage{amsmath,amsthm,amsfonts}
%\usepackage{beamerthemesplit}
\usepackage{pgf,pgfarrows,pgfnodes,pgfautomata,pgfheaps,pgfshade}
\usepackage{mathptmx}  % Font Family
\usepackage{helvet}   % Font Family
\usepackage{color}

\mode<presentation> {
 \usetheme{Frankfurt} % was Frankfurt
 \useinnertheme{rounded}
 \useoutertheme{infolines}
 \usefonttheme{serif}
 %\usecolortheme{wolverine}
% \usecolortheme{rose}
\usefonttheme{structurebold}
}

\setbeamercovered{dynamic}

\title[MA4413]{MA4413 Statistics for Computing \\ {\normalsize Lecture images/ : Normal Distribution}}
\author[Kevin O'Brien]{Kevin O'Brien \\ {\scriptsize kevin.obrien@ul.ie}}
\date{Autumn 2012}
\institute[Maths \& Stats]{Dept. of Mathematics \& Statistics, \\ University \textit{of} Limerick}


\renewcommand{\arraystretch}{1.5}
%------------------------------------------------------------------------%

\begin{document}


\begin{frame}
\titlepage
\end{frame}

%------------------------------------------%

\begin{frame}[fragile]
\frametitle{Today's Class}
\begin{itemize}
\item More on the Normal Distribution
\begin{itemize}
\item Last class: definitions related to statistical inference.
\item Fundamental part of statistical inference is the Central Limit Theorem.
\item CLT based on Normal Distribution.
\end{itemize}
\item Introduction to Quantile functions
\begin{itemize}
\item The \texttt{qnorm} function
\end{itemize}
\item A revision of the course material relevant to the mid-term.
\end{itemize}
\end{frame}

%------------------------------------------%

\begin{frame}[fragile]
\frametitle{Probability Density Function}
Recall: As the Normal distribution is a continuous distribution, the PDF for a particular observed value will not give us an intuitive
result (as far as this module is concerned). It is, in fact, the height of the density curve at a particular point.

Nonetheless, the relevant \texttt{R} code may be included in exam questions, so as to add complexity to questions.

\begin{verbatim}
> dnorm(0.7)
[1] 0.3122539
>
> dnorm(1.7)
[1] 0.09404908

\end{verbatim}
\end{frame}
%------------------------------------------%
\begin{frame}[fragile]
\frametitle{Sample Question}
Suppose $X$ is a normally distributed random variable with mean $\mu = 2000$ and standard deviation $\sigma=200$.
Compute the probability of X being less than (or equal to) 2340.

\[P(X \leq 2340)\]

As always, we compute the z-score that corresponds to 2340.
\[ z_o = \frac{x_o - \mu}{\sigma}  = \frac{2340-2000}{200} = 1.7\]
\end{frame}
%------------------------------------------%

\begin{frame}[fragile]
\frametitle{\texttt{R} Implementation}


Using the following \texttt{R} code, we can determine $P(Z \leq 1.7)$.
\begin{verbatim}

> pnorm(1.7)
[1] 0.9554345

\end{verbatim}
\end{frame}

%------------------------------------------%

\begin{frame}[fragile]
\frametitle{Direct \texttt{R} Implementation}

This can easily be implemented directly - without using the standardization formula, by specifying the normal mean and normal standard deviation directly. However, we will not be using this approach in this module.
\begin{verbatim}

> pnorm(2340,mean=2000,sd=200)
[1] 0.9554345


\end{verbatim}
\end{frame}
\end{document}
