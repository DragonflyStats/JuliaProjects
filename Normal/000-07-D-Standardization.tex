\documentclass[]{report}

\voffset=-1.5cm
\oddsidemargin=0.0cm
\textwidth = 480pt

\usepackage{framed}
\usepackage{subfiles}
\usepackage{graphics}
\usepackage{newlfont}
\usepackage{eurosym}
\usepackage{amsmath,amsthm,amsfonts}
\usepackage{amsmath}
\usepackage{color}
\usepackage{amssymb}
\usepackage{multicol}
\usepackage[dvipsnames]{xcolor}
\usepackage{graphicx}
\begin{document}

\section{The Standard Normal Distribution}

\begin{itemize}

\item Any value X from a normally distributed population can be converted into the equivalent standard
normal value Z (i.e. a `Z value') by the formula
\[ Z = \frac{X - \mu}{\sigma}\]
%(We shall demonstrate how shortly.)

\item The standard normal distribution has been tabulated (usually in the form of value of the cumulative distribution function F), and the other normal distributions are the simple transformations, as described above, of the standard one. \item Therefore, one can use tabulated values of the cdf of the standard normal distribution to 
find values of the cdf of a general normal distribution.


\item For some particular value $x_o$ of the normal distribution $X$, there is a corresponding \textbf{\textit{z-score}} $z_o$. 
\item The z-score is the distance, in terms of standard deviations, that $x_o$ is from the mean $\mu$.
\end{itemize}







\subsection{The Standard Normal Distribution}

\begin{itemize}
\item The standard normal distribution ( commonly called the Z distribution ) is a special case of the \textbf{\emph{normal distribution}}.
\item It is characterized by the following

\begin{itemize}

\item The mean $\mu$ is always equal to $0$.
\item The standard deviation $\sigma$ is always equal to $1$.
\item The variance $\sigma^2$ is therefore equal to $1$ also .
\end{itemize}
%\item A value from the standard normal distribution can be written as $Z \sim N(0,1^2)$
%\item It is very useful for solving normal distribution problems
\end{itemize}

%------------------------------------------------%

\section{Normal Distribution}
The standard normal distribution is a normal distribution with a mean of 0 and a standard deviation of 1. 
Normal distributions can be transformed to standard normal distributions by the formula:
\[ Z = {X - \mu \over \sigma} \]
where X is a score from the original normal distribution, $\mu$ is the mean of the original normal distribution, and $\sigma$ is the standard deviation 
of original normal distribution. The standard normal distribution is sometimes called the Z distribution. 



\section{The Standardized Value}

\begin{itemize}
\item Suppose that mean $\mu = 80 $ and that standard deviation $\sigma = 8$.

\item What is the Z value for $X = 100$?

\[
Z_{100} = {X_0 - \mu \over \sigma}  = {100 - 80 \over 8} = {20 \over 8} = 2.5
\]

\item Therefore $Z_{100} = 2.5$
\end{itemize}

% The standardization formula
% used to find Z values


%=============================================================%

\begin{framed}
\begin{itemize}

\item We can find a probability associated with a value, that is from a normally distribution,  by computing the $Z$ value. 

\[z_0 = {x_0 - \mu \over \sigma}\]
\begin{itemize}
\item[$\ast$] $X_o$  - Some random value from the population of X values.
\item[$\ast$] $\mu$ - The mean of the population of X values.
\item[$\ast$] $\sigma$ - The variance of the population of X values.
\item[$\ast$] $Z_o$ - The Z value that corresponds to $X_o$
\end{itemize}
\end{itemize}
\end{framed}

%=============================================================%




All normally distributed random variables have corresponding $Z$ values, called Z-scores.\\
\bigskip
For normally distributed random variables, the z-score can be found using the \textbf{\emph{standardization formula}};
\[z_o = { x_o - \mu \over \sigma}\]
where $x_o$ is a score from the original normal (``X") distribution, $\mu$ is the mean of the original normal distribution, and $\sigma$ is the standard deviation of original normal distribution.\\
\bigskip
Therefore $z_o$ is the z-score that corresponds to $x_o$.

\begin{itemize}
\item Terms with subscripts mean particular values, and are not variable names.
\item The z distribution will only be a normal distribution if the original distribution (X) is normal.
\end{itemize}



%------------------------------------------------------------------------%

{\textbf{The Standardized Value}

\begin{itemize}
\item The first step in solving the problem is to compute the standardized value, also known as the `Z' value.

\item We must know the value of the mean $\mu$ and the standard deviation $\sigma$.

\item To find the `Z' value $Z_0$ for a particular quantity $X_0$.
\end{itemize}



\vspace{0.1cm}
\[
Z_{0} = {X_0 - \mu \over \sigma}
\]

\subsection{Z-scores}
\begin{itemize}
\item A Z-score always reflects the number of standard deviations above or below the mean a particular score is.
\item 
Suppose the scores of a test are normally distributed with a mean of 50 and a standard deviation of 9
\item For instance, if a person scored a 68 on a test, then they scored 2 standard deviations above the mean.

\item Converting the test scores to z scores, an X value of 68 would yield:
\[ Z = {68 - 50 \over 9} =2 \]

\item So, a Z score of 2 means the original score was 2 standard deviations above the mean.


\item Note that the z distribution will only be a normal distribution if the original distribution (X) is normal. 
\end{itemize}
}







\subsection{Solving using the Z distribution}
When we have a normal distribution with any mean $\mu$  and any standard deviation $\sigma$ , we answer probability questions about the distribution by first converting all values to corresponding values of the standard normal ("z") distribution.





$Z$ is the standard normal random variable with a mean of zero and a standard deviation of 1.



It can be thought of as a measure of how many standard deviations that a value "x" is from mean $\mu$ .





\subsection{Computing the Z-score}
The normal distribution has the following paramters

\begin{itemize}
\item $\mu$ the mean of the normal distribution
\item $\sigma$ the standard deviation of the distribution
\end{itemize}


\[z = \frac{x - \mu}{ \sigma}\]

Suppose $ \mu$ =1000
$\sigma=400$

\[ X \sim N(1000,400) \]

\end{document}


