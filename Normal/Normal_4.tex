

%------------------------------------------------------------------------%
\frame{
\frametitle{The Standard Normal Distribution}

\begin{itemize}

\item The standard normal distribution is a special case of the normal distribution with a mean $\mu= 0$ and a standard deviation $\sigma =1$.
\item We denote the standard normal random variable as $Z$ rather than $X$.
\item The distribution is (i.e. Murdoch Barnes Table 3)
\item Rather than computing probabilities from first principles, which is very difficult, probabilities from distributions other than the Z distribution (e.g. X $\sim$($\mu=100, \sigma =15$)) can be computed using the Z distribution, a much easier approach. (We shall demonstrate how shortly.)
\end{itemize}
}
%------------------------------------------------%
\frame{
\frametitle{Standardization formula}
All normally distributed random variables have corresponding $Z$ values, called Z-scores.\\
\bigskip
For normally distributed random variables, the z-score can be found using the \textbf{\emph{standardization formula}};
\[z_o = { x_o - \mu \over \sigma}\]
where $x_o$ is a score from the original normal (``X") distribution, $\mu$ is the mean of the original normal distribution, and $\sigma$ is the standard deviation of original normal distribution.\\
\bigskip
Therefore $z_o$ is the z-score that corresponds to $x_o$.

\begin{itemize}
\item Terms with subscripts mean particular values, and are not variable names.
\item The z distribution will only be a normal distribution if the original distribution (X) is normal.
\end{itemize}
}

%------------------------------------------------------------------------%
\frame{
\frametitle{The Standardized Value}
\Large
\begin{itemize}
\item Suppose that mean $\mu = 80 $ and that standard deviation $\sigma = 8$.
\item What is the Z-score for $x_o = 100$?
\[
z_{100} = {x_0 - \mu \over \sigma} = {100 - 80 \over 8} = {20 \over 8} = 2.5
\]
\item Therefore $z_{100} = 2.5$
\end{itemize}
}
%------------------------------------------------------------------------%
\frame{
\frametitle{Z scores}
A Z-score always reflects the number of standard deviations above or below the mean a particular score is.
Suppose the scores of a test are normally distributed with a mean of 50 and a standard deviation of 9
For instance, if a person scored a 68 on a test, then they scored 2 standard deviations above the mean.

Converting the test scores to z scores, an X value of 68 would yield:
\[ Z = {68 - 50 \over 9} =2 \]

So, a Z score of 2 means the original score was 2 standard deviations above the mean.
}
%------------------------------------------------------------------------%
%\end{document}
% The standardization formula
% used to find Z values

%

%------------------------------------------------------------------------%
\frame{
\frametitle{The Standard Normal (Z) Distribution Tables}
\begin{itemize}
\item Importantly, probabilities relating to the z distribution are comprehensively tabulated in \textbf{\emph{Murdoch Barnes table 3}}.

\item Given a value of $k$ (with k usually between 0 and 4), the probability of a standard normal "Z" random variable being greater than (or equal to) k $P(Z \geq k)$ is given in Murdoch Barnes table 3 .
\item Other statistical tables can be used, but they may tabulate probabilities in a different way.
\end{itemize}
}

%------------------------------------------------------------------------%
\frame{
\frametitle{An Important Identity}
If two values $z_o$ and $x_o$ are related in the following way, for some values $\mu$ and $\sigma$,
\[
z_{0} = {x_0 - \mu \over \sigma}
\]
Then we can can say

\[ P(X \geq x_o) = P(Z \geq z_o) \]

or alternatively

\[ P(X \leq x_o) = P(Z \leq z_o) \]

This is fundamental to solving problems involving normal distributions.

}

%---------------------------------------------%
\end{document}
