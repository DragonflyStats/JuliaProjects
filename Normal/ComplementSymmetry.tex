\documentclass[a4paper,12pt]{article}
%%%%%%%%%%%%%%%%%%%%%%%%%%%%%%%%%%%%%%%%%%%%%%%%%%%%%%%%%%%%%%%%%%%%%%%%%%%%%%%%%%%%%%%%%%%%%%%%%%%%%%%%%%%%%%%%%%%%%%%%%%%%%%%%%%%%%%%%%%%%%%%%%%%%%%%%%%%%%%%%%%%%%%%%%%%%%%%%%%%%%%%%%%%%%%%%%%%%%%%%%%%%%%%%%%%%%%%%%%%%%%%%%%%%%%%%%%%%%%%%%%%%%%%%%%%%
\usepackage{eurosym}
\usepackage{vmargin}
\usepackage{amsmath}
\usepackage{graphics}
\usepackage{epsfig}
\usepackage{subfigure}
\usepackage{framed}
\usepackage{enumerate}
\usepackage{fancyhdr}

\setcounter{MaxMatrixCols}{10}
%TCIDATA{OutputFilter=LATEX.DLL}
%TCIDATA{Version=5.00.0.2570}
%TCIDATA{<META NAME="SaveForMode"CONTENT="1">}
%TCIDATA{LastRevised=Wednesday, February 23, 201113:24:34}
%TCIDATA{<META NAME="GraphicsSave" CONTENT="32">}
%TCIDATA{Language=American English}

\pagestyle{fancy}
\setmarginsrb{20mm}{0mm}{20mm}{25mm}{12mm}{11mm}{0mm}{11mm}
\lhead{MS4222} \rhead{Kevin O'Brien} \chead{Normal Distribution} %\input{tcilatex}
%%%%%%%%%%%%%%%%%%%%%%%%%%%%%%%%%%%%%%%%%%%%%
\begin{document}
\section*{Complement and Symmetry Rules}

Any normal distribution problem can be solved with some combination of the following rules.
\begin{enumerate}
\item The Complement Rule
\item The Symmetry Rule
\item The Interval Rule
\end{enumerate}
We will look at the first two rules in this section, and revert to the Interval Rule later on.
	
	\section*{The Complement Rule}
	
	\begin{itemize}
		\item The \textbf{\textit{Complement Rule}} is a very simple rule for working with probability distributions.
		\item In this presentation, we will look at the Complement Rule for continuous probability distributions only.
		
		
		\item To compute the probability of the complementary event, simple subtract the probability of the event from 1.
		
		\[P(X \leq k) = 1- P(X \geq k) \]
		\[P(X \geq k) = 1- P(X \leq k) \]
		 \item Common to all continuous random variables


\[P(Z \leq 1.28) = 1 - P(Z \geq 1.28)  = 1-0.1003 = 0.8997\]
	\end{itemize}	
	

	
	
	\begin{framed}
	\noindent \textbf{Important}
	\begin{itemize}
		
	
		\item The probability of being greater than or equal to a value is simply the sum of the probability of being greater than the value and the probability of being equal to the value
		\[  P(X \geq k)  = P( X > k) + P(X = k) \]
			\item Remember, for continuous probability distributions, the probability of an \textbf{exact} value is extremely small, such that it is almost zero.
				\[P(X = k) \approx 0\]
		\item Therefore we neglect the equality components in expressions such as
		$P(X \leq k)$ and $P(X \geq k)$.
		\item In fact we can treat this two expressions as \textbf{\textit{complementary events}}.
	\end{itemize}
	\end{framed}
	
	\newpage
\section*{Symmetry rule}
\begin{itemize}
\item
This rule is based on the property of symmetry for the normal distribution. 
\item (\textbf{Important}) It applies only to the Standard Normal Distribution.
\item
Only the probabilities corresponding to values between 0 and 4 are tabulated in Murdoch Barnes. We use the symmetry rule for cases where the Z-score is negative.
\item
If we have a negative value of k, we can use the symmetry rule.
\[P(Z \leq -k) = P(Z \geq k) \]
\item By extension, we can say
\[P(Z \geq -k) = P(Z \leq k) \]
\end{itemize}

\subsection*{Example}
Find $P(Z \geq -1.28)$
\subsection*{Solution}\
\begin{itemize}
\item Using the symmetry rule
\[P(Z \geq -1.28) = P(Z \leq 1.28) \]
\item Using the complement rule
\[P(Z \geq -1.28) = 1 - P(Z \geq 1.28) \]
\[P(Z \geq -1.28) = 1 - 0.1003 = 0.8997 \]
\end{itemize}



%---------------------------------------------%

\end{document}
