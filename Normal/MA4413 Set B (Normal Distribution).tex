\documentclass[a4]{beamer}
\usepackage{amssymb}
\usepackage{graphicx}
\usepackage{subfigure}
\usepackage{newlfont}
\usepackage{amsmath,amsthm,amsfonts}
%\usepackage{beamerthemesplit}
\usepackage{pgf,pgfarrows,pgfnodes,pgfautomata,pgfheaps,pgfshade}
\usepackage{mathptmx}  % Font Family
\usepackage{helvet}   % Font Family
\usepackage{color}

\mode<presentation> {
 \usetheme{Default} % was Frankfurt
 \useinnertheme{rounded}
 \useoutertheme{infolines}
 \usefonttheme{serif}
 %\usecolortheme{wolverine}
% \usecolortheme{rose}
\usefonttheme{structurebold}
}

\setbeamercovered{dynamic}

\title[MathsCast]{MathsCast Presentations \\ {\normalsize The Continuous Uniform Distribution}}
\author[Kevin O'Brien]{Kevin O'Brien \\ {\scriptsize kevin.obrien@ul.ie}}
\date{Summer 2011}
\institute[Maths \& Stats]{Dept. of Mathematics \& Statistics, \\ University \textit{of} Limerick}

\renewcommand{\arraystretch}{1.5}


%------------------------------------------------------------------------%
\begin{document}


\begin{frame}

\begin{itemize}
\item Continuous Random Variables
\item The Normal Distribution
\item Characteristics of the Normal Distribution
\item The Standard Normal (Z) Distribution
\item Using Murdoch Barnes Table 3
\item Standardization Formula
\item Important Formulae
\end{itemize}

\end{frame}


%------------------------------------------------------------%


%------------------------------------------------------------------------%
\frame{
\frametitle{Using the Murdoch Barnes Tables}
\Large

\begin{itemize}
\item Knowing the Z value is very useful, because it is easy to compute $P(Z \geq Z_0)$ for some value $Z_0$ using statistical tables.
\bigskip
\item $P(X \geq X_0) = P(Z \geq Z_0)$\bigskip
\item From our previous example we can say the following
\[ P(X \geq 100) = P(Z \geq 2.5)\]
\end{itemize}

}


\frame{
\frametitle{Normal Distribution} 
\begin{itemize}
\item Normal distributions are a family of distributions that have the same general shape. 
\item They are symmetric with scores more concentrated in the middle than in the tails. Normal distributions are sometimes described as bell shaped. 
\item Examples of normal distributions are shown below. Notice that they differ in how spread out they are. The area under each curve is the same. 
\item The height of a normal distribution can be specified mathematically in terms of two parameters: the mean ($\mu$) and the standard deviation ($\sigma$). 

\end{itemize}
}
%------------------------------------------------%
\frame{
\frametitle{Normal Distribution}
The standard normal distribution is a normal distribution with a mean of 0 and a standard deviation of 1. 
Normal distributions can be transformed to standard normal distributions by the formula:
\[ Z = {X - \mu \over \sigma} \]
where X is a score from the original normal distribution, $\mu$ is the mean of the original normal distribution, and $\sigma$ is the standard deviation 
of original normal distribution. The standard normal distribution is sometimes called the Z distribution. 
A z score always reflects the number of standard deviations above or below the mean a particular score is. 
For instance, if a person scored a 68 on a test with a mean of 50 and a standard deviation of 9, then they scored 2 standard deviations above the mean. 
Converting the test scores to z scores, an X of 70 would be:
\[ Z = {68 - 50 \over 9} \]
So, a Z score of 2 means the original score was 2 standard deviations above the mean. Note that the z distribution will only be a normal distribution if the original distribution (X) is normal. 
 
}

%------------------------------------------------------------------------%

\frame{

\frametitle{The Normal Distribution}

The probability density function of the normal distribution is given as

\[   f(x) = \frac{1}{\sqrt{2\pi\sigma^2}} e^{ -\frac{(x-\mu)^2}{2\sigma^2} } \]

We will not use this formula.
}
%------------------------------------------------------------------%
\frame{
\frametitle{ Characteristics of the Normal probability distribution}
\begin{itemize}
\item[1] The highest point on the normal curve is at the mean, which is also the median and mode of the distribution.
\item[2] \alert{[VERY IMPORTANT]}
The normal probability curve is bell-shaped and symmetric, with the shape of the curve to the left of the mean a mirror image of the shape of the curve to the right of the mean.
\item[3] The standard deviation determines the width of the curve. Larger values of the the standard deviation result in wider flatter curves, showing more dispersion in data.
\item[4] The total area under the curve for the normal probability distribution is 1.
\end{itemize}
}
%------------------------------------------------------------------%
\frame{
\frametitle{ Characteristics of the Normal probability distribution}
\begin{itemize}
\item The interval defined by \textbf{the mean} $ \pm 1 \times $ standard deviation includes $68\%$ of the observations ,leaving $16\%$ (approx) in each tail.
\item The interval defined by \textbf{the mean} $ \pm 1.96 \times $ standard deviation includes $95\%$ of the observations ,leaving $2.5\%$ (approx) in each tail.
\item The interval defined by \textbf{the mean} $ \pm 2.58 \times $ standard deviation includes $99\%$ of the observations ,leaving $0.5\%$ (approx) in each tail.
\end{itemize}
}

%------------------------------------------------------------------------%
\frame{
\frametitle{The Standardized Value}
\Large
\begin{itemize}
\item Suppose that mean $\mu = 80 $ and that standard deviation $\sigma = 8$.

\item What is the Z value for $X = 100$?

\[
Z_{100} = {X_0 - \mu \over \sigma}  = {100 - 80 \over 8} = {20 \over 8} = 2.5
\]

\item Therefore $Z_{100} = 2.5$
\end{itemize}
}
%------------------------------------------------------------------------%
%\end{document}
% The standardization formula
% used to find Z values

\frame{

\frametitle{The Standardization Formula}
\Large
\[
Z_o = { X_o  - \mu \over \sigma}
\]

 All normally distributed random variables have correspondinng $Z$ values
}
% Note: Terms with subscripts mean particular values, and are not variable names.

%------------------------------------------------------------%

\frame{

\Large


\begin{itemize}



\item We can find a probability associated with a value, that is from a normally distribution,  by computing the $Z$ value. 

\[z_0 = {x_0 - \mu \over \sigma}\]
\begin{itemize}
\item $X_o$  - Some random value from the population of X values.
\item $\mu$ - The mean of the population of X values.
\item $\sigma$ - The variance of the population of X values.
\item $Z_o$ - The Z value that corresponds to $X_o$
\end{itemize}

\end{itemize}
}
%------------------------------------------------------------------------%
\frame{
\frametitle{The Standard Normal Distribution}
\Large
\begin{itemize}
\item The standard normal distribution ( commonly called the Z distribution ) is a special case of the \textbf{\emph{normal distribution}}.
\item It is characterized by the following

\begin{itemize}
\Large
\item The mean $\mu$ is always equal to $0$.
\item The standard deviation $\sigma$ is always equal to $1$.
\item The variance $\sigma^2$ is therefore equal to $1$ also .
\end{itemize}
%\item A value from the standard normal distribution can be written as $Z \sim N(0,1^2)$
%\item It is very useful for solving normal distribution problems
\end{itemize}
% \[Z \sim N(0,1^2)\]


}

\frame{
\frametitle{The Standard Normal (Z) Distribution}
\begin{itemize}
\item A random variable that has a normal distribution with a mean of zero and a standard deviation of one is said to have a standard normal probability distribution. It is often nick-named the "z" distribution.
\item Importantly, probabilities relating to the z distribution are comprehensively tabulated in Murdoch Barnes table 3.
\item Given a value of $k$ (with k usually between 0 and 4), the probability of a standard normal "z" random variable being greater than (or equal to) k is given in Murdoch Barnes table 3 (page 71).
\end{itemize}
}

%------------------------------------------------------------------%
\frame{
\frametitle{Complement and Symmetry Rules}
Any normal distribution problem can be solved with some combination of the following rules.
\begin{itemize} \item The Complement rule (Common to all continuous random variables)
\[P(Z \geq k) = 1 - P(Z \leq k) \]
Similarly
\[P(X \geq k) = 1 - P(X \leq k) \]
\end{itemize}
}
%------------------------------------------------------------------%
\frame{
\frametitle{Complement and Symmetry Rules}
\begin{itemize}
\item
This rule is based on the property of symmetry mentioned previously.
\item
Only the probabilities corresponding to values between 0 and 4 are tabulated in Murdoch Barnes.
\item
If we have a negative value of k, we can use the symmetry rule.
\end{itemize}
\[P(Z \leq -k) = P(Z \geq k) \]
by extension, we can say
\[P(Z \geq -k) = P(Z \leq k) \]
}
%------------------------------------------------------------------%
\frame{
\frametitle{Example}
Find $P(Z \geq -1.28)$
\textbf{Solution}
\begin{itemize}
\item Using the symmetry rule
\[P(Z \geq -1.28) = P(Z \leq 1.28) \]
\item Using the complement rule
\[P(Z \geq -1.28) = 1 - P(Z \geq 1.28) \]
\[P(Z \geq -1.28) = 1 - 0.1003 = 0.8997 \]
\end{itemize}
}
%------------------------------------------------------------------%
\frame{
Find the probability of a ``z" random variable being between -1.8 and 1.96?
i.e. Compute $P(-1.8 \leq Z \leq 1.96)$
Solution
\begin{itemize}
\item Consider the complement event of being in this interval: a combination of being too low or two high.
\item
The probability of being too low for this interval is $P(Z \leq -1.80) = 0.0359$ (from before)
\item
The probability of being too high for this interval is $P(Z \geq 1.96) = 0.0250$ (from before)
\item
Therefore the probability of being \textbf{outside} the interval is 0.0359 + 0.0250 = 0.0609.
\item
Therefore the probability of being \textbf{inside} the interval is 1- 0.0609 = 0.9391
$P(-1.8 \leq Z \leq 1.96) = 0.9391$
\end{itemize}
}
%------------------------------------------------------------------%
\frame{
\frametitle{Solving using the Z distribution}
When we have a normal distribution with any mean $\mu$ and any standard deviation $\sigma$ , we answer probability questions about the distribution by first converting all values to corresponding values of the standard normal ("z") distribution.
The formula used to convert any random variable "X" ( with mean $\mu$ and standard deviation $\sigma$ specified) to the standard normal ("z") distribution is given as follows.
\[ Z_o = {X_o - \mu \over \sigma} \]
$Z$ is the standard normal random variable with a mean of zero and a standard deviation of 1.
It can be thought of as a measure of how many standard deviations that a value "x" is from mean $\mu$ .
}




%------------------------------------------------------------------------%
\frame{
\frametitle{The Standard Normal Distribution}
\Large
\begin{itemize}
\item Special case of the normal distributions
\item The distribution is well described in statistical tables
\item rahter than computing probabilities from first principles, X values 
\end{itemize}


}





%------------------------------------------------------------------------%

\frame{\frametitle{The Standardized Value}
\Large
\begin{itemize}
\item The first step in solving the problem is to compute the standardized value, also known as the `Z' value.

\item We must know the value of the mean $\mu$ and the standard deviation $\sigma$.

\item To find the `Z' value $Z_0$ for a particular quantity $X_0$.
\end{itemize}



\vspace{0.1cm}
\[
Z_{0} = {X_0 - \mu \over \sigma}
\]
}

%------------------------------------------------------------------------%
\frame{
\begin{table}[ht]
\frametitle{Find $ P(Z \geq 0.60)$}
\vspace{-1.5cm}
%\caption{Standard Normal Distribution } % title of Table
\centering % used for centering table
\begin{tabular}{|c|| c c c c c c|} % centered columns (4 columns)
\hline %inserts double horizontal lines
 & 0.00  & 0.01 & 0.02 &0.03&\ldots&\ldots \\
%heading
\hline  \hline% inserts single horizontal line \hline
\ldots  & \ldots  &\ldots &\ldots& \ldots &\ldots&\ldots \\ % inserting body of the table
0.4  & 0.3446  & 0.3409&0.3372 & 0.3336 &\ldots&\ldots\\
0.5  & 0.3085 & 0.3050 &0.3015& 0.2981 &\ldots&\dots \\ % inserting body of the table
0.6  & 0.2743 & 0.2709&0.2676 & 0.2643 &\ldots&\ldots\\
0.7  & 0.2420 & 0.2389 &0.2358& 0.2327 &\ldots&\dots \\ % inserting body of the table
\ldots  & \ldots  &\ldots &\ldots& \ldots &\ldots&\ldots \\ % inserting body of the table
\hline %inserts single line
\end{tabular}
%\label{table:nonlin} % is used to refer this table in the text
\end{table}
}

%------------------------------------------------------------------------%
\frame{
\begin{table}[ht]
\frametitle{Find $ P(Z \geq 1.28)$}
\vspace{-1.5cm}
%\caption{Standard Normal Distribution } % title of Table
\centering % used for centering table
\begin{tabular}{|c|| c c c c c c|} % centered columns (4 columns)
\hline %inserts double horizontal lines
 & \ldots  & \ldots & 0.006 &0.07&0.08&0.09 \\
%heading
\hline  \hline% inserts single horizontal line
\ldots  & \ldots  & \ldots &\ldots& \ldots &\ldots&\dots \\ % inserting body of the table
1.0  & \ldots  & \ldots &0.1446& 0.1423 &0.1401&0.1379 \\ % inserting body of the table
1.1  & \ldots  & \ldots&0.1230& 0.1210 &0.1190&0.1170 \\ % inserting body of the table
1.2  & \ldots  & \ldots&0.1038 & 0.1020 &0.1003&0.0985\\
1.3  & \ldots  & \ldots &0.0869& 0.0853 &0.0838&0.0823 \\ % inserting body of the table
\ldots  & \ldots  &\ldots&\ldots & \ldots &\ldots&\ldots\\
\hline %inserts single line
\end{tabular}
%\label{table:nonlin} % is used to refer this table in the text
\end{table}
}
%------------------------------------------------------------------------%
\frame{
\begin{table}[ht]
\frametitle{Find $ P(Z \geq 1.65)$ and $ P(Z \geq 1.65)$}
\vspace{-1.5cm}
%\caption{Standard Normal Distribution } % title of Table
\centering % used for centering table
\begin{tabular}{|c|| c c c c c c|} % centered columns (4 columns)
\hline %inserts double horizontal lines
 & \ldots  & 0.04 & 0.05 &0.06&0.07&\ldots \\
%heading
\hline  \hline% inserts single horizontal line
\ldots  & \ldots  &\ldots &\ldots& \ldots &\ldots&\ldots \\ %Checked
1.5  & \ldots  & 0.0630&0.0618& 0.0606 &0.0594&\dots \\ % inserting body of the table
1.6  & \ldots  &0.0516&0.0505 & 0.0495 &0.0485&\ldots\\
1.7  & \ldots  &0.0418 &0.0409& 0.0401 &0.0392&\dots \\ % inserting body of the table
\ldots  & \ldots  &\ldots &\ldots& \ldots &\ldots&\ldots \\ %Checked
\hline %inserts single line
\end{tabular}
%\label{table:nonlin} % is used to refer this table in the text
\end{table}
}
%------------------------------------------------------------------------%
\frame{
\begin{table}[ht]
\frametitle{Estimate $ P(Z \geq 1.645)$}
\vspace{-1.5cm}
%\caption{Standard Normal Distribution } % title of Table
\centering % used for centering table
\begin{tabular}{|c|| c c c c c c|} % centered columns (4 columns)
\hline %inserts double horizontal lines
 & \ldots  & 0.04 & 0.05 &0.06&0.07&\ldots \\
%heading
\hline  \hline% inserts single horizontal line
\ldots  & \ldots  &\ldots &\ldots& \ldots &\ldots&\ldots \\ %Checked
1.5  & \ldots  & 0.0630&0.0618& 0.0606 &0.0594&\dots \\ % inserting body of the table
1.6  & \ldots  &0.0516&0.0505 & 0.0495 &0.0485&\ldots\\
1.7  & \ldots  &0.0418 &0.0409& 0.0401 &0.0392&\dots \\ % inserting body of the table
\ldots  & \ldots  &\ldots &\ldots& \ldots &\ldots&\ldots \\ %Checked
\hline %inserts single line
\end{tabular}
%\label{table:nonlin} % is used to refer this table in the text
\end{table}
}




%--------------------------------------------------------------










\end{document}
%----------------------------------------------------------------------------------------------------%