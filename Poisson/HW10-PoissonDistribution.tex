
\begin{document}
	
	%---------------------------------------------------------------------------%
	{
		\textbf{Poisson Distribution}
		A Poisson random variable is the number of successes that result from a Poisson experiment.
		
		The probability distribution of a Poisson random variable is called a Poisson distribution.
		
		Given the mean number of successes ($m$) that occur in a specified region, we can compute the Poisson probability based on the following formula:
	}
	
	%---------------------------------------------------------------------------%
	{
		\textbf{The Poisson Probability Distribution}
		\begin{itemize}
			\item The number of occurrences in a unit period (or space)
			\item The expected number of occurrences is $m$
		\end{itemize}
	}
	
	%---------------------------------------------------------------------------%
	{
		\textbf{Poisson Formulae}
		The probability that there will be $k$ occurrences in a unit time period is denoted $P(X=k)$, and is computed as follows.
		\Large
		\[ P(X = k)=\frac{m^k e^{-m}}{k!} \]
		
	}
	%---------------------------------------------------------------------------%
	{
		\textbf{Poisson Formulae}
		Given that there is on average 2 occurrences per hour, what is the probability of no occurences in the next hour? \\ i.e. Compute $P(X=0)$ given that $m=2$
		\Large
		\[ P(X = 0)=\frac{2^0 e^{-2}}{0!} \]
		\normalsize
		\begin{itemize}
			\item $2^0$ = 1
			\item $0!$ = 1
		\end{itemize}
		The equation reduces to
		\[ P(X = 0)=e^{-2} = 0.1353\]
	}
	%---------------------------------------------------------------------------%
	{
		\textbf{Poisson Formulae}
		What is the probability of one occurrences in the next hour? \\ i.e. Compute $P(X=1)$ given that $m=2$
		\Large
		\[ P(X = 1)=\frac{2^1 e^{-2}}{1!} \]
		\normalsize
		\begin{itemize}
			\item $2^1$ = 2
			\item $1!$ = 1
		\end{itemize}
		The equation reduces to
		\[ P(X = 1) = 2 \times e^{-2} = 0.2706\]
	}
	%---------------------------------------------------------------------------%
