\documentclass[12pt, a4paper]{report}
\usepackage{epsfig}
\usepackage{subfigure}
%\usepackage{amscd}
\usepackage{amssymb}
\usepackage{graphicx}
%\usepackage{amscd}

\usepackage{subfiles}
\usepackage{framed}
\usepackage{subfiles}
\usepackage{amsthm, amsmath}
\usepackage{amsbsy}
\usepackage{framed}
\usepackage[usenames]{color}
\usepackage{listings}
\lstset{% general command to set parameter(s)
	basicstyle=\small, % print whole listing small
	keywordstyle=\color{red}\itshape,
	% underlined bold black keywords
	commentstyle=\color{blue}, % white comments
	stringstyle=\ttfamily, % typewriter type for strings
	showstringspaces=false,
	numbers=left, numberstyle=\tiny, stepnumber=1, numbersep=5pt, %
	frame=shadowbox,
	rulesepcolor=\color{black},
	columns=fullflexible
} %
%\usepackage[dvips]{graphicx}
\usepackage{natbib}
\bibliographystyle{chicago}
\usepackage{vmargin}
% left top textwidth textheight headheight
% headsep footheight footskip
\setmargins{3.0cm}{2.5cm}{15.5 cm}{22cm}{0.5cm}{0cm}{1cm}{1cm}
\renewcommand{\baselinestretch}{1.5}
\pagenumbering{arabic}
%\theoremstyle{plain}
\newtheorem{theorem}{Theorem}[section]
\newtheorem{corollary}[theorem]{Corollary}
\newtheorem{ill}[theorem]{Example}
\newtheorem{lemma}[theorem]{Lemma}
\newtheorem{proposition}[theorem]{Proposition}
\newtheorem{conjecture}[theorem]{Conjecture}
\newtheorem{axiom}{Axiom}
\theoremstyle{definition}
\newtheorem{definition}{Definition}[section]
\newtheorem{notation}{Notation}
\theoremstyle{remark}
\newtheorem{remark}{Remark}[section]
\newtheorem{example}{Example}[section]
\renewcommand{\thenotation}{}
%\renewcommand{\thetable}{\thesection.\arabic{table}}
%\renewcommand{\thefigure}{\thesection.\arabic{figure}}

\author{ } \date{ }

\begin{document}
	
	%---------------------------------------------------------------------------%
	{
		\textbf{Poisson Distribution}
		A Poisson random variable is the number of successes that result from a Poisson experiment.
		
		The probability distribution of a Poisson random variable is called a Poisson distribution.
		
		Given the mean number of successes ($m$) that occur in a specified region, we can compute the Poisson probability based on the following formula:
	}
	
	%---------------------------------------------------------------------------%
	{
		\textbf{The Poisson Probability Distribution}
		\begin{itemize}
			\item The number of occurrences in a unit period (or space)
			\item The expected number of occurrences is $m$
		\end{itemize}
	}
	
	%---------------------------------------------------------------------------%
	{
		\textbf{Poisson Formulae}
		The probability that there will be $k$ occurrences in a unit time period is denoted $P(X=k)$, and is computed as follows.
		\Large
		\[ P(X = k)=\frac{m^k e^{-m}}{k!} \]
		
	}
	%---------------------------------------------------------------------------%
	{
		\textbf{Poisson Formulae}
		Given that there is on average 2 occurrences per hour, what is the probability of no occurences in the next hour? \\ i.e. Compute $P(X=0)$ given that $m=2$
		\Large
		\[ P(X = 0)=\frac{2^0 e^{-2}}{0!} \]
		\normalsize
		\begin{itemize}
			\item $2^0$ = 1
			\item $0!$ = 1
		\end{itemize}
		The equation reduces to
		\[ P(X = 0)=e^{-2} = 0.1353\]
	}
	%---------------------------------------------------------------------------%
	{
		\textbf{Poisson Formulae}
		What is the probability of one occurrences in the next hour? \\ i.e. Compute $P(X=1)$ given that $m=2$
		\Large
		\[ P(X = 1)=\frac{2^1 e^{-2}}{1!} \]
		\normalsize
		\begin{itemize}
			\item $2^1$ = 2
			\item $1!$ = 1
		\end{itemize}
		The equation reduces to
		\[ P(X = 1) = 2 \times e^{-2} = 0.2706\]
	}
	%---------------------------------------------------------------------------%
\section*{Probability of events for a Poisson distribution}
An event can occur 0, 1, 2, … times in an interval. The average number of events in an interval is designated ${\displaystyle \lambda }$   (lambda). Lambda is the event rate, also called the rate parameter. The probability of observing k events in an interval is given by the equation

\[{\displaystyle P(k{\text{ events in interval}})=e^{-\lambda }{\frac {\lambda ^{k}}{k!}}} \]
where

\begin{itemize}
	\item ${\displaystyle \lambda }$  is the average number of events per interval
	\item	e is the number 2.71828... (Euler's number) the base of the natural logarithms
	\item	k takes values 0, 1, 2, …
	\item	$k! = k \times (k − 1) \times (k − 2) \times \ldots \times 2 \times 1$ is the factorial of k.
\end{itemize}

This equation is the probability mass function (PMF) for a Poisson distribution.
%------------------------------------------------------------------%
\frame{
	\frametitle{Poisson Expected Value and Variance}
	
	
	If the random variable X has a Poisson distribution with parameter $m$, we write
	\[ X \sim Poisson(m) \]
	
	
	\begin{itemize}
		\item Expected Value of X : E(X) = m
		\item Variance of X : $\mbox{Var}(X) = m$
		\item Standard Deviation of X : $SD(X) = \sqrt{m}$
	\end{itemize}
}
%------------------------------------------------------------------%
\frame{
	\frametitle{Poisson Distribution : Example} 
	
	\begin{itemize}
		\item The number of faults in a fibre optic cable were recorded for each kilometre length of cable.
		\item The mean number of faults was found to be 4 faults per kilometre.
		\item The standard deviation of the number of faults was found to be 2 faults per kilometre.
		\item Is the Poisson Distribution is a useful technique for modelling the number of faults in fibre optic cable?
		\item (Looking at the last slide, the answer is yes, because the variance and mean are equal). 
	\end{itemize}
	
}
%---------------------------------------------------------------------%
\begin{frame}
	\frametitle{Poisson Approximation of the Binomial}
	\begin{itemize}
		\item The Poisson distribution can sometimes be used to approximate the
		binomial distribution
		\item When the number of observations n is large, and the success probability
		p is small, the $B(n,p)$ distribution approaches the Poisson distribution
		with the parameter given by $m = np$.
		\item This is useful since the computations involved in calculating binomial
		probabilities are greatly reduced.
		\item As a rule of thumb, n should be greater than 50 with p very small, such
		that np should be less than 5.
		\item If the value of p is very high, the definition of what constitutes a
		``success" or ``failure" can be switched.
	\end{itemize}
\end{frame}

%---------------------------------------------------------------------%
\begin{frame}
	\frametitle{Poisson Approximation: Example}
	
	\begin{itemize}
		\item Suppose we sample 1000 items from a production line that is producing, on
		average, $0.1\%$ defective components.
		\item Using the binomial distribution, the probability of exactly 3 defective items in
		our sample is
		\[P(X = 3) = ^{1000}C_{3} \times 0.001^{3} \times 0.999^{997}\]
	\end{itemize}

	\frametitle{Poisson Approximation: Example}
	Lets compute each of the component terms individually.
	
	\begin{itemize}
		\item $^{1000}C_{3}$
		\[^{1000}C_{3} = \frac{1000 \times 999 \times 998}{3 \times 2 \times 1} = 166,167,000\]
		\item $0.001^3$
		\[0.001^3 = 0.000000001\]
		\item $0.999^{997}$
		\[0.999^{997} = 0.36880\]
	\end{itemize}
	
	
	Multiply these three values to compute the binomial probability
	$P(X = 3) = 0.06128$


	\begin{itemize}
		\item Lets use the Poisson distribution to approximate a solution.
		\item First check that $n \geq 50$ and $np < 5$ (Yes to both).
		\item We choose as our parameter value $m = np = 1000 \times 0.001 = 1$
	\end{itemize}
	\[P(X = 3) = \frac{e^{-1} \times 1^3}{3!} = \frac{e^{-1}}{6} = \frac{0.36787}{6} = 0.06131 \]
	Compare this answer with the Binomial probability
	P(X = 3) = 0.06128.
	Very good approximation, with much less computation effort.


\end{document}
