
\begin{document}
	
	%---------------------------------------------------------------------------%
	{
		\textbf{Poisson Distribution}
		A Poisson random variable is the number of successes that result from a Poisson experiment.
		
		The probability distribution of a Poisson random variable is called a Poisson distribution.
		
		Given the mean number of successes ($m$) that occur in a specified region, we can compute the Poisson probability based on the following formula:
	}
	
	%---------------------------------------------------------------------------%
	{
		\textbf{The Poisson Probability Distribution}
		\begin{itemize}
			\item The number of occurrences in a unit period (or space)
			\item The expected number of occurrences is $m$
		\end{itemize}
	}
	
	%---------------------------------------------------------------------------%
	{
		\textbf{Poisson Formulae}
		The probability that there will be $k$ occurrences in a unit time period is denoted $P(X=k)$, and is computed as follows.
		\Large
		\[ P(X = k)=\frac{m^k e^{-m}}{k!} \]
		
	}
	%---------------------------------------------------------------------------%
	{
		\textbf{Poisson Formulae}
		Given that there is on average 2 occurrences per hour, what is the probability of no occurences in the next hour? \\ i.e. Compute $P(X=0)$ given that $m=2$
		\Large
		\[ P(X = 0)=\frac{2^0 e^{-2}}{0!} \]
		\normalsize
		\begin{itemize}
			\item $2^0$ = 1
			\item $0!$ = 1
		\end{itemize}
		The equation reduces to
		\[ P(X = 0)=e^{-2} = 0.1353\]
	}
	%---------------------------------------------------------------------------%
	{
		\textbf{Poisson Formulae}
		What is the probability of one occurrences in the next hour? \\ i.e. Compute $P(X=1)$ given that $m=2$
		\Large
		\[ P(X = 1)=\frac{2^1 e^{-2}}{1!} \]
		\normalsize
		\begin{itemize}
			\item $2^1$ = 2
			\item $1!$ = 1
		\end{itemize}
		The equation reduces to
		\[ P(X = 1) = 2 \times e^{-2} = 0.2706\]
	}
	%---------------------------------------------------------------------------%
\section*{Probability of events for a Poisson distribution}
An event can occur 0, 1, 2, … times in an interval. The average number of events in an interval is designated ${\displaystyle \lambda }$   (lambda). Lambda is the event rate, also called the rate parameter. The probability of observing k events in an interval is given by the equation

\[{\displaystyle P(k{\text{ events in interval}})=e^{-\lambda }{\frac {\lambda ^{k}}{k!}}} \]
where

\begin{itemize}
	\item ${\displaystyle \lambda }$  is the average number of events per interval
	\item	e is the number 2.71828... (Euler's number) the base of the natural logarithms
	\item	k takes values 0, 1, 2, …
	\item	$k! = k \times (k − 1) \times (k − 2) \times \ldots \times 2 \times 1$ is the factorial of k.
\end{itemize}

This equation is the probability mass function (PMF) for a Poisson distribution.
%------------------------------------------------------------------%
\frame{
	\frametitle{Poisson Expected Value and Variance}
	
	
	If the random variable X has a Poisson distribution with parameter $m$, we write
	\[ X \sim Poisson(m) \]
	
	
	\begin{itemize}
		\item Expected Value of X : E(X) = m
		\item Variance of X : $\mbox{Var}(X) = m$
		\item Standard Deviation of X : $SD(X) = \sqrt{m}$
	\end{itemize}
}
%------------------------------------------------------------------%
\frame{
	\frametitle{Poisson Distribution : Example} 
	
	\begin{itemize}
		\item The number of faults in a fibre optic cable were recorded for each kilometre length of cable.
		\item The mean number of faults was found to be 4 faults per kilometre.
		\item The standard deviation of the number of faults was found to be 2 faults per kilometre.
		\item Is the Poisson Distribution is a useful technique for modelling the number of faults in fibre optic cable?
		\item (Looking at the last slide, the answer is yes, because the variance and mean are equal). 
	\end{itemize}
	
}
