
\begin{document}

%---------------------------------------------------------------------%
%---------------------------------------------------------------------------%
\begin{frame}
\frametitle{The Poisson Probability Distribution}
Consider cars passing a point on a rarely used country road. Is this a Poisson Random Variable?
Suppose
\begin{enumerate}
\item Arrivals occur at an average rate of $m$ cars per unit time.
\item The probability of an arrival in an interval of length k
is constant.
\item The number of arrivals in two non-overlapping
intervals of time are independent.
\end{enumerate}
This would be an appropriate use of the Poisson Distribution.
\end{frame}

%---------------------------------------------------------------------%
\begin{frame}
\frametitle{Changing the unit time.}

\begin{itemize}
\item The number of arrivals, X, in an interval of length $t$ has a
Poisson distribution with parameter $\mu = mt$.
\item $m$ is the expected number of arrivals in a unit time period.
\item $\mu$ is the expected number of arrivals in a time period $t$, that is different from the unit time period.
\item Put simply : if we change the time period in question, we adjust the Poisson mean accordingly.
\item If 10 occurrences are expected in 1 hour, then 5 are expected in 30 minutes. Likewise, 20 occurrences are expected in 2 hours, and so on.
\item (Remark : we will not use $\mu$ in this context anymore).
\end{itemize}
\end{frame}



%---------------------------------------------------------------------%
\begin{frame}
\frametitle{Knowing which distribution to use}
\begin{itemize}
\item For the end of semester examination, you will be required to know when it is appropriate to use the Poisson distribution, and when to use the binomial distribution.
\item Recall the key parameters of each distribution.
\item Binomial : number of \textbf{\emph{successes}} in $n$ \textbf{\emph{independent trials}}.
\item Poisson : number of \textbf{\emph{occurrences}} in a \textbf{\emph{unit space}}.
\end{itemize}
\end{frame}

\end{document}
%---------------------------------------------------------------------%
\begin{frame}
\frametitle{Poisson Approximation of the Binomial}

\begin{itemize}
\item The Poisson distribution can sometimes be used to approximate the
binomial distribution
\item When the number of observations n is large, and the success probability
p is small, the $Bin(n,p)$ distribution approaches the Poisson distribution
with the parameter given by $m = np$.
\item This is useful since the computations involved in calculating binomial
probabilities are greatly reduced.
\item As a rule of thumb, n should be greater than 50 with p very small, such
that np should be less than 5.
\item If the value of p is very high, the definition of what constitutes a
``success" or ``failure" can be switched.
\end{itemize}
\end{frame}

%---------------------------------------------------------------------%
\begin{frame}
\frametitle{Poisson Approximation: Example}

\begin{itemize}
\item Suppose we sample 1000 items from a production line that is producing, on
average, $0.1\%$ defective components.
\item Using the binomial distribution, the probability of exactly 3 defective items in
our sample is
\[P(X = 3) = ^{1000}C_{3} \times 0.001^{3} \times 0.999^{997}\]
\end{itemize}
\end{frame}

%---------------------------------------------------------------------%
\begin{frame}
\frametitle{Poisson Approximation: Example}
Lets compute each of the component terms individually.

\begin{itemize}
\item $^{1000}C_{3}$
\[^{1000}C_{3} = \frac{1000 \times 999 \times 998}{3 \times 2 \times 1} = 166,167,000\]
\item $0.001^3$
\[0.001^3 = 0.000000001\]
\item $0.999^{997}$
\[0.999^{997} = 0.36880\]
\end{itemize}


Multiply these three values to compute the binomial probability
$P(X = 3) = 0.06128$
\end{frame}

\begin{frame}
\frametitle{Poisson Approximation: Example}
\begin{itemize}
\item Lets use the Poisson distribution to approximate a solution.
\item First check that $n \geq 50$ and $np < 5$ (Yes to both).
\item We choose as our parameter value $m = np = 1000 \times 0.001 = 1$
\end{itemize}
\[P(X = 3) = \frac{e^{-1} \times 1^3}{3!} = \frac{e^{-1}}{6} = \frac{0.36787}{6} = 0.06131 \]
Compare this answer with the Binomial probability
P(X = 3) = 0.06128.
Very good approximation, with much less computation effort.
\end{frame}

\end{document}
