\documentclass[a4paper,12pt]{article}
%%%%%%%%%%%%%%%%%%%%%%%%%%%%%%%%%%%%%%%%%%%%%%%%%%%%%%%%%%%%%%%%%%%%%%%%%%%%%%%%%%%%%%%%%%%%%%%%%%%%%%%%%%%%%%%%%%%%%%%%%%%%%%%%%%%%%%%%%%%%%%%%%%%%%%%%%%%%%%%%%%%%%%%%%%%%%%%%%%%%%%%%%%%%%%%%%%%%%%%%%%%%%%%%%%%%%%%%%%%%%%%%%%%%%%%%%%%%%%%%%%%%%%%%%%%%
\usepackage{eurosym}
\usepackage{vmargin}
\usepackage{amsmath}
\usepackage{graphics}
\usepackage{epsfig}
\usepackage{subfigure}
\usepackage{framed}
\usepackage{enumerate}
\usepackage{fancyhdr}

\setcounter{MaxMatrixCols}{10}
%TCIDATA{OutputFilter=LATEX.DLL}
%TCIDATA{Version=5.00.0.2570}
%TCIDATA{<META NAME="SaveForMode"CONTENT="1">}
%TCIDATA{LastRevised=Wednesday, February 23, 201113:24:34}
%TCIDATA{<META NAME="GraphicsSave" CONTENT="32">}
%TCIDATA{Language=American English}

\pagestyle{fancy}
\setmarginsrb{20mm}{0mm}{20mm}{25mm}{12mm}{11mm}{0mm}{11mm}
\lhead{MS4222} \rhead{Kevin O'Brien} \chead{The Poisson Distribution} %\input{tcilatex}

\begin{document}
%---------------------------------------------------------------------%


\section*{Probability Density Function for the Poisson Distribution}
A discrete random variable $X$ is said to follow a Poisson distribution with parameter $m$, written $X \sim \mbox{Poisson}(m)$.



\begin{itemize}
	\item Here the key parameter, the \textbf{Poisson Mean}, is the expected number of occurence per unit period. 
	\item It is usually denoted as either $m$ or $\lambda$.
\end{itemize}

\noindent Given the mean number of successes ($m$) that occur in a specified region, we can compute the Poisson probability based on the \textit{probability density function} formula.



\begin{framed}
	\noindent \textbf{Poisson Formula}\\
	The probability that there will be $k$ occurrences in a \textbf{unit time period} is denoted $P(X=k)$, and is computed as:
	{
		\large
		\[ P(X = k)=\frac{m^k e^{-m}}{k!} \]
	}
\end{framed}
\noindent where




\begin{itemize} 
	%\itemf(x) the probability of x occurrences in an interval. 
	\item $m$ (or $\lambda$) is the expected value of the mean number of occurrences in any interval (i.e. the Poisson mean),
	\item $e=2.718$,
	\item $k = 0, 1, 2, \ldots$,
	\item $m > 0$.
\end{itemize}




%---------------------------------------------------------------------------%
{
\subsection*{Worked Example 1}

	Given that there is on average 2 occurrences per hour, what is the probability of no occurences in the next hour? \\ Compute $P(X=0)$ given that $m=2$.
	
	\[ P(X = 0)=\frac{2^0 e^{-2}}{0!} \]
	\normalsize
	\begin{itemize}
		\item $2^0$ = 1
		\item $0!$ = 1
	\end{itemize}
	The equation reduces to
	\[ P(X = 0)=e^{-2} = 0.1353\]

\noindent What is the probability of one occurrences in the next hour? \\ Compute $P(X=1)$ given that $m=2$
	
	\[ P(X = 1)=\frac{2^1 e^{-2}}{1!} \]
	\normalsize
	\begin{itemize}
		\item $2^1$ = 2
		\item $1!$ = 1
	\end{itemize}
	The equation reduces to
	\[ P(X = 1) = 2 \times e^{-2} = 0.2706\]
}
%---------------------------------------------------------------------------%



\subsection*{Changing the Unit Time.}

\begin{itemize}
	\item The number of arrivals, X, in an interval of length $t$ has a
	Poisson distribution with parameter $\mu = mt$.
	\item Recall $m$ is the expected number of arrivals in a unit time period.
	\item $\mu$ is the expected number of arrivals in a time period $t$, that is different from the unit time period.
	\item Put simply: if we change the time period in question, we adjust the Poisson mean accordingly.
	\begin{itemize}
		\item[$\ast$] If we double the length of the time period, we double the value of the Poisson mean.
		\item[$\ast$] If we halve the length of the time period, we halve the value of the Poisson mean
	\end{itemize}
	\item If 10 occurrences are expected in 1 hour, then 5 are expected in 30 minutes. Likewise, 20 occurrences are expected in 2 hours, and so on.
	\item (Remark : we will not use $\mu$ in this context anymore.)
\end{itemize}

\begin{framed}
	\begin{itemize}
		% \item If the specified period changes, the Poisson mean changes accordingly.
		\item If the Poisson mean $m=8$ for a 15 minute period, what is the Poisson mean for a unit period of one hour?
		\item The answer is 32.
	\end{itemize}
\end{framed}



\subsection*{Worked Example 2}

Given that there is on average 4 occurrences per day, what is the probability of one occurrences in a given day? \\
Compute $P(X=1)$ given that $m=4$

\[ P(X = 1)=\frac{4^1 e^{-4}}{1!} \]


\noindent The equation reduces to
\[ P(X = 1)=4 \times e^{-4} = {0.07326} \]


\noindent What is the probability of one occurrences in a six hour period ? \\ i.e. Compute $P(X=1)$ given that $m=1$

\[ P(X = 1)=\frac{1^1 e^{-1}}{1!}  = e^{-1} = 0.3678\]


\subsection*{Worked Example 3}

% 2 Marks
% m=2 for 60 Minutes  0.2706706
% m=1 for 30 Minutes  0.3678794

Past experience shows that there, on average, are 2 traffic accidents on a particular stretch of road every week. 
\\
\bigskip
What is the probability of: 
\begin{itemize}
	
	\item Four accidents during a randomly selected week?  
	
	\item No accidents during a randomly selected week?  
\end{itemize}


\begin{itemize}
	\item The Poisson mean $m$ = 2  per week.
	\item (Unit period is 1 week for both questions)
	\item We use this following formula
	\[ P(X=k) =  \frac{e^{-m} \times m ^k}{k!}  \]
	\item Using our value for the Poisson mean
	\[ P(X=k) =  \frac{e^{-2} \times 2^k}{k!} . \]
	
	\item Probability of four accidents during a randomly selected week : $ P(X = 4)$.
	
	\[ P(X=4) =  \frac{e^{-2} \times 2^4}{4!} = \frac{e^{-2} \times 16}{24} = 0.0902\]
	
	\item Probability of no accidents during a randomly selected week : $ P(X = 0)$.
	
	\[ P(X=0) =  \frac{e^{-2} \times 2^0}{0!} = e^{-2} = 0.1353. \]
	
%	\item What is the expected value and standard deviation of the distribution? 
	
\end{itemize}


%==============================================================%

\newpage
\section*{Poisson Expected Value and Variance}


If the random variable X has a Poisson distribution with parameter $m$, we write
\[ X \sim Poisson(m) \]
Note the expected number of occurrences per unit time is often denoted $\lambda$ (lambda) rather than $m$.

% If X is a Poisson distribution stochastic variable with parameter $\lambda$, then
% 
% \begin{itemize}
% \item The expected value $E[X]=\lambda$
% \item The variance $Var[X]=\lambda$
% \end{itemize}




\begin{itemize}
	\item Expected Value of X : $\mbox{E}(X)=  m \mbox{ or } \lambda$
	\item Variance of X : $\mbox{Var}(X) = m \mbox{ or } \lambda$
	\item Standard Deviation of X : $SD(X) = \sqrt{m} \mbox{ or } \sqrt{\lambda}$
\end{itemize}

\noindent Important 
\[ \mbox{E}(X) = \mbox{Var}(X)\]


\end{document}
