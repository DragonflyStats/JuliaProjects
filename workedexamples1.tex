\documentclass[a4paper,12pt]{article}
%%%%%%%%%%%%%%%%%%%%%%%%%%%%%%%%%%%%%%%%%%%%%%%%%%%%%%%%%%%%%%%%%%%%%%%%%%%%%%%%%%%%%%%%%%%%%%%%%%%%%%%%%%%%%%%%%%%%%%%%%%%%%%%%%%%%%%%%%%%%%%%%%%%%%%%%%%%%%%%%%%%%%%%%%%%%%%%%%%%%%%%%%%%%%%%%%%%%%%%%%%%%%%%%%%%%%%%%%%%%%%%%%%%%%%%%%%%%%%%%%%%%%%%%%%%%
\usepackage{eurosym}
\usepackage{vmargin}
\usepackage{amsmath}
\usepackage{graphics}
\usepackage{epsfig}
\usepackage{subfigure}
\usepackage{framed}
\usepackage{enumerate}
\usepackage{fancyhdr}

\setcounter{MaxMatrixCols}{10}
%TCIDATA{OutputFilter=LATEX.DLL}
%TCIDATA{Version=5.00.0.2570}
%TCIDATA{<META NAME="SaveForMode"CONTENT="1">}
%TCIDATA{LastRevised=Wednesday, February 23, 201113:24:34}
%TCIDATA{<META NAME="GraphicsSave" CONTENT="32">}
%TCIDATA{Language=American English}

\pagestyle{fancy}
\setmarginsrb{20mm}{0mm}{20mm}{25mm}{12mm}{11mm}{0mm}{11mm}
\lhead{MS4222} \rhead{Kevin O'Brien} \chead{Exponential Distribution} %\input{tcilatex}

\begin{document}


\section*{Exponential Distribution Example : Phone Call Durations}


Assume that the length of a phone call in minutes is an exponential random variable $X$ with parameter
$\lambda = 1/10$. If someone arrives at a phone booth just before you arrive, find the probability that you
will have to wait \begin{itemize}
\item[(a)] less than 5 minutes,
\item[(b)] greater than 5 minutes,
\item[(c)] between 5 and 10 minutes.
%\item[(d)] Also compute the expected value and variance.
\end{itemize}

\subsection*{Solutions to Part (a)}

Compute $P(X \leq 5)$ with $\lambda = 1/10$

\[ P(X \leq x) = 1-e^{-\lambda x} \]

\begin{itemize}
	\item $ P(X \leq x) = 1-e^{-\lambda x} $
	\item $ P(X \leq 5) = 1-e^{-5/10}  = 1-e^{-0.5}$
	\item $ P(X \leq 5) = 1-0.6065  = 0.3934 $
\end{itemize} 

\subsection*{Solutions to Part (b)}
Compute $P(X \geq 10)$ with $\lambda = 1/10$.

\[ P(X \leq x) = 1-e^{-\lambda x} \]
Complement rule
\[ P(X \geq x) = 1- P(X \leq x) =  e^{-\lambda x} \]

\medskip
\noindent Compute $P(X \geq 10)$ with $\lambda = 1/10$

\[ P(X \geq x) = e^{-\lambda x} \]


\begin{itemize}
	\item $ P(X \geq x) = e^{-\lambda x} $
	\item $ P(X \geq 10) = e^{-10/10}  $
	\item $ P(X \geq 10) = e^{-1} =  0.3678 $

\end{itemize} 


\subsection*{Solutions to Part (c)}


Compute $P(5 \leq X \leq 10)$ with $\lambda = 1/10$  \\
\vspace{0.3cm}
\begin{itemize}
	\item (Important) Probability of being inside this interval is the complement of being outside the interval.
	\item The probability of being outside the interval is the composite event of being too low for the interval (i.e. $P( X \leq 5)$)
	and being too high for the interval (i.e. $P( X \leq 10)$).
\end{itemize}

\[ P(\mbox{Inside}) = 1 - \left[ P(\mbox{Outside})  \right] \]

\[ P(5 \leq X \leq 10) = 1 - \left[ P( X \leq 5) + P( X \geq 10)  \right] \]

\begin{itemize}
\item \textbf{Too Low} $P(X \leq 5)$ = 0.3934
\vspace{0.2cm}
\item \textbf{Too High} $P(X \geq 10)$ = 0.3678
\vspace{0.2cm}
\item \textbf{Outside} $P( X \leq 5) + P( X \geq 10)$ = 0.7612
\vspace{0.2cm}
\item \textbf{Inside} $P(5 \leq X \leq 10)$ = 1 - 0.7612 = 0.2388
\end{itemize}

\subsection*{Expected Value and Variance}

\begin{itemize}
    \item $\operatorname{E}(X) = 1/\lambda = 10$ minutes.
    \item $\operatorname{Var}(X)$ = $1/(\lambda^2)$= 100 squared minutes.
\end{itemize}

\newpage




\section*{Exponential Distribution Example: Laptop Lifetimes}

The average lifespan of a laptop is 5 years. You may assume that
the lifespan of computers follows an exponential probability
distribution. 
\begin{enumerate}[(a)]
\item  What is the
probability that the lifespan of the laptop will be at least 6
years? \item 

What is the probability that the lifespan of the laptop will not
exceed 4 years? \item  What is the probability of the
lifespan being between 5 years and 6 years?
\end{enumerate}
\subsection*{Solution}
Suppose the lifetime of a PC is exponentially distributed with
mean $\mu =5$. We should be told the average lifetime $\mu$.

\[
P( X \geq X_o) = e^{{-X_o \over \mu}}
\]
\[
P( X \leq X_o) = 1 - e^{{-X_o \over \mu}}
\]

\begin{itemize}
\item[(a)] $P(X \geq 6) = e^{-6/5} =  e^{-1.2} = 0.3012$
\item[(b)] $P(X \leq 4) =  1 - e^{-0.8} = 0.5506$
\item[(c)] $P(5 \leq X \leq 6) = 1- [ P( X \leq 5) +  P( X \geq 6)]$ \\ \smallskip
$ P( X \leq 5) = 1 - e^{-1} = 0.6321$\\ \smallskip
$P(5 \leq X \leq 6) = 1 - (0.6321 + 0.3012) \\ = 0.0667\\= 6.67 \%$
\end{itemize}	
%%	\item[(c)] Alternative approach to (b)\\$P(5 \leq X \leq 10)$ \\ = $P( X \geq 5) - P( X \geq 10)$ \\
%%	= $e^{-0.5} - e^{-1}$
%%	=0.6065 - 0.3678\\
%%	= 0.2386 = 23.86 $\%$
%%\end{itemize}

\subsection*{Remark}

Here we are told the exponential mean $\mu$, which is related to the rate parameter as follows:
\[ \mu = \frac{1}{\lambda}\]

\end{document}

