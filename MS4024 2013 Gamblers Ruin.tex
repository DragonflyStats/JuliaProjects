\documentclass[a4paper,12pt]{article}
%%%%%%%%%%%%%%%%%%%%%%%%%%%%%%%%%%%%%%%%%%%%%%%%%%%%%%%%%%%%%%%%%%%%%%%%%%%%%%%%%%%%%%%%%%%%%%%%%%%%%%%%%%%%%%%%%%%%%%%%%%%%%%%%%%%%%%%%%%%%%%%%%%%%%%%%%%%%%%%%%%%%%%%%%%%%%%%%%%%%%%%%%%%%%%%%%%%%%%%%%%%%%%%%%%%%%%%%%%%%%%%%%%%%%%%%%%%%%%%%%%%%%%%%%%%%
\usepackage{eurosym}
\usepackage{vmargin}
\usepackage{amsmath}
\usepackage{graphics}
\usepackage{epsfig}
\usepackage{subfigure}
\usepackage{fancyhdr}
%\usepackage{listings}
\usepackage{framed}
\usepackage{graphicx}

\setcounter{MaxMatrixCols}{10}
%TCIDATA{OutputFilter=LATEX.DLL}
%TCIDATA{Version=5.00.0.2570}
%TCIDATA{<META NAME="SaveForMode" CONTENT="1">}
%TCIDATA{LastRevised=Wednesday, February 23, 2011 13:24:34}
%TCIDATA{<META NAME="GraphicsSave" CONTENT="32">}
%TCIDATA{Language=American English}

\pagestyle{fancy}
\setmarginsrb{20mm}{0mm}{20mm}{25mm}{12mm}{11mm}{0mm}{11mm}
\lhead{MS4024} \rhead{Mr. Kevin O'Brien}
\chead{Numerical Computation}
%\input{tcilatex}

\begin{document}

\tableofcontents
\newpage


\begin{verbatim}
\end{verbatim}
\newpage
\section{Monty Hall Problem}
Suppose that there are three closed doors on the set of the  \textbf{\emph{Let's Make a Deal}}, a TV game show presented by Monty Hall. Behind one of these doors is a car; behind the other two are goats. The contestant does not know where the car is, but Monty Hall does.

The contestant selects a door, but not the outcome is not immediately evident. Monty opens one of the remaining ``wrong" doors. If the contestant has already chosen the correct door, Monty is equally likely to open either of the two remaining doors.

After Monty has shown a goat behind the door that he opens, the contestant is always given the option to switch doors. What is the probability of winning the car if she stays with her first choice? What if she decides to switch?



\subsection{Implementation (part 1)}
We have 3 doors to choose from, so we will define a numeric sequence between 1 and 3. The command \texttt{\textbf{sample(,n)}} takes a sample of size n from a specified set of values. Here we just want to select one door to be our "correct door" and another to be ``selected" door (i.e. the contestant selects)

These events are independent. We will perform the selection for both doors separately, but this can be implemented in one command.


\begin{framed}
\begin{verbatim}
Doors = 1:3
Correct = sample(Doors,1)
Choice = sample(Doors,1)
\end{verbatim}
\end{framed}

A wrong door must be selected to be opened. The door selected by the contestant can't be chosen. First let us select the doors that must stay closed, then find the ones we can choose from to open

\begin{framed}
\begin{verbatim}
StayClosed = union(Correct, Choice)
CanOpen = setdiff(Doors, StayClosed)
\end{verbatim}
\end{framed}

The \texttt{R} command \texttt{\textbf{sample()}} poses an interesting problem. Lets look at the help file to see what it is.

Create a single element vector (let's call it \texttt{\textbf{Vec}}, with that single element being 4. Now sample a single value from that data set. You may expect to get 4 each time, but \texttt{R} doesn’t work that way in this instance.

\begin{framed}
\begin{verbatim}
Vec=c(4)
sample(Vec,1)
\end{verbatim}
\end{framed}

\begin{verbatim}
> sample(Vec,1)
[1] 3
> sample(Vec,1)
[1] 4
> sample(Vec,1)
[1] 1
\end{verbatim}
To overcome this we need some control statements. Now we have the doors we are able to open.

\begin{framed}
\begin{verbatim}
if(length(CanOpen)>1)
{
Open = sample(CanOpen,1)
}else {Open=CanOpen}
\end{verbatim}
\end{framed}


\begin{framed}
\begin{verbatim}
NotOpen = setdiff(Doors,Open)

Stick = Choice
    #The previous statement is to aid the narrative.

Switch = setdiff(NotOpen,Choice)

# Was sticking the right decision? Or was switching?
# The following logicial statements  will tell us.

Stick==Choice
Switch==Choice
\end{verbatim}
\end{framed}


\subsection{Writing a Function}
We are going to create a function \texttt{\textbf{MHfunc()}} to help us simulate a solution for the Monty Hall Problem. The function is constructed using \texttt{R} code we have seen already. The output of the function is returned as a data frame, with three columns:
\begin{description}
\item[Correct] : The number of the correct door.
\item[Choice] :  The door that the contestant chose originally, and the door selected if the contestant decided to ``stick".
\item[Switch] : The door selected if the contestant chose to ``switch".
\end{description}
Necessarily the Correct option must correspond with a value in one of the other two columns.
\begin{framed}
\begin{verbatim}
MHfunc <- function(numdoors=3){
 Doors = 1:numdoors
 Correct = sample(Doors,1)
 Choice = sample(Doors,1)

 StayClosed = union(Correct, Choice)
 CanOpen = setdiff(Doors, StayClosed)

 if(length(CanOpen)>1)
    {
    Open = sample(CanOpen,1) #Problem here
    }else {Open=CanOpen}

 NotOpen = setdiff(Doors,Open)
 Switch = setdiff(NotOpen,Choice)


 MHoutput=c(Correct,Choice,Switch)
 names(MHoutput)= c("Correct","Choice","Switch")
 return(MHoutput)
}
MHfunc()

\end{verbatim}
\end{framed}

No arguments are needed to be passed to the function. The output should appear something like this:
\begin{verbatim}
> MHfunc()
  Correct Choice Switch
1       1      2      1
> MHfunc()
  Correct Choice Switch
1       1      3      1
> MHfunc()
  Correct Choice Switch
1       3      3      1
\end{verbatim}

\subsection{Transforming logical values}

The \texttt{R} command \texttt{\textbf{as.numeric()}} can be use to convert logical values, such as TRUE or FALSE into binary form (i.e. 1 and 0).

\begin{framed}
\begin{verbatim}
LogVec=c(TRUE,FALSE,TRUE)
as.numeric(LogVec)
\end{verbatim}
\end{framed}
\begin{verbatim}
> output = MHfunc()
> output
Correct  Choice  Switch
      1       2       1
> output[1]
Correct
      1
> output[2]
Choice
     2
> output[1]==output[2]
Correct
  FALSE
> output[1]==output[3]
Correct
   TRUE
> as.numeric(output[1]==output[2])
[1] 0
> as.numeric(output[1]==output[3])
[1] 1
   >
\end{verbatim}

\subsection{A Simulation Study of the Monty Hall Problem}

Let us perform a simulation study of the Monty Hall problem. We are putting the code we have written already, placed in a \texttt{\textbf{for}} loop, although we alter the ending to make for more readable output.

\begin{framed}
\begin{verbatim}
StickCount = 0
SwitchCount = 0
M=1000

for(i in 1:M)
 {
 output=MHfunc()
 Correct=output[1]
 Choice=output[2]
 Switch=output[3]
 # Install counters for both “count variables”

 StickCount = StickCount + as.numeric(Choice  ==  Correct)
 SwitchCount = SwitchCount + as.numeric(Switch  ==  Correct)
}
StickCount
SwitchCount
\end{verbatim}
\end{framed}

The output of this program would look like this. The proportion is approximately 2 to 1 in favour of the switching option.
\begin{verbatim}
> StickCount
[1] 323
> SwitchCount
[1] 677
\end{verbatim}
\newpage
%---------------------------------------------------------------%
\section{Gambler's Ruin}

Consider a gambler who starts with an initial fortune of \$1 and then on each successive gamble
either wins \$1 or loses \$1 independent of the past with probabilities p and q = 1-p respectively.

Suppose the gambler has a starting kitty of A.
This gamblers places bets with the �Banker�, who has an initial fortune B. We will look at the game from the perspective of the gambler only.
The Banker is, by convention, the richer of the two.

\begin{itemize}
\item Probability of successful gamble for gambler : p
\item Probability of unsuccessful gamble for gambler : q 	(where q =  1 - p )
\item Ratio of success probability to failure success:	$s = p / q$
\item Conventionally the game is biased in favour of the Banker (i.e. $q>p$ and $s<1$)
\end{itemize}

Let $R_n$ denote the Gambler�s total fortune after the $n-$th gamble.

If the Gambler wins the first game, his wealth becomes $R_n =A+1$.
If he loses the first gamble, his wealth becomes $R_n = A-1$.
The entire sum of money at stake is the �Jackpot� i.e.   $A+B$.
The game ends when the Gambler wins the Jackpot ($R_n = A+B$) or loses everything ($R_n = 0$).


\subsection{Simulation a Single Gamble}


To simulate one single bet, compute a single random number between 0 and 1.
\begin{framed}
\begin{verbatim}
runif(1)
\end{verbatim}
\end{framed}

Lets assume that the game is biased in favour of the Banker
p = 0.45 , q = 0.55.
If the number is less than 0.45, the gamble wins. Otherwise the Banker wins.

\begin{verbatim}
> runif(1)
[1] 0.1251274
>#Gambler Loses
>
> runif(1)
[1] 0.754075
>#Gambler wins
>
> runif(1)
[1] 0.2132148
>#Gambler Loses
>
> runif(1)
[1] 0.8306269
\end{verbatim}

%----------------------------------------------------------------------%
\newpage
Let A be the Gambler's Kitty at the start of the gambling.
Let B be the Banker's Wealth.
The probability of A winning a gamble is p.
The vector $R_n$ records the gambler's worth on an ongoing basis. At the start, The first value is A.
\begin{framed}
\begin{verbatim}

A=20;B=100;p=0.47
Rn=c(A)

probval = runif(1)

if (probval < p)
 {
 A = A+1; B =B-1
 }else{A=A-1;B=B+1}

#Save the values from each bet
Rn=c(Rn,A)
\end{verbatim}
\end{framed}
\newpage
Should the Gambler win the entire jackpot (A+B). The game would also cease. We include a \textbf{\texttt{break}} statement to stop the loop if the gambler wins the entire jackpot. A break statement will stop a loop if a certain logical condition is met.

\begin{framed}
\begin{verbatim}
A=20;B=100;p=0.47
Rn=c(A)
Total=A+B

while(A>0)
  {
  UnifVal=runif(1)

  if(UnifVal <= p)
     {
     A = A+1; B =B-1
     }else{A=A-1;B=B+1}
  Rn=c(Rn,A)
  if(A==Total){break}
  }
\end{verbatim}
\end{framed}
We can construct a plot to depict the gambler's ongoing fortunes in the game. We use a \texttt{for} loop, that implements the game, recording the duration of the game each time. The duration of each game is the dimension of the \texttt{Rn} vector, i.e. \texttt{length(Rn)}.


\begin{framed}
\begin{verbatim}
A.ini=20;B=100;p=0.47
M=1000
RnDist=numeric()
Total=A+B

for (i in 1:M)
    {
    Rn=numeric()
    Rn[1]=A
    A=A.ini
    while(A>0)
        {
       
        UnifVal=runif(1)

        if(UnifVal <= p)
            {
            A = A+1; B =B-1
            }else{A=A-1;B=B+1}
        Rn=c(Rn,A)
        if(A==Total){break}
        }
    RnDist=c(RnDist,length(Rn))
    }
    

\end{verbatim}
\end{framed}
\subsection{Distribution of Durations}
We can use for loop to get a sense of the duration (i.e. the number of rounds a game will last).
\newpage
\end{document}

