\documentclass[a4]{beamer}
\usepackage{amssymb}
\usepackage{graphicx}
\usepackage{subfigure}
\usepackage{newlfont}
\usepackage{amsmath,amsthm,amsfonts}
%\usepackage{beamerthemesplit}
\usepackage{pgf,pgfarrows,pgfnodes,pgfautomata,pgfheaps,pgfshade}
\usepackage{mathptmx}  % Font Family
\usepackage{helvet}   % Font Family
\usepackage{color}

\mode<presentation> {
 \usetheme{Default} % was Frankfurt
 \useinnertheme{rounded}
 \useoutertheme{infolines}
 \usefonttheme{serif}
 %\usecolortheme{wolverine}
% \usecolortheme{rose}
\usefonttheme{structurebold}
}

\setbeamercovered{dynamic}

\title[MA4413]{Statistics for Computing \\ {\normalsize MA4413 Lecture 4B}}
\author[Kevin O'Brien]{Kevin O'Brien \\ {\scriptsize Kevin.obrien@ul.ie}}
\date{Autumn Semester 2012}
\institute[Maths \& Stats]{Dept. of Mathematics \& Statistics, \\ University \textit{of} Limerick}

\renewcommand{\arraystretch}{1.5}

\begin{document}

\begin{frame}
\titlepage
\end{frame}

%---------------------------------------------------------------%
\frame{
\frametitle{Today's Class}
\begin{itemize}
\item Review of Discrete Probability Distributions
\item \texttt{R} Implementation
\item A Few Examples
\item Introduction to Continuous Probability Distributions
\item The Uniform Distribution
\end{itemize}
}
%---------------------------------------------------------------------------%
\frame{
\frametitle{Discrete Probability Distributions}
Three main distributions
\begin{itemize}
\item Binomial Distribution
\item Poisson Distribution
\item Geometric Distribution (mentioned, but not as important as the other two.)
\end{itemize}
}
%---------------------------------------------------------------------------%
\frame{
\frametitle{Binomial Probability Distribution}
Important Points:
\begin{itemize}
\item The experiment is a series of $n$ independent trials.
\item Two possible outcomes from each trial: a success and a failure.
\item The probability of success (i.e. $p$) is constant. 
\item A binomial random variable can be written as 
\[ X \sim B(n,p) \]
\end{itemize}
}
%---------------------------------------------------------------------------%
\frame{
\frametitle{Poisson Probability Distribution}
Important Points:
\begin{itemize}
\item This distribution is concerned with the number of occurrences per unit space.
\item Unit space can mean a unit length, a unit area, a unit volume or a unit period of time.
\item We will concern ourselves with unit time periods mostly.
\item The average number of occurrence per unit period is denoted $m$ i.e. the Poisson mean. In many texts the Poisson mean is denoted $\lambda$. \item If the unit period changes, then the Poisson mean changes accordingly.
\item A Poisson random variable can be written as
\[ X \sim Pois(m) \]
\item The Poisson distribution can be used to approximate the binomial distribution under certain conditions.
\end{itemize}
}
%---------------------------------------------------------------------------%
\frame{
\frametitle{Implementation in \texttt{R}}
Important Points:
\begin{itemize}
\item The binomial parameters $n$ and $p$ are written as \texttt{size} and \texttt{prob}.
\item The Poisson mean $m$ is written as \texttt{lambda}. \bigskip
\item The probability density function (PDF) is the probability of a random variable taking a specific value i.e. $P(X = k)$
\item The appropriate \texttt{R} functions are \texttt{dbinom} and \texttt{dpois}.\bigskip
\item The cumulative distribution function (CDF) is the probability of a random variable not exceeding a specific value i.e. $P(X \leq k)$
\item The appropriate \texttt{R} functions are \texttt{pbinom} and \texttt{ppois}.
\end{itemize}
}

%---------------------------------------------------------------------------%
\begin{frame}[fragile]
\frametitle{Implementation in \texttt{R}}
\begin{itemize}
\item We can compute probabilities for a series of values, rather than just one at a time.
\item For this, we will use sequences using the ``:" operator.
\end{itemize}
\begin{verbatim}
> 2:7
[1] 2 3 4 5 6 7
> dpois(2:7,lambda=3)
[1] 0.22404181 0.22404181 0.16803136 0.10081881 0.05040941
[6] 0.02160403
\end{verbatim}
\begin{itemize}
\item Poisson process with mean $m=3$
\item $P(X=2) =0.224$, $P(X=3) = 0.224$, $P(X=4) = 0.168)$ etc
\end{itemize}
\end{frame}

%---------------------------------------------------------------------------%
\frame{
\frametitle{Geometric Probability Distribution}
Important Points:
\begin{itemize}
\item This distribution is closely related to the binomial distribution.
\item This distribution described the number of failures that occur before the first success, when the probability of success is $p$.
\item The relevant \texttt{R} functions are \texttt{dgeom} and \texttt{pgeom}.
\end{itemize}
}

%---------------------------------------------------------------------------%
\begin{frame}[fragile]
\frametitle{Geometric Probability Distribution : Example}

If the probability of inserting a USB correctly is $0.40$, what is the probability of successfully doing so on the second attempt.\\
\bigskip
In essence we have one failure, then one success, and these are independent events. So the probability the second attempt will be successful is $0.6 \times 0.4$. The probability that we are successful on the first attempt (i.e. no failures beforehand) is 0.4\\
\bigskip
Question: What is the probability of needing more than two attempts?
\begin{verbatim}
> dgeom(0,prob=0.4)
[1] 0.4
> dgeom(1,prob=0.4)
[1] 0.24
> dgeom(2,prob=0.4)
[1] 0.144
\end{verbatim}

\end{frame}




%---------------------------------------------------------------------------%
\begin{frame}[fragile]
\frametitle{Binomial Probability Distribution: Example}

\begin{itemize}
\item Consider a binomial experiment with $n = 20$ and $p = 0.50$.
\item Use the following output to compute $P(X > 10)$
\end{itemize}
\begin{verbatim}
> dbinom(9,size=20,prob=0.50)
[1] 0.1601791
> dbinom(10,size=20,prob=0.50)
[1] 0.1761971
> dbinom(11,size=20,prob=0.50)
[1] 0.1601791
>
> pbinom(9,size=20,prob=0.50)
[1] 0.4119015
> pbinom(10,size=20,prob=0.50)
[1] 0.5880985
> pbinom(11,size=20,prob=0.50)
[1] 0.7482777
\end{verbatim}
\end{frame}

%---------------------------------------------------------------------------%
\begin{frame}[fragile]
\frametitle{Binomial Probability Distribution: Example}

\begin{itemize}
\item Consider the sample space for the number of successes. There can be between 0 and 20 successes.

\[ S = \{0,1,2,3,\ldots,8,9,10,11,12 \ldots,19,20\}
\]

\item  $P(X > 10)$ is the probability of X being greater than 10.
\item This is equivalent to the probability of X being X or greater: i.e. $P(X \geq 11)$.
\item We can't determine this directly using our \texttt{R} output.

\item We can determine the probability of the complementary event. i.e. $P(X \leq 10)$ using calculators.
\end{itemize}
\begin{verbatim}
> pbinom(10,size=20,prob=0.50)
[1] 0.5880985
>
> 1-pbinom(10,size=20,prob=0.50)
[1] 0.4119015
\end{verbatim}
\end{frame}
%---------------------------------------------------------------------------%

%---------------------------------------------------------------------------%
\frame{
\frametitle{Sample Space and Sample Points }

\begin{itemize}
\item Knowledge and use of sample points, and the sample space is quite helpful for these questions.
\item What are the sample points for the event where the number of success is less than or equal to 9? (Lets call this event $A$.)
\[ A = \{0,1,2,3,\ldots,8,9\}
\]
\item What are the sample points for the \textbf{\emph{complement event}}. (Lets call this event $A^c$).
\[ A^c = \{10,11,12 \ldots,19,20\}
\]

\end{itemize}
}




%---------------------------------------------------------------------------%
\frame{
\frametitle{Binomial Probability Distribution: Example 2}
The vice-president of a computer firm has reviewed the records of the firm�s personnel and has found that 70\% of the employees read a well known industry magazine ``The IT Journal". \\ \bigskip
If the vice-president was to choose 10 employees at random, what is the probability that the number of these employees who do not read the ``IT Journal" is the following?
\normalsize
\begin{itemize}
\item [1] At least five.
\item [2] Between four and eight, inclusive.
\item [3] No more than seven.
% \item [4] What are the mean and variance of this distribution?
\end{itemize}
}

%---------------------------------------------------------------------------%
\begin{frame}[fragile]
\frametitle{Binomial Probability Distribution: Example 2}


\begin{itemize}
\item
Firstly, identify the probability distribution to be used?
    \begin{itemize}
    \item
    Answer: the binomial distribution
    \end{itemize}
\item We are given the number of trials ( `` choose 10 employees")

\item We are given a definition of a ``success", which is finding an employee that did NOT reads the journal.

\item We are given the probability of such a success : 30\%  or 0.30

\item So our binomial parameters are n= 10 and p = 0.30

\item Let's use the following \texttt{R} code to solve.

\end{itemize}
\end{frame}
%---------------------------------------------------------------------------%
\begin{frame}[fragile]
\frametitle{Binomial Probability Distribution: Example 2}

\begin{verbatim}
> 0:10
 [1]  0  1  2  3  4  5  6  7  8  9 10
>
> dbinom(0:10,size=10,prob=0.30)
 [1] 0.0282475249 0.1210608210 0.2334744405 0.2668279320
 [5] 0.2001209490 0.1029193452 0.0367569090 0.0090016920
 [9] 0.0014467005 0.0001377810 0.0000059049
>
> pbinom(0:10,size=10,prob=0.30)
 [1] 0.02824752 0.14930835 0.38278279 0.64961072 0.84973167
 [6] 0.95265101 0.98940792 0.99840961 0.99985631 0.99999410
[11] 1.00000000
\end{verbatim}
\end{frame}
%---------------------------------------------------------------------------%
\begin{frame}[fragile]
\frametitle{Binomial Probability Distribution: Example 2}
Question 1: Probability of at least five $P(X \geq 5)$.

\bigskip We have already determined the probability of the complement event $P(X \leq 4)$, which is $84.97\%$. Therefore the answer is $P(X \geq 5)$ = $15.03\%$.

\begin{verbatim}
> pbinom(0:10,size=10,prob=0.30)
 [1] 0.02824752 0.14930835 0.38278279 0.64961072 0.84973167
 [6] 0.95265101 0.98940792 0.99840961 0.99985631 0.99999410
[11] 1.00000000
\end{verbatim}
\end{frame}

%---------------------------------------------------------------------------%
\begin{frame}[fragile]
\frametitle{Binomial Probability Distribution: Example 2}
Question 2: Probability of between 4 and 8 inclusive $P(4 \leq X \leq 8)$.

\bigskip Lets look at the sample space again, with the relevant sample points in bold: $S = \{0,1,2,3,\textbf{4,5,6,7,8},9,10,\}$.
\begin{verbatim}
> pbinom(0:10,size=10,prob=0.30)
 [1] 0.02824752 0.14930835 0.38278279 0.64961072 0.84973167
 [6] 0.95265101 0.98940792 0.99840961 0.99985631 0.99999410
[11] 1.00000000
\end{verbatim}
\begin{itemize}
\item $P(X \leq 8)$ is $99.98\%$, but this includes the probability of $X = \{0,1,2,3\}$
\item We can simply subtract $P(X \leq 3)$ from $P(X \leq 8)$ to get the desired value.
\item $(P 4 \leq X \geq 8)$ = $99.98\%$ - $64.96\%$ = $35.02\%$
\end{itemize}
\end{frame}


%---------------------------------------------------------------------------%
\begin{frame}[fragile]
\frametitle{Binomial Probability Distribution: Example 2}
Question 3: Probability of no more than 7 $P(X \leq 7)$.
\begin{verbatim}
> pbinom(0:10,size=10,prob=0.30)
 [1] 0.02824752 0.14930835 0.38278279 0.64961072 0.84973167
 [6] 0.95265101 0.98940792 0.99840961 0.99985631 0.99999410
[11] 1.00000000
\end{verbatim}
\[ P(X \leq 7) = 99.84\% \]
\end{frame}
%---------------------------------------------------------------------------%



\begin{frame}
\frametitle{Continuous Random variables}
\begin{itemize}
\item Previously we have been studying discrete random variables, such as the Binomial and the Poisson random variables.
\item Now we turn our attention to continuous random variables.
\item Recall that a continuous random variable is one which takes an infinite number of possible values, rather than just a countable number of distinct values.
\item Continuous random variables are usually measurements.
\item Examples include height, weight, the amount of sugar in an orange, the time required to run a mile.
\end{itemize}
\end{frame}
%------------------------------------------------------------------%
\frame{
\frametitle{Exact Probabilities}
\large
\alert{Remarks:} This is for continuous distributions only.
\begin{itemize}
\item The probability that a continuous random variable will take an exact value is infinitely small.
We will usually treat it as if it was zero.
\item
When we write probabilities for continuous random variables in mathematical notation, we often retain the equality component (i.e. the "...or equal to..").\\
For example, we would write expressions $P(X \leq 2)$ or $P(X \geq 5)$.
\item
Because the probability of an exact value is almost zero, these two expression are equivalent to $P(X < 2)$
or $P(X > 5)$. \item The complement of $P(X \geq k)$ can be written as $P(X \leq k)$.
\end{itemize}
}
%----------------------------------------------------------------------------------------------------%
\begin{frame}
\frametitle{Functions and Definite integrals}
\begin{itemize}
\item Integration is not part of the syllabus, and it is assumed that students have are not familiar with how to compute definite integrals.
\item However,  it is useful to know what the purpose of definite integrals are, because we will be using the results derived from definite integrals. \item It is assumed that students are familiar with functions.
\end{itemize}
\end{frame}
%----------------------------------------------------------------------------------------------------%
\begin{frame}
\frametitle{Functions}

\vspace{-0.5cm}

\begin{center}
\includegraphics[scale=0.30]{6AFunction}

\end{center}

Some function $f(x)$ evaluated at $x=1$.
\end{frame}
%----------------------------------------------------------------------------------------------------%
\begin{frame}
\frametitle{Definite Integral}

\vspace{-0.5cm}
\begin{center}
\includegraphics[scale=0.35]{6ADefiniteIntegral}
\end{center}
Definite integral of function is area under curve between X=1 and X=2.
\end{frame}
%----------------------------------------------------------------------------------------------------%
\begin{frame}
\frametitle{Definite Integral}
\begin{itemize}
\item Definite integrals are used to compute the ``area under curves".
\item Definite integrals are defined by a lower and upper limit.
\item The area under the curve between X=1 and  X=2 is depicted in the previous slide.
\item By computing the definite integral, we are able to determine a value for this area.
\item Probability can be represented as an area under a curve.
\end{itemize}
\end{frame}

%----------------------------------------------------------------------------------------------------%
\frame{

\frametitle{Probability Density Function}
\begin{itemize}
\item
In probability theory, a \textbf{\emph{probability density function}} (PDF) (or ``density" for short ) of a continuous random variable is a function that describes the relative likelihood for this random variable to occur at a given point.

\item The pdf for a continuous random variable $X$ is often denoted $f_X(x)$

\item The probability density function can be integrated to obtain the probability that the random variable takes a value in a given interval.

\item The probability for the random variable to fall within a particular interval is given by the integral of this variable's density over the region.

\item The probability density function is non-negative everywhere, and its integral over the entire space is equal to one.
\end{itemize}
}

\begin{frame}

\frametitle{Density Curves}


\begin{itemize}
\item A plot of the PDF is referred to as a `\textbf{\emph{density curve}}'.
\item A density curve that is always on or above the horizontal axis and has total area underneath equal to one.
\item Area under the curve in a range of values indicates the proportion of values in that range.
\item Density curves come in a variety of shapes, but the normal distribution's bell-shaped densities are the commonly used.
\item Remember the density is only an approximation, but it simplifies analysis and is generally accurate enough for practical use.
\end{itemize}
\end{frame}

%----------------------------------------------------------------------------------------------------%
\frame{
\frametitle{The Cumulative Distribution Function }
\begin{itemize}
\item The \textbf{\emph{cumulative distribution function}} (CDF), (or just distribution function), describes the probability that a continuous random variable X with a given probability distribution will be found at a value less than or equal to x.\\

\[ F_X(x) = P(X \leq x) \]

\item Intuitively, it is the ``area so far" function of the probability distribution.
\end{itemize}
}
%----------------------------------------------------------------------------------------------------%
\begin{frame}
\frametitle{Cumulative Distribution Function}

\vspace{-0.5cm}
\begin{center}
\includegraphics[scale=0.35]{6ACDF}

\end{center}
Cumulative Distribution Function $P(Z \leq 1)$.
\end{frame}


%---------------------------------------------------------------------------%
\frame{
\frametitle{Continuous Uniform Distribution}
A random variable X is called a continuous uniform random variable over the interval $(a,b)$ if it's probability density function is given by
\[ f_{X}(x) = { 1 \over b-a} \hspace{1cm} \mbox{ when } a \leq x \leq b \mbox{     (otherwise } f_X(x) = 0 ) \]
The corresponding cumulative density function is
\[ F_x(x) = { x-a \over b-a} \hspace{1cm} \mbox{ when } a \leq x \leq b\]
}
%----------------------------------------------------------------------------------------------------%
\begin{frame}
\frametitle{The Continuous Uniform Distribution}

\vspace{-0.5cm}

\begin{center}
\includegraphics[scale=0.35]{6AUniform}

\end{center}
\end{frame}
%---------------------------------------------------------------------------%
\frame{
\frametitle{Continuous Uniform Distribution}
\begin{itemize}
\item The continuous distribution is very simple to understand and implement, and is commonly used in computer applications (e.g. computer simulation).
\item It is also known as the `Rectangle Distribution' for obvious reasons.
\item We specify the word ``continuous" so as to distinguish it from it's discrete equivalent: the discrete uniform distribution.
\item Remark; the dice distribution is a discrete distribution with lower and upper limits 1 and 6 respectively.
\end{itemize}
}
%-----------------------------------------------------%
\frame{
\frametitle{Uniform Distribution Parameters}


The continuous uniform distribution is characterized by the following parameters

\begin{itemize}
\item The lower limit $a$
\item The upper limit $b$
\item We denote a uniform random variable $X$ as $X \sim U(a,b)$
\end{itemize}

It is not possible to have an outcome that is lower than $a$ or larger than $b$.

\[ P(X < a) = P(X > b) = 0\]
}

%------------------------------------------------------------------------%
\frame{\frametitle{Interval Probability}

\begin{itemize}
\item We wish to compute the probability of an outcome being within a range of values.
\item We shall call this lower bound of this range $L$ and the upper bound $ U$.
\item Necessarily $L$ and $U$ must be possible outcomes.
\item The probability of $X$ being between $L$ and $U$ is denoted $P( L \leq X \leq U )$.

\[
P( L \leq X \leq U ) = { U - L \over b - a}
\]
\item (This equation is based on a definite integral).
\end{itemize}
}

%---------------------------------------------------------------------------------------------------------%
\frame{
\frametitle{Uniform Distribution: Cumulative Distribution}
\begin{itemize}

\item For any value ``c" between the minimum value a and the maximum
value $b$, we can say
\item $P(X \geq c)$ \[P(X \geq c) = {b-c \over b-a}\]
here $b$ is the upper bound while $c$ is the lower bound
\item $P(X \leq c)$ \[P(X \leq c) = {c-a \over b-a}\]
here $c$ is the upper bound while $a$ is the lower bound.
\end{itemize}
}

%-----------------------------------------------------------------------------%
\frame{
\frametitle{Uniform Distribution: Mean and Variance}
\begin{itemize}
\item The mean of the continuous uniform distribution, with parameters $a$ and $b$ is
\[ E(X) = {a+b \over 2}\]
\item The variance is computed as
\[ V(X) = {(b-a)^2 \over 12}\]
\end{itemize}
}
%------------------------------------------------------------------------%
\frame{
\frametitle{Uniform Distribution: Example}

\begin{itemize}
\item Suppose there is a platform in a subway station in a large large city. \item Subway trains arrive \textbf{every three minutes} at this platform. \item What is the shortest possible time a passenger would have to wait for a train?
\item What is the longest possible time a passenger will have to wait?
\end{itemize}

}


%------------------------------------------------------------------------%
\frame{
\frametitle{Uniform Distribution: Example}

\begin{itemize}
 \item What is the shortest possible time a passenger would have to wait for a train?
%\begin{itemize}
\item If the passenger arrives just before the doors close, then the waiting time is zero.
\[ a = 0 \mbox{ minutes } \]
\end{itemize}
}


%------------------------------------------------------------------------%
\frame{
\frametitle{Uniform Distribution: Example}

\begin{itemize}
\item What is the longest possible time a passenger will have to wait?
%\begin{itemize}
\item If the passenger arrives just after the doors close, and missing the train, then he or she will have to wait the full three minutes for the next one.
\[ b = 3 \mbox{ minutes }  = 180 \mbox{ seconds}  \]
\end{itemize}
%\end{itemize}

}

%------------------------------------------------------------------------%
\frame{
\frametitle{Uniform Distribution: Example}

\begin{itemize}
\item What is the longest probability that he will have to wait longer than 2 minutes?
\[ P(X \geq 2)  = {3-2 \over 3-0} = {1/3} = 0.33333   \]
\end{itemize}
%\end{itemize}

}

%----------------------------------------------------------------------------------------------------%
\begin{frame}
\frametitle{The Continuous Uniform Distribution}

\vspace{-0.5cm}

\begin{center}
\includegraphics[scale=0.35]{6AUniform4}

\end{center}
\end{frame}
%------------------------------------------------------------------------%
\frame{
\frametitle{Uniform Distribution: Example}

\begin{itemize}
\item What is the longest probability that he will have to wait less than than 45 seconds (i.e. 0.75 minutes)?
\[ P(X \leq 0.75)  = {0.75 - 0 \over 3-0} = {0.75/3} = 0.250  \]
\end{itemize}
%\end{itemize}

}



%----------------------------------------------------------------------------------------------------%
\begin{frame}
\frametitle{The Continuous Uniform Distribution}

\vspace{-0.5cm}

\begin{center}
\includegraphics[scale=0.35]{6AUniform3}

\end{center}
\end{frame}
%------------------------------------------------------------------------%
\frame{
\frametitle{Uniform Distribution: Expected Value}

We are told that, for waiting times,  the lower limit $a$ is 0, and the upper limit $b$ is 3 minutes. \\ \bigskip The expected waiting time $\textrm{E}[X]$ is computed as follows
\vspace{0.1cm}
\[
\textrm{E}[X] = {b + a \over 2} =  {3 + 0  \over 2}  = 1.5 \mbox{ minutes }
\]

}
%------------------------------------------------------------------------%
\frame{\frametitle{Uniform Distribution: Variance}

The variance of the continuous uniform distribution, denoted $\textrm{V}[X]$,  is  computed using the following formula
\vspace{0.1cm}
\[
\textrm{V}[X] = {(b - a)^2 \over 12}
\]
\vspace{0.1cm}
For our previous example this is
\[
\textrm{V}[X] = {(3 - 0)^2 \over 12} =  {3^2 \over 12} = {9 \over 12} = 0.75
\]
}

\end{document}
%--------------------------------------------------------------------------------------%
\begin{frame}
\frametitle{The Exponential Distribution}
\begin{itemize}
\item The exponential distribution is a continuous probability distribution commonly used to model durations or ``lifetimes".
\item A lifetime could mean
\begin{itemize}
\large
\item the lifespan of a component
\item the time it takes to complete a task
\item the amount of time between two successive occurrences, such as withdrawals from a bank machine.
\end{itemize}
\item The average lifetime is denoted $E(X) = \mu$.
\item The variance of lifetimes is computed as $V(X) = \mu^2$
\end{itemize}
\end{frame}

%--------------------------------------------------------------------------------------%
\frame{
\frametitle{Important Formulae}
\Large
The probability that a lifetime $X$ will be less than a period of $k$ time units is given by
\[
P( X \leq k) = 1- e^{{-k \over \mu}}.
\]
Similarly, the probability that a lifetime $X$ will be greater than a period of $k$ time units is given by
\[
P( X \geq k) = e^{{-k \over \mu}}.
\]
}
%--------------------------------------------------------------------------------------%
\frame{
\frametitle{Sample Question}
\Large
In a large company computer network, there is an average of 40 log-ons to the network per hour.
\begin{enumerate}
\item[1] What is the average amount of time between log-ons?
\item[2] What is the probability that there will be no log-ons for at least 2.4 minutes
\item[3] What is the probability that the next log-on within 1 minutes of the last?
\item[4] What proportions of log-ons occur between 1 minutes and 2.4 minutes of the last log-on?
\end{enumerate}
}
%--------------------------------------------------------------%
\frame{
\frametitle{Solution (Part 1) }

\Large
\begin{itemize} \item What is the average amount of time between log-ons?

\item If there is 40 log-ons in 60 minutes, it is reasonable to think that someone logs on every 1.5 minutes.
\item Therefore $\mu = 1.5$
\end{itemize}

}
%--------------------------------------------------------------------------------------%
\frame{
\frametitle{Solution (Part 2) }
\Large

What is the probability that there will be no log-ons for at least 2.4 minutes?\\
\bigskip
From the formulae:
\[
P( X \geq k) = e^{{-k \over \mu}} .
\]
From the formulae:
\[
P( X \geq 2.4) = e^{{-2.4 \over 1.5}} = e^{-1.6} = 0.2018.
\]
}

%--------------------------------------------------------------------------------------%
\frame{
\frametitle{Solution (Part 3) }
\Large

What is the probability that the next log-on within 1 minutes of the last?\\
i.e. $P(X \leq 1)$
\bigskip
From the formulae:
\[
P( X \leq 1) = 1 - e^{{-1 \over 1.5}} = 1 -  e^{-0.6666}
\]

\[
P( X \leq 1) = 1 -  0.5135  = 0.4865
\]
}

%--------------------------------------------------------------------------------------%
\frame{
\frametitle{Solution (Part 4) }
\Large

What proportions of log-ons occur between 1 minutes and 2.4 minutes of the last log-on?\\
\bigskip
\begin{itemize}
\item \textbf{Too Low} $P(X \leq 1) = 0.4865$\\
\item \textbf{Too High} $P(X \geq 2.4) = 0.2018$\\
\item Probability of being inside interval $P(1 \leq X \leq 2.4) = 0.31152$.
\item $P(1 \leq X \leq 2.4) = 1- ( 0.4865 + 0.2018) = 0.3117$
\end{itemize}
}
\end{document}





\item What is the probability that the lifespan of the laptop will be at least
6 years?
\item What is the probability that the lifespan of the laptop will not exceed
4 years?
\item What is the probability of the lifespan being between 5 years and 6
years?


%----------------------------------------------------------------------------%
\frame{
\frametitle{The Exponential Distribution}
A continuous random variable having p.d.f. f(x), where:
$f(x) = \lambda x e ^{-\lambda x} $
is said to have an exponential distribution, with parameter $\lambda$.
The cumulative distribution is given by:
$F(x) = 1 - e^{\lambda x}$

Expectation and Variance
$E(X) = 1 / \lambda$\\
$V(X) = 1 / \lambda^2$\\
}

%----------------------------------------------------------------------------%
\frame{
\frametitle{Example}
Suppose that the service time for a customer at a fast-food outlet
has an exponential distribution with mean 3 minutes. What is the probability that a
customer waits more than 4 minutes?

\[ P(X  \leq 4) = 1 -  e^{-4/3} \]

\[ P(X  \leq 4) = e^{-4/3} = 0.2636 \]
}


%---------------------------------------------------------------------------------%
\begin{frame}
\frametitle{Exponential Distribution Lifetimes}
The average lifespan of a laptop is 5 years. You may assume that
the lifespan of computers follows an exponential probability
distribution. \begin{itemize}\item (3 marks) What is the
probability that the lifespan of the laptop will be at least 6
years? \item
What is the probability that the lifespan of the laptop will not
exceed 4 years? \item What is the probability of the
lifespan being between 5 years and 6 years?
\end{itemize}
Suppose the lifetime of a PC is exponentially distributed with
mean $\mu =5$
We should be told the average lifetime $\mu$.
\[
P( X \geq x_o) = e^{{-x_o \over \mu}}
\]
\end{frame}


\end{document}

