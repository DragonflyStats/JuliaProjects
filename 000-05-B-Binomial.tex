	\documentclass[a4paper,12pt]{article}
%%%%%%%%%%%%%%%%%%%%%%%%%%%%%%%%%%%%%%%%%%%%%%%%%%%%%%%%%%%%%%%%%%%%%%%%%%%%%%%%%%%%%%%%%%%%%%%%%%%%%%%%%%%%%%%%%%%%%%%%%%%%%%%%%%%%%%%%%%%%%%%%%%%%%%%%%%%%%%%%%%%%%%%%%%%%%%%%%%%%%%%%%%%%%%%%%%%%%%%%%%%%%%%%%%%%%%%%%%%%%%%%%%%%%%%%%%%%%%%%%%%%%%%%%%%%
\usepackage{eurosym}
\usepackage{vmargin}
\usepackage{amsmath}
\usepackage{framed}
\usepackage{graphics}
\usepackage{epsfig}
\usepackage{subfigure}
\usepackage{enumerate}
\usepackage{fancyhdr}

\setcounter{MaxMatrixCols}{10}
%TCIDATA{OutputFilter=LATEX.DLL}
%TCIDATA{Version=5.00.0.2570}
%TCIDATA{<META NAME="SaveForMode"CONTENT="1">}
%TCIDATA{LastRevised=Wednesday, February 23, 201113:24:34}
%TCIDATA{<META NAME="GraphicsSave" CONTENT="32">}
%TCIDATA{Language=American English}

\pagestyle{fancy}
\setmarginsrb{20mm}{0mm}{20mm}{25mm}{12mm}{11mm}{0mm}{11mm}
\lhead{MS4222} \rhead{Kevin O'Brien} \chead{Binomial Distribution} %\input{tcilatex}
\begin{document}

% - http://www.wbs.eu.com/SharedFiles/Maths/statistics%201%20revision/introducing%20binomial.pdf

The Binomial distribution is a discrete distribution used.


The outcome of interest is known as a `success'. If we are
interested in how many times we get a six when a dice is rolled.

The probability of success is denoted $p$.

%============================================================ %
\subsection{Binomial Coefficients}
\[
\left( {\begin{array}{*{20}c} n \\ k \\ \end{array}} \right) =
\frac{{n!}}{{k!\left( {n - k} \right)!}} \]

%============================================================ %

\subsection{Binomial Probability}
\[y = \frac{{n!}}{{k!\left( {n - k} \right)!}}p^k q^{n - k} = \left( {\begin{array}{*{20}c} n \\ k \\ \end{array}} \right)p^k q^{n -
	k}\]

mean and variance of Binomial distribution

$M_{bin} = np$  and $\sigma ^2 _{bin} = np(1-p)$

For example, if the sample size is 12 and the probability of
success is 0.25, the mean is $12 \times 0.25 = 3$ and the variance
is $\sigma ^2 _{bin} = 12 \times 0.25 \times 0.75 = 2.25$.
%=============================================================%
\[E(X) = \mu = np\]


The variance of the binomial distribution is

\[ \operatorname{Var}(X) = \sigma^2 = npq\]

Note: In a binomial distribution, only 2 parameters, namely n and p, are needed to determine the probability.


P(X) gives the probability of successes in n binomial trials.

%http://www.wbs.eu.com/SharedFiles/Maths/statistics%201%20revision/introducing%20binomial.pdf

%==========================================================================================================%
\section{Binomial Distribution}
A Quick Review of the Binomial Distribution
\begin{itemize}
\item The number of independent trials is denoted $n$.
\item The outcome of interest is known as a ``Success".
\item The other outcome is known as a ``failure".  
\item Often the applications of these names is counter-intuitive, i.e. defective components being the ``success".
\item The probability of a `success' is $p$ 
\item The expected number of `successes' from $n$ trials is $E(X) = np$
\item The \texttt{binom} family of commands in \texttt{R} are what we use to compute necessary values.
\end{itemize}




\begin{itemize}
	\item The formula can be understood as follows: we want exactly k successes ($p^k$) and $n - k$ failures ($(1 - p)^{n - k}$).
	\item However, the k successes can occur anywhere among the n trials, and there are ${n \choose k}$ different ways of distributing k successes in a sequence of n trials.
\end{itemize}












%=============================================================%

\section{Binomial Distribution : Worked Example}
\begin{itemize}
\item A manufacturer of hospital equipment knows from experience that 5\% of the production will have some type of minor default, and will require adjustment.

\begin{itemize}
\item Number of independent trials $n$

\item Probability of a "success" $p$

\end{itemize}
\end{itemize}




\textbf{The Binomial Distribution}


\begin{itemize}
	\item The formula can be understood as follows: we want exactly k successes ($p^k$) and n ? k failures $(1 ? p)^{n ? k}$.
	\item However, the k successes can occur anywhere among the n trials, and there are ${n \choose k}$ different ways of distributing k successes in a sequence of n trials.
\end{itemize}


%=================================================%
\newpage
\section{The Binomial Distribution}


The number of ways of choosing x items from n different items with no concern for order.

It is how we calculate the all the numbers of ways we can get x successes from n trials

Remark : 
How many ways are there of getting two heads when a coin is tossed three times?
\[\{HHT,			HTH,			THH\}\]
3 different ways
With a larger number of trials or successes, this is difficult to compute without using the above formula.




%----------------------------------------------%




\textbf{Binomial Distribution: Example 2}
\textbf{Example}




The Binomial distribution is a discrete distribution used.


The outcome of interest is known as a `success'. If we are
interested in how many times we get a six when a dice is rolled.

The probability of success is denoted $p$.

binomial coefficients
\[
\left( {\begin{array}{*{20}c} n \\ k \\ \end{array}} \right) =
\frac{{n!}}{{k!\left( {n - k} \right)!}} \]

%============================================================ %

binomial probability
\[y = \frac{{n!}}{{k!\left( {n - k} \right)!}}p^k q^{n - k} = \left( {\begin{array}{*{20}c} n \\ k \\ \end{array}} \right)p^k q^{n -
	k}\]

mean and variance of Binomial distribution

$M_{bin} = np$  and $\sigma ^2 _{bin} = np(1-p)$



%====================================================================%

\subsection{The Binomial Probability Distribution}



P(X) gives the probability of successes in n binomial trials.



		




The word "success" means that the outcome is the outcome of interest.

If the outcome of interest is something like a flat tire, using the word "success" is coutner intuituive.








%--------------------------------------------------------------------------------------%
{




\end{document}

