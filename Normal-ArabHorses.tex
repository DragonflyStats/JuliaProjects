	\documentclass[a4paper,12pt]{article}
%%%%%%%%%%%%%%%%%%%%%%%%%%%%%%%%%%%%%%%%%%%%%%%%%%%%%%%%%%%%%%%%%%%%%%%%%%%%%%%%%%%%%%%%%%%%%%%%%%%%%%%%%%%%%%%%%%%%%%%%%%%%%%%%%%%%%%%%%%%%%%%%%%%%%%%%%%%%%%%%%%%%%%%%%%%%%%%%%%%%%%%%%%%%%%%%%%%%%%%%%%%%%%%%%%%%%%%%%%%%%%%%%%%%%%%%%%%%%%%%%%%%%%%%%%%%
\usepackage{eurosym}
\usepackage{vmargin}
\usepackage{framed}
\usepackage{amsmath}
\usepackage{graphics}
\usepackage{epsfig}
\usepackage{subfigure}
\usepackage{enumerate}
\usepackage{fancyhdr}

\setcounter{MaxMatrixCols}{10}
%TCIDATA{OutputFilter=LATEX.DLL}
%TCIDATA{Version=5.00.0.2570}
%TCIDATA{<META NAME="SaveForMode"CONTENT="1">}
%TCIDATA{LastRevised=Wednesday, February 23, 201113:24:34}
%TCIDATA{<META NAME="GraphicsSave" CONTENT="32">}
%TCIDATA{Language=American English}

\pagestyle{fancy}
\setmarginsrb{20mm}{0mm}{20mm}{25mm}{12mm}{11mm}{0mm}{11mm}
\lhead{MS4222} \rhead{Kevin O'Brien} \chead{Normal Distribution} %\input{tcilatex}

\begin{document}
\section*{Summary of Normal Distribution Rules}



\begin{framed}
	\begin{itemize}
			\item The Standardisation Formula
			
			\[P(X \leq Xo) = P(Z \leq Zo)	  \]  
			
			when   \[Zo=\frac{Xo- \mu}{\sigma}\]
		\item The Complement Rule
		\[
		P(Z \leq z_o) = 1 - P(Z \geq z_o)
		\]
		\item The Symmetry Rule
		\[
		P(Z \leq -z_o) = P(Z \geq z_o)
		\]
		\item The Interval Rule.
		Where $L$ and $U$ are the lower and upper bounds of an interval.
		\[
		P(L \leq Z \leq U) = P(Z \geq L) -  P(Z \geq U)
		\]
		
	\end{itemize}
\end{framed}
	
\section*{Normal Distribution : Arab Horses Worked Example}
The mass of Arab horses is normally distributed with mean 900 lbs and standard deviation of 50lbs.
\begin{enumerate}[(a)]
\item  Calculate the probability that an Arab horse weighs more than 940 lbs.
\item Calculate the probability than an Arab horse weighs between 880 lbs and 960 lbs.
\end{enumerate}

\noindent \textbf{Solution to Part A}

\begin{itemize}
	\item Let $X$ be mass of Arab horses.
	
	\item We have to find $P(X\leq940)$.  \\          
	\textit{(Remark ``equality component" is included as a formality, but it is not important).}
	
	
	\item	Find the Z value that corresponds to 940 
	
	\[Zo=\frac{Xo-\mu}{\sigma}= \frac{940 -900}{50}= 0.80\]
	
	\[P(X \leq 940) = P(Z \leq 0.80) \]
	
	
	\item	From Murdoch Barnes tables 3, we find that $P(Z \leq 0.8) = 0.2119$
	
	\item 
	Therefore $P(X \leq 940) = 0.2119$
\end{itemize}
\noindent \textbf{Solution to Part B}




\noindent What proportion of horses are between 880 lbs and 960 lbs?
\[P ( 880 \leq X \leq 960).\]
\begin{itemize}
	\item Find out the probability of the complement event.
	\item The complement event is the combination of being too high  or too low for this interval.
	
	\item \textbf{Inside interval:} $P ( 880 \leq X \leq 960).$
	
	\item \textbf{Outside interval:} $P (X \leq 880) + P(X \geq 960)$
	
	\item Complement Rule $P ( 880\leq X \leq 960)  = 1 - [P (X\leq 880) +P(X\geq 960)]$
	
\end{itemize}

\noindent  Find the probability of being too high?

\[Z_o=\frac{X_o-\mu}{\sigma}= \frac{960 -900}{50}= 1.20\]

\[P(X \leq 960) = P(Z \leq 1.20) = 0.1151\]


\noindent Find the probability of being too low?
\[Zo= \frac{Xo-\mu}{\sigma}= \frac{880 -900}{50}= -0.40 \]
\[P(X \leq 880) = P(Z \leq -0.40)  \]

%------------------------------------------------------------- %
\noindent How to compute $P(Z \leq -0.40)$

\noindent Symmetry: 	$P(Z \leq -0.40)$ = $P(Z \geq 0.40)$ = 0.3446


\begin{itemize}
	\item Outside Interval = 0.4596        (0.3446 +  0.1151)
	\item Inside Interval = 0.5404
\end{itemize}


\[P ( 880 \leq X \leq 960)=0.5404 \]
%------------------------------------------------------------- %
\noindent \textbf{Solution to Part C}

\noindent What weight is exceeded by 97.5\% of Arab horses? 

\noindent Find $X_o$  such that $P(X\geq X_o) = 0.975$

\begin{itemize}
	\item 	$P(Z \geq 1.96) = 0.025$ [From Murdoch Barnes Tables] 
	
	\item	$P(Z \leq -1.96) = 0.025$ [Symmetry Rule]
	
	\item	$P(Z \geq -1.96) = 0.975$ [Complement Rule]     
	
	\item	Using this Z-score to find $X_o$
	\[-1.96 = \frac{X_o- 900}{50} \]
	
	
	\item	$X_o= 802$ lbs  [Answer]
\end{itemize}	

\end{document}

